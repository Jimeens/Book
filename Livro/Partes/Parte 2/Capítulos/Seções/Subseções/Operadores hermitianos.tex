    Um fato importante sobre os observáveis é que o valor esperado de um operador que se relaciona diretamente a ele deve sempre ser um número real, isto é, para um dado evento $A$ que está associado a um operador $\hat{\mathcal{A}}$, temos que:
        \begin{equation*}
            \expval*{\hat{\mathcal{A}}}^{\ast} = \expval*{\hat{\mathcal{A}}} \in\mathbb{R}
        \end{equation*}
    
    Como consequência, podemos escrever:
        \begin{align*}
            \expval*{\hat{\mathcal{A}}}^{\ast} = (\bra{\psi}\hat{\mathcal{A}}\ket{\psi})^{\ast} &\eq \bra{\psi}^{\ast}\hat{\mathcal{A}}^{\ast}\ket{\psi}^{\ast} \\
            &\eq (\hat{\mathcal{A}}^{\ast}\ket{\psi}^{\ast})^{\mathtt{T}}(\bra{\psi}^{\ast})^{\mathtt{T}} \\
            &\eq \bra{\psi}\hat{\mathcal{A}}^{\dagger}\ket{\psi}
        \end{align*}
        \begin{align*}
            \expval*{\hat{\mathcal{A}}} = \bra{\psi}\hat{\mathcal{A}}\ket{\psi}
        \end{align*}
    
    Mas então, como $\expval*{\hat{\mathcal{A}}}^{\ast} = \expval*{\hat{\mathcal{A}}}$, podemos concluir que quando um operador está associado a um observável, vale que:
        \begin{answer}\label{eq: operadores hermitianos}
            \hat{\mathcal{A}}^{\dagger} = \hat{\mathcal{A}}
        \end{answer}
    
    Esses operadores são denominados \textit{operadores hermitianos} e são de suma importância para a realização de uma medida em mecânica quântica, de modo que operadores hermitianos são condição necessária para caracterizar um observável.
    
    Vamos supor então que iremos realizar um experimento qualquer no qual pretendemos medir um observável relacionado a um operador $\hat{\mathcal{A}}$, de tal forma que conseguimos determinar o valor esperado desse operador ($\expval*{\hat{\mathcal{A}}} = a$). Nesse experimento, obtemos que todas as medidas são equivalentes ao valor esperado, isto implica diretamente que a variância do valor esperado de $\hat{\mathcal{A}}$ é zero, ou seja:
        \begin{equation*}
            \sigma^2 = \expval*{(\hat{\mathcal{A}} - \expval*{\hat{\mathcal{A}}})^2} = 0 \Rightarrow \bra{\psi}(\hat{\mathcal{A}} - \expval*{\hat{\mathcal{A}}})^2\ket{\psi} = 0
        \end{equation*}
    
    Expandindo essa relação, temos que:
        \begin{equation*}
            \bra{\psi}(\hat{\mathcal{A}} - \expval*{\hat{\mathcal{A}}})^{\dagger}(\hat{\mathcal{A}} - \expval*{\hat{\mathcal{A}}})\ket{\psi} = 0
        \end{equation*}
    
    De tal forma que podemos definir:
        \begin{equation*}
            \bra{\phi} \coloneqq \bra{\psi}(\hat{\mathcal{A}} - \expval*{\hat{\mathcal{A}}})^{\dagger} \qquad \& \qquad 
            \ket{\phi} \coloneqq (\hat{\mathcal{A}} - \expval*{\hat{\mathcal{A}}})\ket{\psi}
        \end{equation*}
    
    Portanto, para que a equação seja verdadeira:
        \begin{equation*}
            \bra{\phi} = 0 \qquad \textrm{ou} \qquad \ket{\phi} = 0
        \end{equation*}
    
    O que implica diretamente em:
        \begin{equation*}
            (\hat{\mathcal{A}} - \expval*{\hat{\mathcal{A}}})\ket{\psi} = 0 \Rightarrow 
            (\hat{\mathcal{A}} - a)\ket{\psi} = 0 \Rightarrow \hat{\mathcal{A}}\ket{\psi} = a\ket{\psi}
        \end{equation*}
        \begin{equation*}
            \bra{\psi}(\hat{\mathcal{A}} - \expval*{\hat{\mathcal{A}}})^{\dagger} = 0 \Rightarrow 
            \bra{\psi}(\hat{\mathcal{A}} - a)^{\dagger} = 0 \Rightarrow 
            \bra{\psi}\hat{\mathcal{A}}^{\dagger} = \bra{\psi}a^{\ast}
        \end{equation*}
    
    Ou seja, independente do caso, caímos em uma equação de autovetores e autovalores, de modo que isso ocorre somente quanto $\ket{\psi}$ for um autovetor de $\hat{\mathcal{A}}$ e $a$ um autovalor de $\hat{\mathcal{A}}$. 
    
    Resumindo, se a variância de um observável for nula (zero) o estado do sistema é um autoestado do operador relacionado ao observável implicando que:
        \begin{answer}\label{eq: autoestados}
                \hat{\mathcal{A}}\ket{\psi} = a\ket{\psi}
        \end{answer}
    
    \begin{theorem}{Ortogonalidade de operadores hermitianos}{}
        Seja $\hat{\mathcal{A}}$ um operador hermitiano arbitrário, com autovalores $a_{1},a_{2}\in\mathbb{R}$ e dois autovetores (autoestados) $\ket{1},\ket{2}\in\mathscr{H}$, onde $\mathscr{H}$ é um espaço de Hilbert arbitrário. Os autoestados deste operador são sempre ortogonais, ou seja
            \begin{equation*}
                \braket{1}{2} = \braket{2}{1} = 0
            \end{equation*}
    \end{theorem}
    
    \begin{proof}
        Tomemos então os seguintes resultados:
            \begin{equation}\label{eq: 1}
                \bra{1}\underbrace{\hat{\mathcal{A}}\ket{2}}_{a_{2}\ket{2}} = \bra{1}a_{2}\ket{2} = a_{2}\braket{1}{2}
            \end{equation}
            
            \begin{equation}\label{eq: 2}
                \bra{2}\underbrace{\hat{\mathcal{A}}\ket{1}}_{a_{1}\ket{1}} = \bra{2}a_{1}\ket{1} = a_{1}\braket{2}{1}
            \end{equation}
        \begin{note}{}
            Aplicando o fato de ser hermitiano: $\overset{h}{=}$.
        \end{note}
        
        Calculando então o conjugado transposto de \eqref{eq: 1}:
            \begin{equation}\label{eq: 3}
                (\bra{1}\hat{\mathcal{A}}\ket{2})^{\ast} = \bra{2}\hat{\mathcal{A}}^{\dagger}\ket{1} \overset{h}{=} \bra{2}\hat{\mathcal{A}}\ket{1} = a_{1}\braket{2}{1}
            \end{equation}
        
        Vemos então que se $\hat{\mathcal{A}}$ for hermitiano:
            \begin{equation*}
                (\bra{1}\hat{\mathcal{A}}\ket{2})^{\ast} = \bra{2}\hat{\mathcal{A}}\ket{1}
            \end{equation*}
        
        Então aplicando \eqref{eq: 1} em \ref{eq: 3}:
            \begin{equation*}
                (a_{2}\braket{1}{2})^{\ast} = a_{1}\braket{2}{1} \Rightarrow 
                a_{2}\braket{2}{1} = a_{1}\braket{2}{1}
            \end{equation*}
            \begin{equation*}
                (a_{2} - a_{1})\braket{2}{1} = 0
            \end{equation*}
        Caso $a_{2} = a_{1}$, temos um caso trivial, pois se isso ocorrer teremos que $\ket{1} = \ket{2}$, que claramente é um caso irrelevante.
        Portanto, se $a_{2}\neq a_{1}$, temos obrigatoriamente que $\braket{2}{1} = 0$, ou seja os autoestados são ortogonais.
    \end{proof}

    \begin{corollary}\label{cor: base de autoestados}
        O conjunto de autoestados $\{\ket{i}\}$, com $i=1,2$, pode ser utilizado como uma base
    \end{corollary}
    
    Vamos supor agora que tenhamos um estado $\ket{\phi}$ e um observável relacionado a um dado operador $\hat{\mathcal{A}}$ de tal forma que temos um conjunto de autoestados $\ket{a_{i}}$ e autovalores $a_{i}$. Escrevendo $\ket{\phi}$ em termos dos autoestados:
        \begin{equation*}
            \ket{\phi} = \sum_{i}\phi_{i}\ket{a_{i}}
        \end{equation*}
    
    Com isso, queremos saber como determinar $\expval*{\hat{\mathcal{A}}}$ no estado $\ket{\phi}$:
        \begin{align*}
            \expval*{\hat{\mathcal{A}}} &\eq \bra{\phi}\hat{\mathcal{A}}\ket{\phi} 
            = \sum_{i}\phi_{i}^{\ast}\bra{a_{i}}\hat{\mathcal{A}}\sum_{j}\phi_{j}\ket{a_{j}} \\
            &\eq \sum_{i,j}\phi_{i}^{\ast}\phi_{j}\bra{a_{i}}\hat{\mathcal{A}}\ket{a_{j}} 
            = \sum_{i,j}\phi_{i}^{\ast}\phi_{j}\bra{a_{i}}a_{j}\ket{a_{j}} \\
            &\eq \sum_{i,j}\phi_{i}^{\ast}\phi_{j}a_{j}\braket{a_{i}}{a_{j}} 
            = \sum_{i,j}\phi_{i}^{\ast}\phi_{j}a_{j}\delta_{ij} \\
            &\eq \sum_{i}\phi_{i}^{\ast}\phi_{i}a_{i}
        \end{align*}
        \begin{answer*}
            \expval*{\hat{\mathcal{A}}} = \sum_{i}\abs{\phi_{i}}^2a_{i}
        \end{answer*}
        
    Imaginemos então que o estado $\ket{\phi}$ esteja normalizado, ou seja $\braket{\phi}{\phi} = 1$:
        \begin{align*}
            \braket{\phi}{\phi} &\eq \sum_{i}\phi_{i}^{\ast}\bra{a_{i}}\sum_{j}\phi_{j}\ket{a_{j}} \\
            &\eq \sum_{i,j}\phi_{i}^{\ast}\phi_{j}\braket{a_{i}}{a_{j}} \\
            &\eq \sum_{i,j}\phi_{i}^{\ast}\phi_{j}\delta_{ij} \\
            &\eq \sum_{i}\phi_{i}^{\ast}\phi_{i}
        \end{align*}
        \begin{answer*}
            \braket{\phi}{\phi} = 1 = \sum_{i}\abs{\phi_{i}}^2
        \end{answer*}
    
    Isso nos dá uma ideia de média ponderada pelas probabilidades. Voltando diretamente ao processo de medidas, temos que quando fazemos uma única medida, vamos obter uma resposta dentre os possíveis valores de um observável, que são os \textit{autovalores}. Ou seja, o estado $\ket{\psi}$ se transforma no autoestado correspondente ao autovalor obtido.
    