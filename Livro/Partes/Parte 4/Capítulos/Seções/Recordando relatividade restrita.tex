A relatividade restrita é um conceito da física que pode ser compactada em dois postulados principais
    \begin{myitemize}
        \item As leis da física são invariantes em relação a mudanças entre referenciais inerciais;
        \item A velocidade da luz independe da velocidade da fonte de luz.
    \end{myitemize}

O conteúdo destes postulados gera algumas consequências fundamentais para toda teoria, algumas delas são: o espaço e o tempo não são absolutos, o espaço pode ser contraído e o tempo dilatado, a simultaneidade de eventos é diferente dependendo do referencial, etc. E para entender melhor sobre esses postulados e suas consequências, é necessário relembrar sobre as transformações de Lorentz.