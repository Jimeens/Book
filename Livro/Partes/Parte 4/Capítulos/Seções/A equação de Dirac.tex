O fato da equação de Klein--Fock--Gordon ser de segunda ordem no tempo faz com que $\rho$ possa assumir valores negativos, como vimos anteriormente. Ao estudar sobre a teoria relativística dos elétrons, Dirac não ficou satisfeito com esta possibilidade, mesmo que houvessem ``remendos'' interpretativos em relação ao significado de $\rho < 0$. Além disso, a equação de Klein--Fock--Gordon não explica a existência do spin do elétron.

Afim de tentar solucionar estes problemas, Dirac buscou obter uma equação relativística de primeira ordem no tempo que incluísse soluções para diferentes spins do elétron. O que ele buscava era uma equação que satisfizesse as transformações de Lorentz, ou seja, se a equação é de primeira ordem no tempo, então ela também deve ser de primeira ordem no espaço, pois apenas assim as transformações serão satisfeitas.

Para encontrar tal equação, Dirac postulou uma hamiltoniana que fosse linear nas derivadas espaciais, mas que mantivesse a hermiticidade que a própria equação de Schrödinger possui. Esta equação é dada por
    \begin{equation}\label{eq: Dirac proposals}
        \hat{\mathcal{H}}\psi(\vb{r},t) = \hat{\boldsymbol{\alpha}}\cdot\vb{p} \psi(\vb{r},t) + \hat{\beta} m \psi(\vb{r},t)
    \end{equation}
onde $\psi(\vb{r},t)$ é um estado quântico, $\vb{\alpha}$ e $\beta$ operadores a serem determinados que devem ser ao mesmo tempo, independentes de $(\vb{p},t)$ por linearidade e independente de $(\vb{r},t)$ pela homogeneidade do espaço-tempo e $\vb{p}$ denota o vetor momento em 3 dimensões espaciais. Além disso, os operadores a serem determinados são adimensionais e devemos respeitar a equação \eqref{eq: dispersion relation}, tal que 
    \begin{equation}\label{eq: needs to satisfy this Dirac}
        \hat{\mathcal{H}}^{2} \psi(\vb{r},t) = (p^{2} + m^{2})\psi(\vb{r},t)
    \end{equation}

Podemos reescrever a equação \eqref{eq: Dirac proposals} na forma
    \begin{equation*}
        \hat{\mathcal{H}}\psi(\vb{r},t) = \sum_{i=1}^{3}\hat{\alpha}_{i}p_{i}\psi(\vb{r},t) + \hat{\beta} m \psi(\vb{r},t)
    \end{equation*}
de modo que ao fazer esta equação ao quadrado, temos a expansão
    \begin{align*}
        \hat{\mathcal{H}}^{2}\psi(\vb{r},t) &\eq \Bigg[
            \sum_{i=1}^{3}\hat{\alpha}_{i}^{2}p_{i}^{2} + 
            \sum_{i=1}^{3}(\hat{\alpha}_{i}\hat{\beta} + \hat{\beta}\hat{\alpha}_{i})p_{i}m + \\
        &\noeq
            \sum_{i=1}^{3}\sum_{j\neq i}(\hat{\alpha}_{i}\hat{\alpha}_{j} + \hat{\alpha}_{j}\hat{\alpha}_{i})p_{i}p_{j} + 
            \hat{\beta}^{2} m^{2}
        \Bigg]\psi(\vb{r},t) \\
        (\color{myLColor}\text{deve ser}) &\eq (p^{2} + m^{2})\psi(\vb{r},t)
    \end{align*}

Comparando então termo-a-termo, temos que
    \begin{align*}
        \sum_{i=1}^{3}\hat{\alpha}_{i}^{2}p_{i}^{2} \overset{!}{=} p^{2} &{\color{myLColor}\ \Rightarrow\ } \hat{\alpha}_{i}^{2} = 1 \\
        \sum_{i=1}^{3}(\hat{\alpha}_{i}\hat{\beta} + \hat{\beta}\hat{\alpha}_{i})p_{i}m \overset{!}{=} 0 &{\color{myLColor}\ \Rightarrow\ } \hat{\alpha}_{i}\hat{\beta} + \hat{\beta}\hat{\alpha}_{i} = \{\hat{\alpha}_{i},\hat{\beta}\} = 0 \\
        \sum_{i=1}^{3}\sum_{j\neq i}(\hat{\alpha}_{i}\hat{\alpha}_{j} + \hat{\alpha}_{j}\hat{\alpha}_{i})p_{i}p_{j} \overset{!}{=} 0 &{\color{myLColor}\ \Rightarrow\ } \hat{\alpha}_{i}\hat{\alpha}_{j} + \hat{\alpha}_{j}\hat{\alpha}_{i} = \{\hat{\alpha}_{i},\hat{\alpha}_{j}\} = 0 \\
        \hat{\beta}^{2}m^{2} \overset{!}{=} m^{2} &{\color{myLColor}\ \Rightarrow\ } \hat{\beta}^{2} = 1
    \end{align*}
e a partir destas equações/condições, é evidente que os operadores $\hat{\alpha}_{i}$ e $\hat{\beta}$ não podem ser números, mas sim \textbf{matrizes}. Matrizes estas que satisfazem relações de anti-comutação entre $\hat{\alpha}_{1},\ \hat{\alpha}_{2},\ \hat{\alpha}_{3},\ \hat{\alpha}_{3}$ e $\hat{\beta}$, além de que
    \begin{equation*}
        \hat{\alpha}_{1}^{2} = \hat{\alpha}_{2}^{2} = \hat{\alpha}_{3}^{2} = \hat{\beta} = \boldone
    \end{equation*}

    Para que tais condições sejam satisfeitas, os operadores $\hat{\alpha}_{i}$ e $\hat{\beta}$ devem ser matrizes hermitianas, de traço nulo e auto-valores $\pm 1$, cuja menor dimensão possível destas matrizes é 4, onde a demonstração deste último pode ser vista em detalhes em \textcite{Schiff}. Para demonstrar as outras 3 propriedades, precisamos de uma representação para esses operadores e nestas notas, a representação que será utilizada é a representação de Dirac-Pauli, dada por
        \begin{equation*}
            (\hat{\boldsymbol{\alpha}}) = \begin{pmatrix}
                0 & \hat{\boldsymbol{\sigma}} \\
                \hat{\boldsymbol{\sigma}} & 0
            \end{pmatrix} \qquad \& \qquad 
            (\hat{\beta}) = \begin{pmatrix}
                \boldone_{2\times 2} & 0 \\
                0 & -\boldone_{2\times 2}
            \end{pmatrix}
        \end{equation*}
    onde $\hat{\boldsymbol{\sigma}} = (\sigma_{1},\sigma_{2},\sigma_{3})$ são as matrizes de Pauli e $\boldone_{2\times2}$ a matriz identidade de dimensão 2.







    

    Agora, sabendo que podemos representar os operadores $\hat{\mathcal{H}} \mapsto i\pdv{}{t}$ e $\vb{p} = -i\vb{\nabla}$, temos que a equação \eqref{eq: Dirac proposals} se reescreve na forma
        \begin{equation*}
            i \pdv{\psi(\vb{r},t)}{t} = -i\hat{\boldsymbol{\alpha}}\cdot\vb{\nabla}\psi(\vb{r},t) + \beta m \psi(\vb{r},t)
        \end{equation*}

    Multiplicando ambos os lados por $\hat{\beta}$ à esquerda e usando o fato de que $\hat{\beta}^{2} = \boldone$, temos
        \begin{equation*}
            i\hat{\beta}\pdv{\psi(\vb{r},t)}{t} = -i\hat{\beta}\boldsymbol{\alpha}\cdot\vb{\nabla}\psi(\vb{r},t) + \boldone m\psi(\vb{r},t)
        \end{equation*}
    e portanto, sendo $\boldone m \mapsto m$ agora uma matriz com $\diag{m} = (m,m,m,m)$,
        \begin{equation*}
            \qty(i\hat{\beta}\pdv{}{t} + i\hat{\beta}\boldsymbol{\alpha}\cdot\vb{\nabla} - m)\psi(\vb{r},t) = 0
        \end{equation*}

    Definindo então o quadrivetor
        \begin{equation*}
            \gamma^{\mu} = (\hat{\beta}, \hat{\beta}\hat{\boldsymbol{\alpha}})
        \end{equation*}
    onde $\gamma^{\mu}$ será um quadrivetor com seus elementos sendo matrizes de dimensão 4, denominadas ``matrizes gamma'', podemos obter de fato a equação de Dirac em sua forma covariante:
        \begin{answer}\label{eq: Dirac equation 1}
            (i\gamma^{\mu}\partial_{\mu} - m)\psi(\vb{r},t) = 0
        \end{answer}

    Ou, através da notação ``slash'' de Feynman, onde ele define para um quadrivetor $A_{\mu}$ a seguinte notação: $\slashed{A} \coloneqq \gamma^{\mu}A_{\mu}$. Implicando em
        \begin{answer}\label{eq: Dirac equation 2}
            (i\slashed{\partial} - m)\psi(\vb{r},t) = 0
        \end{answer}

    Note então que a única forma da equação de Dirac ser satisfeita é que $\psi(\vb{r},t)$ seja um objeto de 4 componentes na forma
        \begin{equation*}
            \psi(\vb{r},t) \equiv \begin{pmatrix}
                \psi_{1}(\vb{r},t) \\
                \psi_{2}(\vb{r},t) \\
                \psi_{3}(\vb{r},t) \\
                \psi_{4}(\vb{r},t)
            \end{pmatrix}
        \end{equation*}

    Com tais informações em mente, o que podemos aprender com essa equação? Uma primeira forma de responder essa pergunta é analisando a solução de um dos casos mais simples possível: partículas livres. 

    Para uma partícula livre, podemos escrever simplesmente que
        \begin{equation}\label{eq: free particle for Dirac}
            \psi(\vb{r},t) = u e^{i(\vb{p}\cdot\vb{r} - Et)}
        \end{equation}
    porém, agora, $\psi(\vb{r},t)$ é um objeto com 4 componentes, o que nos dá
        \begin{equation*}
            \psi(\vb{r},t) = \begin{pmatrix}
                u_{1} \\
                u_{2} \\
                u_{3} \\
                u_{4}
            \end{pmatrix} e^{i(\vb{p}\cdot\vb{r} - Et)}
        \end{equation*}
    onde $u$ é um objeto de 4 componentes chamado \textbf{espinor}.

    Tomando como base a equação \eqref{eq: Dirac proposals} para facilitar a análise, temos
        \begin{align*}
            i\pdv{\psi(\vb{r},t)}{t} &\eq -i\hat{\boldsymbol{\alpha}}\cdot\vb{\nabla}\psi(\vb{r},t) + \hat{\beta}m\psi(\vb{r},t) \\
            i(-iE)\psi(\vb{r},t) &\eq -i\hat{\boldsymbol{\alpha}}\cdot(i\vb{p})\psi(\vb{r},t) + \hat{\beta} m \psi(\vb{r},t) 
        \end{align*}
    e portanto
        \begin{equation}\label{eq: eigenenergy free particle Dirac equation}
            Eu = (\hat{\boldsymbol{\alpha}}\cdot\vb{p} + \hat{\beta} m)u
        \end{equation}

    Olhando então para soluções de partículas no limite não-relativístico, ou seja $p\ll m$, temos
        \begin{equation*}
            Eu = \hat{\beta} m u = 
            \begin{pmatrix}
                \boldone_{2\times 2} m & 0 \\
                0 & -\boldone_{2\times 2} m
            \end{pmatrix} u = 
            \begin{pmatrix}
                m & 0 & 0 & 0 \\
                0 & m & 0 & 0 \\ 
                0 & 0 & -m & 0 \\
                0 & 0 & 0 & -m
            \end{pmatrix}u
        \end{equation*}

    É evidente então o fato de que tempos 4 auto-valores de energia $E = +m,\ +m,\ -m,\ -m$ e consequentemente 4 auto-estados
        \begin{equation*}
            u_{1} = \begin{pmatrix}
                1 \\ 0 \\ 0 \\ 0
            \end{pmatrix} \qquad \& \qquad 
            u_{2} = \begin{pmatrix}
                0 \\ 1 \\ 0 \\ 0
            \end{pmatrix} \qquad \& \qquad 
            u_{3} = \begin{pmatrix}
                0 \\ 0 \\ 1 \\ 0
            \end{pmatrix} \qquad \& \qquad 
            u_{4} = \begin{pmatrix}
                0 \\ 0 \\ 0 \\ 1
            \end{pmatrix}
        \end{equation*}
    onde associamos os 2 primeiros espinores $u_{1}$ e $u_{2}$ à energia $E = +m$, em que $u_{1}$ representa o espinor de uma partícula de energia $E = m$ com spin positivo e $u_{2}$ representa o espinor de uma partícula de energia $E = m$ com spin negativo. Analogamente, $u_{3}$ representa o espinor de uma partícula de energia $E = -m$ com spin positivo e $u_{4}$ o espinor de uma partícula de energia $E = -m$ com spin negativo.

    Agora passando para o caso relativístico, onde $p \sim m$, temos através da equação \eqref{eq: eigenenergy free particle Dirac equation} que
        \begin{equation*}
            Eu = \begin{pmatrix}
                \boldone_{2\times 2}m & \hat{\boldsymbol{\sigma}}\cdot\vb{p} \\
                \hat{\boldsymbol{\sigma}}\cdot\vb{p} & -\boldone_{2\times 2}m
            \end{pmatrix}u
        \end{equation*}

    Sendo então
        \begin{equation*}
            u = \begin{pmatrix}
                u_{p} \\ u_{d}
            \end{pmatrix}
        \end{equation*}
    podemos determinar os valores dos estados $u_{p}$ e $u_{d}$ por
        \begin{equation*}
            \begin{pmatrix}
                \boldone_{2\times 2}E & 0 \\
                0 & \boldone_{2\times 2}
            \end{pmatrix}
            \begin{pmatrix}
                u_{p} \\ u_{d}
            \end{pmatrix} = 
            \begin{pmatrix}
                \boldone_{2\times 2}m & \hat{\boldsymbol{\sigma}}\cdot\vb{p} \\
                \hat{\boldsymbol{\sigma}}\cdot\vb{p} & -\boldone_{2\times 2}m
            \end{pmatrix}
            \begin{pmatrix}
                u_{p} \\ u_{d}
            \end{pmatrix}
        \end{equation*}
    implicando em 
        \begin{equation*}
            \begin{cases}
                \boldone_{2\times 2}(E - m)u_{p} - \hat{\boldsymbol{\sigma}}\cdot \vb{p} u_{d} = 0 \\
                -\hat{\boldsymbol{\sigma}}\cdot\vb{p}u_{p} + \boldone_{2\times 2}(E + m)u_{d} = 0
            \end{cases}
        \end{equation*}

    A segunda equação nos dá que
        \begin{equation}\label{eq: equation for ud}
            u_{d} = \dfrac{\hat{\boldsymbol{\sigma}}\cdot\vb{p}}{E+m}u_{p}
        \end{equation}

    Substituindo isto na primeira equação e usando o fato de que $(\hat{\boldsymbol{\sigma}} - \vb{p})^{2} = p^{2}$, obtemos para $u_{p}$ que
        \begin{equation*}
            \boldone_{2\times 2}\qty[E^{2} - \qty(m^{2} + p^{2})]u_{p} = 0
        \end{equation*}
    onde as soluções possíveis são: o caso trivial de $u_{p} = 0$ e 
        \begin{equation*}
            E^{2} - (m^{2} + p^{2}) = 0 \Rightarrow E = \pm \sqrt{p^{2} + m^{2}}
        \end{equation*}

    O que é condizente com o que obtivemos utilizando a equação de Klein-Fock-Gordon! Olhando particularmente para o caso em que $E>0$ (onde o caso $E<0$ é análogo), podemos escolher por conveniência
        \begin{equation*}
            u_{p} = \begin{pmatrix}
                1 \\ 0
            \end{pmatrix} \qquad \text{ou} \qquad 
            u_{p} = \begin{pmatrix}
                0 \\ 1
            \end{pmatrix}
        \end{equation*}
    e utilizar a equação \eqref{eq: equation for ud} para obter duas formas para $u_{d}$:
        \begin{equation*}
            u_{d} = \dfrac{\hat{\boldsymbol{\sigma}}\cdot\vb{p}}{E+m}\begin{pmatrix}
                1 \\ 0
            \end{pmatrix} \qquad \text{ou} \qquad 
            u_{d} = \dfrac{\hat{\boldsymbol{\sigma}}\cdot\vb{p}}{E+m}\begin{pmatrix}
                0 \\ 1
            \end{pmatrix}
        \end{equation*}
    em que, como $\hat{\boldsymbol{\sigma}}\cdot\vb{p} = \sigma_{1}{p_{1}} + \sigma_{2}{p_{2}} + \sigma_{3}{p}_{3}$, temos explicitamente
        \begin{equation*}
            \hat{\boldsymbol{\sigma}}\cdot\vb{p} = \begin{pmatrix}
                {p}_{3} & {p}_{1} - ip_{2} \\
                {p}_{1} + i{p}_{2} & -{p}_{3}
            \end{pmatrix}
        \end{equation*}

    Concluindo então que
        \begin{equation*}
            u_{d} = \dfrac{1}{E+m}\begin{pmatrix}
                {p}_{3} \\
                {p}_{1} + i{p}_{2}
            \end{pmatrix} \qquad \text{ou} \qquad 
            u_{d} = \dfrac{1}{E+m}\begin{pmatrix}
                {p}_{1} - i{p}_{2} \\
                -{p}_{3}
            \end{pmatrix}
        \end{equation*}
    ou seja, as soluções de $u$ para $E > 0$ são
        \begin{equation*}
            u = \dfrac{1}{E+m}\begin{pmatrix}
                E+m \\ 0 \\ p_{3} \\ p_{1} + ip_{2}
            \end{pmatrix} \qquad \text{ou} \qquad 
            u = \dfrac{1}{E+m}\begin{pmatrix}
                0 \\ E+m \\ p_{1} - ip_{2} \\ -p_{3}
            \end{pmatrix}
        \end{equation*}

    Lembrando então da equação \eqref{eq: free particle for Dirac}, temos para $E > 0$ que os estados possíveis para partícula livre são
        \begin{equation*}
            \psi(\vb{r},t) = \dfrac{e^{i(\vb{p}\cdot\vb{r} - Et)}}{E+m}\begin{pmatrix}
                E+m \\ 0 \\ p_{3} \\ p_{1} + ip_{2}
            \end{pmatrix} \qquad \text{ou} \qquad 
            \psi(\vb{r},t) = \dfrac{e^{i(\vb{p}\cdot\vb{r} - Et)}}{E+m}\begin{pmatrix}
                0 \\ E+m \\ p_{1} - ip_{2} \\ -p_{3}
            \end{pmatrix}
        \end{equation*}

    No caso de $E < 0$, temos que
        \begin{equation*}
            u = \dfrac{1}{E-m}\begin{pmatrix}
                p_{3} \\
                p_{1} + ip_{2} \\
                E-m \\
                0
            \end{pmatrix} \qquad \text{ou} \qquad 
            u = \dfrac{1}{E-m}\begin{pmatrix}
                p_{1} - ip_{2} \\
                -p_{3} \\
                0 \\
                E - m
            \end{pmatrix}
        \end{equation*}
    e portanto os possíveis estados possíveis para partícula livre são
        \begin{equation*}
            \psi(\vb{r},t) = \dfrac{e^{i(\vb{p}\cdot\vb{r} - Et)}}{E-m}\begin{pmatrix}
                p_{3} \\
                p_{1} + ip_{2} \\
                E-m \\
                0
            \end{pmatrix} \qquad \text{ou} \qquad 
            \psi(\vb{r},t) = \dfrac{e^{i(\vb{p}\cdot\vb{r} - Et)}}{E-m}\begin{pmatrix}
                p_{1} - ip_{2} \\
                -p_{3} \\
                0 \\
                E - m
            \end{pmatrix}
        \end{equation*}

    Para determinar a equação da continuidade através da equação de Klein-Fock-Gordon, tomamos o conjugado complexo das equações e desenvolvemos as contas. No caso da equação de Dirac, o fato de termos matrizes e vetores, nos leva a fazer algo análogo a tomar o conjugado complexo, ou seja, tomar os hermitianos conjugados!

    A partir da definição das matrizes gamma e da construção dos operadores $\hat{\alpha}_{i}$ e $\hat{\beta}$, temos que
        \begin{equation}\label{eq: conjugated hermitian for gamma0}
            \gamma^{0} = \beta = \begin{pmatrix}
                \boldone_{2\times 2} & 0 \\
                0 & -\boldone_{2\times 2}
            \end{pmatrix} \Rightarrow (\gamma^{0})^{\dagger} = \beta = \gamma^{0}
        \end{equation}
        \begin{equation}
            \gamma^{i} = \hat{\beta}\hat{\alpha}_{i} = \begin{pmatrix}
                0 & \sigma_{i} \\
                -\sigma_{i} & 0
            \end{pmatrix} \Rightarrow (\gamma^{i})^{\dagger} = -\gamma^{i},\ i = 1,2,3
        \end{equation}

    Tendo então estas equações, temos primeiro que a equação de Dirac expandida é
        \begin{equation*}
            i\gamma^{0} \pdv{\psi(\vb{r},t)}{t} + i\sum_{k=1}^{3} \gamma^{k}\pdv{\psi(\vb{r},t)}{x^{k}} - m\psi(\vb{r},t) = 0
        \end{equation*}
    e em segundo que a equação de Dirac conjugada é
        \begin{equation*}
            -i\pdv{\psi^{\dagger}(\vb{r},t)}{t}\gamma^{0} - i\sum_{k=1}^{3} {\pdv{\psi^{\dagger}(\vb{r},t)}{t}} (-\gamma^{k}) - m\psi^{\dagger}(\vb{r},t) = 0
        \end{equation*}

    O sinal negativo em $(-\gamma^{k})$ é inconveniente se compararmos as duas equações, de modo que para resolver isso, multiplicamos por $\gamma^{0}$ pela direita e usemos o fato de que $\gamma^{k}\gamma^{0} = -\gamma^{0}\gamma^{k}$, tal que ficamos com
        \begin{equation*}
            -i\pdv{\psi^{\dagger}(\vb{r},t)}{t}\; \gamma^{0}\gamma^{0} - i\sum_{k=1}^{3} {\pdv{\psi^{\dagger}(\vb{r},t)}{t}}\gamma^{0}\gamma^{k} - m\psi^{\dagger}(\vb{r},t)\gamma^{0} = 0
        \end{equation*}

    Se definirmos $\bar{\psi}(\vb{r},t) \coloneqq \psi^{\dagger}(\vb{r},t)\gamma^{0}$, obtemos
        \begin{equation*}
            -i\pdv{\bar{\psi}(\vb{r},t)}{t}\gamma^{0} - \sum_{k=1}^{3}\pdv{\bar{\psi}(\vb{r},t)}{x^{k}}\gamma^{k} - m\bar{\psi}(\vb{r},t) = 0
        \end{equation*}
    que pode ser escrita de forma mais compacta como
        \begin{equation*}
            i\partial_{\mu}\bar{\psi}(\vb{r},t)\gamma^{\mu} + m \bar{\psi}(\vb{r},t) = 0
        \end{equation*}

    Multiplicando a equação de Dirac por $\bar{\psi}(\vb{r},t)$ e esta última equação por $\psi(\vb{r},t)$, temos, respectivamente
        \begin{equation*}
            i\bar{\psi}(\vb{r},t)\gamma^{\mu}\partial_{\mu}\psi(\vb{r},t) - m\bar{\psi}(\vb{r},t)\psi(\vb{r},t) = 0
        \end{equation*}
        \begin{equation*}
            i\partial_{\mu}\bar{\psi}(\vb{r},t)\gamma^{\mu}\psi(\vb{r},t) + m\bar{\psi}(\vb{r},t)\psi(\vb{r},t) = 0
        \end{equation*}
    e somando ambas,
        \begin{equation*}
            \bar{\psi}(\vb{r},t)\gamma^{\mu}[\partial_{\mu}\psi(\vb{r},t)] + [\partial_{\mu}\bar{\psi}(\vb{r},t)]\gamma^{\mu}\psi(\vb{r},t) = 0
        \end{equation*}

    Se definirmos $j^{\mu} \coloneqq \bar{\psi}(\vb{r},t)\gamma^{\mu}\psi(\vb{r},t)$, temos pela regra da cadeia que
        \begin{equation*}
            \partial_{\mu}j^{\mu} = \bar{\psi}(\vb{r},t)\gamma^{\mu}[\partial_{\mu}\psi(\vb{r},t)] + [\partial_{\mu}\bar{\psi}(\vb{r},t)]\gamma^{\mu}\psi(\vb{r},t) 
        \end{equation*}

    Ou seja, a equação acima é uma equação da continuidade
        \begin{answer}
            \partial_{\mu}j^{\mu} = 0
        \end{answer}
    para $j^{\mu} \coloneqq \bar{\psi}(\vb{r},t)\gamma^{\mu}\psi(\vb{r},t)$. Mas qual o impacto deste resultado? Tendo que $\rho = j^{0}$, temos também que
        \begin{align*}
            \rho &\eq \bar{\psi}(\vb{r},t)\gamma^{0}\psi(\vb{r},t) \\
            &\eq \psi^{\dagger}(\vb{r},t)\underbrace{\gamma^{0}\gamma^{0}}_{\boldone}\psi(\vb{r},t) \\
            &\eq \psi^{\dagger}(\vb{r},t)\psi(\vb{r},t) \\
            &\eq |\psi(\vb{r},t)|^{2}
        \end{align*}
    implicando então em
        \begin{answer}\label{eq: rho is a probability density again}
            \rho = j^{0} = |\psi(\vb{r},t)|^{2}
        \end{answer}
    ou seja, resgatamos a ideia de que $\rho$ é sempre positivo, o que vai equivaler a uma densidade de probabilidade, resolvendo o problema que a equação de Klein-Fock-Gordon possuía!

    \begin{note}{}
        Normalmente definimos $j^{\mu} \coloneqq \pm e\bar{\psi}(\vb{r},t)\gamma^{\mu}\bar{\psi}(\vb{r},t)$, onde $+e$ seria para partículas e $-e$ para antipartículas.
    \end{note}

    \begin{example}[Partícula de Dirac em um campo eletromagnético]
        Antes de tratar de fato as equações, salientamos aqui que faremos a abordagem utilizando o potencial vetor $\vb{A}(\vb{r})$ (associado ao campo magnético por $\vb{B}(\vb{r}) = \vb{\nabla}\times\vb{A}(\vb{r})$) e o potencial escalar $\phi(\vb{r})$ (associado ao campo elétrico por $\vb{E}(\vb{r}) = -\vb{\nabla}V(\vb{r})$). Além disso, fazemos um deslocamento no vetor momento $\hat{\vb{p}} \mapsto \hat{\vb{p}} - q\vb{A}$, pois esta mudança é invariante de gauge e nos permite determinar o momento cinético da partícula de forma mais exata. 

        Com estas ideias em mente, podemos passar de fato ao problema em questão. Utilizando as mudanças enunciadas, a equação \eqref{eq: Dirac proposals} se altera para forma
            \begin{equation*}
                \hat{\mathcal{H}}\psi(\vb{r},t) = \qty[\hat{\boldsymbol{\alpha}}\cdot(\hat{\vb{p}} - q\vb{A}) + \hat{\beta}m + q\phi]\psi(\vb{r},t)
            \end{equation*}
        que se estende para equação \eqref{eq: eigenenergy free particle Dirac equation} para assumir a forma
            \begin{equation}\label{eq: eigenenergy for a dirac particle}
                Eu = \qty[\hat{\boldsymbol{\alpha}}\cdot(\hat{\vb{p}} - q\vb{A}) + \hat{\beta}m + q\phi]u
            \end{equation}

        Sendo então
            \begin{equation*}
                \psi(\vb{r},t) \equiv \begin{pmatrix}
                    \psi_{u}(\vb{r},t) \\ \psi_{d}(\vb{r},t)
                \end{pmatrix}
            \end{equation*}
        podemos reescrever a equação \eqref{eq: eigenenergy for a dirac particle} por
            \begin{equation*}
                E \begin{pmatrix}
                    \psi_{u} \\ \psi_{d}
                \end{pmatrix} = \begin{pmatrix}
                    0 & \hat{\boldsymbol{\sigma}} \\
                    \hat{\boldsymbol{\sigma}} & 0
                \end{pmatrix}[\hat{\vb{p}} - q\vb{A}]
                \begin{pmatrix}
                    \psi_{u} \\ \psi_{d}
                \end{pmatrix} + 
                \begin{pmatrix}
                    m+q\phi & 0 \\
                    0 & -m+q\phi
                \end{pmatrix}
                \begin{pmatrix}
                    \psi_{u} \\
                    \psi_{d}
                \end{pmatrix}
            \end{equation*}

        Formamos então um sistema de equações tal que
            \begin{equation*}
                \begin{cases}
                    E\psi_{d} = \hat{\boldsymbol{\sigma}}\cdot(\hat{\vb{p}} - q\vb{A})\psi_{d} + (m + q\phi)\psi_{u} \\
                    E\psi_{u} = \hat{\boldsymbol{\sigma}}\cdot (\hat{\vb{p}} - q\vb{A})\psi_{u} - (m - q\phi)\psi_{d}
                \end{cases}
            \end{equation*}

        A partir da segunda equação, temos 
            \begin{equation*}
                \psi_{d}(\vb{r},t) = \dfrac{\hat{\boldsymbol{\sigma}}\cdot(\hat{\vb{p}} - q\vb{A})}{E - q\phi + m}\psi_{u}(\vb{r},t)
            \end{equation*}
        de modo que colocando esta expressão na primeira equação, acabamos com
            \begin{equation*}
                E\psi_{u}(\vb{r},t) = \qty{
                    \dfrac{[\hat{\boldsymbol{\sigma}}\cdot(\hat{\vb{p}} - q\vb{A})]^{2}}{E - q\phi + m} + m+q\phi
                }\psi_{u}(\vb{r},t)
            \end{equation*}

        No limite não-relativístico, ou seja, para $E\sim m$ e $E \gg q\phi$, temos que o denominador $E-q\phi+m \approx 2m$ e portanto
            \begin{equation*}
                E\psi_{u}(\vb{r},t) = \qty{
                    \dfrac{[\hat{\boldsymbol{\sigma}}\cdot(\hat{\vb{p}} - q\vb{A})]^{2}}{2m} + m + q\phi
                }\psi_{u}(\vb{r},t)
            \end{equation*}

        Utilizando a relação
            \begin{equation*}
                (\hat{\boldsymbol{\sigma}}\cdot\vb{u})(\hat{\boldsymbol{\sigma}}\cdot\vb{v}) = \vb{u}\cdot\vb{v} + 
                i\hat{\boldsymbol{\sigma}}\cdot(\vb{u}\times\vb{v})
            \end{equation*}
        obtemos
            \begin{align*}
                [\hat{\boldsymbol{\sigma}}\cdot(\hat{\vb{p}} - q\vb{A})]^{2} &\eq 
                [\hat{\boldsymbol{\sigma}}\cdot(\hat{\vb{p}} - q\vb{A})][\hat{\boldsymbol{\sigma}}\cdot(\hat{\vb{p}} - q\vb{A})] \\
                &\eq (\hat{\vb{p}} - q\vb{A})^2 + i\hat{\boldsymbol{\sigma}}\cdot(q\hat{\vb{p}}\times\vb{A} + q\vb{A}\times\hat{\vb{p}}) \\
                &\eq (\hat{\vb{p}} - q\vb{A})^2 + iq\hat{\boldsymbol{\sigma}}(\hat{\vb{p}}\times\vb{A} + \vb{A}\times\hat{\vb{p}})
            \end{align*}
        implicando em
            \begin{equation*}
                E\psi_{u}(\vb{r},t) = \qty[
                    \dfrac{(\hat{\vb{p}} - q\vb{A})^2}{2m} + \dfrac{iq\hat{\boldsymbol{\sigma}}}{2m}\cdot (\hat{\vb{p}}\times\vb{A} + \vb{A}\times\hat{\vb{p}}) + m + q\phi
                ]\psi_{u}(\vb{r},t)
            \end{equation*}

        Podemos simplificar esta expressão usando $\hat{\vb{p}} = -i\vb{\nabla}$ no termo intermediário, tal que
            \begin{align*}
                \hat{\vb{p}}\times\vb{A} + \vb{A}\times\hat{\vb{p}} &\eq -i\vb{\nabla}\times\vb{A} -i\vb{A}\times\vb{\nabla} \\
                &\eq -i\vb{B} - i\vb{A}\times\vb{\nabla}
            \end{align*}

        Note, porém, que tudo isto é aplicado a $\psi_{u}(\vb{r},t)$, de modo que ao tomarmos $\vb{A}\times\vb{\nabla}\psi_{u}(\vb{r},t)$ temos um termo identicamente nulo, pois $\vb{\nabla}\psi_{u}(\vb{r},t)$ vai ser paralelo ao potencial vetor, o que indica que a expressão se resume a
            \begin{equation*}
                E\psi_{u}(\vb{r},t) = \qty[
                    \dfrac{(\hat{\vb{p}} - q\vb{A})^2}{2m} + \dfrac{q\hat{\boldsymbol{\sigma}}}{2m}\cdot \vb{B} + m + q\phi
                ]\psi_{u}(\vb{r},t)
            \end{equation*}

        Neste limite não-relativístico, $E = m + E'$, logo 
            \begin{equation*}
                E'\psi_{u}(\vb{r},t) = \qty[\dfrac{1}{2m}(\hat{\vb{p}} - q\vb{A})^2 + q\phi]\psi_{u}(\vb{r},t) + \dfrac{q}{2m}(\hat{\boldsymbol{\sigma}}\cdot\vb{B})\psi_{u}(\vb{r},t)
            \end{equation*}

        Analisando esta expressão, temos no primeiro termo a estrutura da hamiltoniana não-relativística para partículas em um campo eletromagnético e no segundo termo uma quantidade que relaciona as matrizes de Pauli com o campo magnético. Então, definindo o spin como sendo $\hat{\vb{S}} = \dfrac{1}{2}\hat{\boldsymbol{\sigma}}$, caímos em
            \begin{equation*}
                E'\psi_{u}(\vb{r},t) = \qty[\dfrac{1}{2m}(\hat{\vb{p}} - q\vb{A})^2 + q\phi + \dfrac{q}{m}(\hat{\vb{S}}\cdot\vb{B})]\psi_{u}(\vb{r},t)
            \end{equation*}
        e assim vemos que a quantidade que conhecemos como spin aparece de forma automática, basicamente como uma propriedade puramente quântica intrínseca à característica relativística das partículas.
    \end{example}