Afim de construir uma definição formal dos números complexos, utilizamos como referência o trabalho de diversos grandes nomes como Bombelli\footnote{Rafael Bombelli (1526--1572).}, Cardano\footnote{Gerolamo Cardano (1501--1576).}, Descartes\footnote{René Descartes (1596--1650).}, Gau\ss\footnote{Johann Carl Friedrich Gaus\ss (1777--1855).}, Euler\footnote{Leonhard Euler (1707--1783).} e muitos outros nomes que contribuíram no desenvolvimento da teoria. Começamos inicialmente definindo o que são \textit{números imaginários}:
    \begin{definition}\label{def: numeros imaginarios}
        (\textbf{Números imaginários}) Um número imaginário é definido em termos de uma unidade imaginária definida por
            \begin{answer}\label{eq: unidade imaginaria}
                i^2 = -1
            \end{answer}
        de modo que todo número imaginário é aquele cujo quadrado é negativo.
    \end{definition}

Em prol desta definição, definimos o que são de modo geral números complexos.
    \begin{definition}\label{def: numeros complexos}
        (\textbf{Números complexos}) Um número $z$ é dito ser complexos se para quaisquer $a,b\in\mathbb{R}$ ele puder ser escrito como
            \begin{answer}\label{eq: numero complexo}
                z \coloneqq a + ib
            \end{answer}

        O conjunto de todos os números complexos é denotado por $\mathbb{C}$. Pode-se também definir o \textit{complexo conjugado} deste número, tal que
            \begin{answer}\label{eq: complexo conjugado}
                z^{\ast} \equiv \bar{z} \coloneqq a - ib
            \end{answer}

        E por fim, o módulo de um número complexo é da forma
            \begin{answer}\label{eq: modulo}
                |z|^2 \coloneqq \mathfrak{Re}^2(z) + \mathfrak{Im}^2(z) = a^2 + b^2
            \end{answer}
    \end{definition}
    
    \begin{note}{}
        Em termos de notação, $a$ é dito ser a parte real de $z$, comumente denotado por $\mathfrak{Re}(z) = a$ e $b$ é dito ser a parte imaginária de $z$, comumente denotado por $\mathfrak{Im}(z) = b$.
    \end{note}
    
    \begin{corollary}\label{cor: contencao dos reais em C}
        Como um número real $x\in\mathbb{R}$ é tal que a parte imaginária é nula, ou seja $\mathfrak{Im}(x) = 0$, vale a relação de contenção $\mathbb{R}\subset\mathbb{C}$.
    \end{corollary}
    \begin{corollary}\label{cor: modulo}
        Em termos das definições, constata-se que $|z|^2 = z^{\ast}z$
    \end{corollary}
    \begin{proof}
        A demonstração é simples:
            \begin{align*}
                z^{\ast}z &\eq (a-ib)(a+ib) = a^2 + iab - iab - i^2b^2 \\
                &\eq a^2 + b^2
            \end{align*}
        logo
            \begin{equation*}
                |z|^2 = z^{\ast}z
            \end{equation*}
    \end{proof}

Além destas definições e conceitos básicos, existem algumas propriedades básicas que podem ser úteis em diversos casos e cujas demonstrações são deixadas como exercícios ao final do capítulo.
    \begin{properties}
        Sejam $z,w\in\mathbb{C}$ dois números complexos quaisquer. Constata-se que:
        \begin{enumerate}[(1)]
            \item $z+w = w+z$; 
            \hfill (comutatividade da adição)
            \item $zw = wz$; 
            \hfill (comutatividade da multiplicação)
            \item $(z+w)^{\ast} = z^{\ast} + w^{\ast}$; 
            \hfill (conjugação é distributiva sob adição)
            \item $(zw)^{\ast} = z^{\ast}w^{\ast}$; 
            \hfill (conjugação é distributiva sob multiplicação)
            \item $z^{\ast}z = zz^{\ast} = |z|^2$;
            \hfill (conjugação é comutativa sob multiplicação)
            \item $(z^{\ast})^{\ast} = z$;
            \hfill (conjugação é uma involução)
            \item $|z| = |z^{\ast}|$; 
            \hfill (módulo invariante sob conjugação)
            \item $|zw| = |z||w|$; 
            \hfill (módulo é distributivo sob multiplicação)
            \item $|z+w|\leqslant |z| + |w|$; 
            \hfill (desigualdade triangular)
            \item $\dfrac{1}{z} = \dfrac{z^{\ast}}{|z|^2},\ z\neq 0+i0$.
            \hfill (inverso multiplicativo)
        \end{enumerate}
    \end{properties}

    