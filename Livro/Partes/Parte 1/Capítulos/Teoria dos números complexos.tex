\epigraph{Complex numbers, as much as reals, and perhaps even more, find a unity with nature that is truly remarkable. It is as though Nature herself is as impressed by the scope and consistency of the complex-number system as we are ourselves, and has entrusted to these numbers the precise operations of her world at its minutest scales.}{{\scshape Penrose, R. (2016)}}

O conjunto dos números complexos abriga inúmeras ferramentas matemáticas de grande utilidade para física, especialmente para a mecânica quântica. Isso talvez soe como um fato curioso, tendo em mente que estamos tratando de fenômenos físicos, em que podemos realizar medidas experimentalmente. Além disso ao pensarmos em números imaginários, surgem diversas confusões, pois o que estamos acostumados no dia--a--dia, são essencialmente números pertencentes o conjunto dos reais $\mathbb{R}$.

No entanto, nem sempre os números reais são suficientes, veja por exemplo a equação $x^2 + 1 = 0$. Ao tentarmos buscar por raízes, encontramos inconsistências, pois as soluções para tal seriam $\pm\sqrt{-1}$, mas estes valores não são reais por diversos motivos, um deles é o fato de que a função raiz quadrada definida nos reais só assume entradas $\geqslant 0$, o que já exclui por completo a possibilidade das raízes da equação serem reais. E é neste momento que os números complexos surgem como ferramentas fundamentais para solucionar o problema.