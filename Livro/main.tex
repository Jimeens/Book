\documentclass{header}

\begin{document}

\capa{Fundamentos Teóricos de Mecânica Quântica}{Lucas R. Ximenes}{Felipe Gimenez S.}{Alexandre A. Suaide}

\bintro{Fundamentos Teóricos de Mecânica Quântica} % Título do Livro
{1\textordfeminine\ Ed.} % Edição
{Com o propósito de proporcionar uma introdução inicial à mecânica quântica, este livro didático aborda de forma abrangente os conceitos fundamentais da disciplina. Partindo desde os princípios básicos, como espaços de Hilbert, e avançando para ferramentas mais avançadas, como a teoria de perturbação (independente e dependente do tempo), teoria de espalhamento e um breve capítulo sobre mecânica quântica relativística. O texto abrange desde resultados teóricos cruciais até exemplos experimentais da teoria, como o Experimento de Stern-Gerlach e experimentos relacionados à detecção de neutrinos.

Dirigindo-se a estudantes que tenham uma compreensão de nível de graduação em física, este livro busca ser acessível a uma ampla gama de leitores interessados em explorar o campo da mecânica quântica.

O livro oferece uma variedade de exemplos ao longo do texto e inclui inúmeros exercícios e leituras adicionais ao final de cada capítulo. Isso permite aos estudantes aprimorar e praticar os métodos apresentados. Baseando-se em cursos de graduação em mecânica quântica, os autores se esforçam para apresentar os conceitos de maneira didática, mantendo o formalismo e a elegância da teoria. Isso atende às necessidades dos alunos que desejam aprender, seja de forma independente ou como parte de um curso de mecânica quântica.} % Uma breve introdução sobre o que o livro aborda
{Descrição do Jimeens de 3 ou 4 linhas.} % Uma breve descrição do Jimeens
{Descrição do Felps de 3 ou 4 linhas.} % Uma breve descrição do Felps
{Descrição do Suaide de 3 ou 4 linhas.} % Uma breve descrição do Jimeens
{Fundamentos Teóricos de\\ \medskip Mecânica Quântica}


\pagenumbering{roman}

\frontmatter

\pagestyle{toc}

\tableofcontents


% \thispagestyle{fancy}
\addcontentsline{toc}{chapter}{\itshape Prefácio}
\chapter*{\hspace{0.8cm}Prefácio}

    \lipsum[1-6]

\addcontentsline{toc}{chapter}{\itshape Abreviações e notações}
\chapter*{\hspace{0.8cm}Abreviações e notações}

    \begin{itemize}
        \item Vetores são escritos em \textbf{negrito}: $\vb{v}$
        \item Vetores unitários (versores) são denotados por $\vb{e}_{i}$
        \item Produtos internos (fora da notação de Dirac) são escritos por $\expval{\cdot,\cdot}$
        \item Operadores são sempre escritos com um acento circunflexo: $\hat{\mathcal{A}}$
        \item Operador ou matriz unitária é da forma $\boldone$, em particular, algumas vezes enfatiza-se a dimensão $d$ por $\boldone_{d\times d}$
        \item O conjunto dos naturais é $\mathbb{N} = \{1,2,3,...\}$ e o conjunto dos naturais incluindo o zero é denotado por $\mathbb{N}_{0}$
        \item Produtos vetoriais e cartesianos são denotados com $\times$
        \item Produtos tensoriais são denotados com $\otimes$
        \item Coordenadas cartesianas são denotadas por $(x,y,z)$
        \item Coordenadas cilíndricas são denotadas por $(r,\phi,z)$
        \item Coordenadas esféricas são denotadas por $(r,\phi,\theta)$
        \item Complexos conjugados são denotados por $z^{\ast}$
        \item O símbolo de Levi--Civita é $\epsilon_{ijk} = \begin{cases}
            +1 &, \text{ permutação par de }ijk \\
            -1&, \text{ permutação ímpar de }ijk \\
            0 &, \text{ índices iguais}
        \end{cases}$
        \item O delta de Kronecker é $\delta_{ij} = \begin{cases}
            +1 &, i = j \\
            0  &, i \neq j
        \end{cases}$
        \item O delta de Dirac é $\delta(x-x') = \begin{cases}
            \infty &, x=x' \\
            0 &, x\neq x'
        \end{cases}$
        \item Equivalências são denotadas por $\equiv$
        \item Definições são denotadas por $\coloneqq$
        \item Mapeamentos entre elementos de um conjunto são denotados por $\mapsto$
        \item Mapeamentos entre conjuntos são denotados por $\rightarrow$
        \item Valores esperados são denotados por $\expval{\cdot}$
        \item Alguns fatores de conversão de unidades:
            \begin{itemize}
                \item 1 eV $= 1.602\cdot 10^{-19}$ J
                \item 1 J $= 6.242\cdot 10^{18}$ eV
                \item 1 \AA\ $= 10^{-10}$ m
            \end{itemize}
        \item Alguns valores de constantes físicas:
            \begin{itemize}
                \item $c = 2.998\cdot 10^{8}$ m/s 
                \hfill (velocidade da luz no vácuo)
                \item $e = 1.602\cdot 10^{-19}$ C 
                \hfill (carga do elétron)
                \item $h = 6.626\cdot 10^{-34}$ J$\cdot$s 
                \hfill (constante de Planck) 
                \item $\hbar = h/2\pi$\hfill (constante de Planck reduzida) \\
                $\phantom{\hbar} = 1.055\cdot 10^{-34}$ J$\cdot$s \\ $\phantom{\hbar} = 0.658\cdot 10^{-15}$ eV$\cdot$s 
                \item $k_{b} = 1.381\cdot 10^{-23}$ J/K
                \hfill (constante de Boltzmann) \\
                $\phantom{k_{b}} = 8.617 \cdot 10^{-5}$ eV/K
                \item $m_{e} = 9.109\cdot 10^{-31}$ kg 
                \hfill (massa de repouso do elétron) \\
                $\phantom{m_{e}} = 0.511$ MeV/$c^2$
                \item $m_{p} = 1.672\cdot 10^{-27}$ kg 
                \hfill (massa de repouso do próton) \\
                $\phantom{m_{p}} = 938.3$ MeV/$c^2$
                \item $m_{n} = 1.675\cdot 10^{-27}$ kg 
                \hfill (massa de repouso do nêutron) \\
                $\phantom{m_{n}} = 939.6$ MeV/$c^2$
                \item $\varepsilon_{0} = 8.854\cdot 10^{-12}$ C$^2$/Nm$^2$
                \hfill (permissividade elétrica do vácuo)
            \end{itemize}
    \end{itemize}

\mainmatter

\pagenumbering{arabic}
\setcounter{page}{0}
\thispagestyle{empty}

\part{Fundamentos básicos}

    
    \chapter{\hspace{0.8cm}Introdução à mecânica quântica}
        Durante muitos anos, ficou estabelecido que a física era uma ciência próxima de sua conclusão, em que haviam apenas algumas lacunas a serem preenchidas. Um dos fenômenos de onde originavam questionamentos era o da \Acomment{radiação de corpo negro}, que foi profundamente estudada por grandes cientistas que viriam a se tornar os \Acomment{pais} da mecânica quântica. Nomes como Kirchhoff\footnote{Gustav Robert Kirchhoff (1824--1887).}, Wien\footnote{Wilhelm Carl Werner Otto Fritz Franz Wien (1864--1928).}, Planck\footnote{Max Karl Ernst Ludwig Planck (1858--1947).}, Einstein\footnote{Albert Einstein (1879--1955).} e muitos outros se tornaram muito importantes ao longo de seus estudos sobre esse fenômeno.

Em especial Planck, quando em 1900 apresentou quatro artigos propondo resoluções para o problema da radiação de corpo negro. Em seu primeiro artigo, \textcite{Planck1} expressava uma nova equação, que em suas palavras, representava a \Acomment{lei da distribuição de energia de radiação em todo o espectro}, sendo ela dada por
    \begin{equation*}
        E = \dfrac{C\lambda^{-5}}{e^{c/\lambda T} - 1}
    \end{equation*}
onde $\lambda$ é o comprimento de onda, $c$ a velocidade da luz, $T$ a temperatura e $C$ uma constante. Em seus próximos 3 artigos, \textcite{Planck2, Planck3, Planck4}, ele viria a desenvolver e determinar mais resultados que culminariam em muitos outros fenômenos que se originaram do estudo aprofundado deste problema. Alguns de grande renome são: Efeito fotoelétrico, Efeito Compton\footnote{Arthur Holly Compton (1892--1962).}, descoberta dos raios-X por Röntgen\footnote{Wilhelm Conrad Röntgen (1845--1923).}, e muitos outros de extrema importância que podem ser estudados com mais profundidade em livros como \textcite{Eisberg}, que contextualizam muito bem historicamente o surgimento da mecânica quântica e muitos dos experimentos que solidificaram a teoria.

Neste livro, abordaremos as bases para um estudante de mecânica quântica, começando por exemplo com a definição de estados quânticos, operadores e como utilizamos eles para descrever fenômenos quânticos, o que são espaços de Hilbert\footnote{David Hilbert (1862--1943).}, como o tempo evolui quanticamente e muitos outros conteúdos que exigem uma boa base de Álgebra Linear e todo o ciclo básico de um curso de física. Um curso introdutório de física quântica também é útil para contextualizar alguns conteúdos que veremos mais adiante.

Em um contexto geral, podemos descrever qualquer sistema quântico completamente utilizando-se de cinco princípios não--triviais.
    \begin{myitemize}
        \item Um sistema quântico é caracterizado por um \textbf{vetor de estado} $\ket{\psi}$;
        
        \item Todo observável $A$ é representado por um \textbf{operador hermitiano} $\hat{\mathcal{A}}$;
        
        \item O processo de medida é descrito através da atuação de um operador sobre o estado e resulta sempre em um \textbf{autovalor} do operador;
        
        \item O \textbf{valor esperado} de um observável $\hat{\mathcal{A}}$ é dado por:
            \begin{answer*}
                \langle \hat{\mathcal{A}} \rangle = \bra{\psi}\hat{\mathcal{A}}\ket{\psi}
            \end{answer*}
        
        \item Um sistema quântico evolui no tempo de acordo com a equação de Schrödinger:
            \begin{answer*}
                i\hbar\pdv{}{t}\ket{\psi(t)} = \hat{\mathcal{H}}\ket{\psi(t)}
            \end{answer*}
    \end{myitemize}

    Obviamente que esses princípios básicos não são as únicas ferramentas que utilizamos na descrição da teoria quântica, no entanto eles são essenciais para entender qualquer outro artifício que iremos aplicar, como por exemplo teoria de perturbações, que utiliza-se basicamente de todos esses princípios.
    
    
        \section{Espaço de Hilbert}
            Classicamente, quando consideramos um sistema em um certo instante de tempo $t$ e sabemos como descrever a posição $\vb{r}$ e a velocidade $\vb{v}$, conseguimos caracterizar completamente o estado da partícula. 
    \begin{example}\label{exemple 1.1}
        Suponha que uma partícula clássica esteja em $t_{0}$ na posição $\vb{r}_{0}$ com velocidade $v_{0}=0$ e aceleração constante $\vb{a}$. Sabendo que essa partícula esteja com $\vb{v}=v_{x}\vb{e}_{x}+v_{y}\vb{e}_{y}+v_{z}\vb{e}_{z}$ em $t\neq t_{0}$, qual será o vetor $\vb{r}$ no instante de tempo $t$?
        
        \divider
        
        Utilizando a equação horária do movimento, podemos determinar a posição da partícula facilmente, tal que
            \begin{equation*}
                \vb{r}(t) = \vb{r}_{0} + \vb{v}_{0}t + \dfrac{1}{2}\vb{a}t^2
            \end{equation*}
    \end{example}
    
Em sistemas quânticos a situação é outra. A variável temporal $t$ torna-se um objeto matemático vinculado a uma distribuição de probabilidade, e esta distribuição está diretamente relacionada ao \textbf{vetor de estado} $\ket{\psi}$.

Resumidamente um vetor de estado $\ket{\psi}$ é um vetor em um espaço de Hilbert que armazena todas as informações do estado da partícula associada a ele. Para entender um pouco melhor, vamos entender o que é um \Acomment{espaço de Hilbert}.


        
            \subsection{Base ortonormal em um espaço vetorial euclidiano}
                Comecemos relembrando rapidamente o funcionamento de um espaço vetorial euclidiano. Por definição, um espaço vetorial é da seguinte forma:
\begin{definition}
\label{def: vector space}
    Um espaço vetorial $\mathbb{V}$ sobre um corpo $\mathbb{K}$ (exemplos de corpos são $\mathbb{C}$, $\mathbb{R}$, $\mathbb{Q}$ e muitos outros) é um conjunto de elementos, chamados de \textit{vetores}, dotado de uma operação de \textit{soma vetorial} $+:\mathbb{V}\times\mathbb{V}\to\mathbb{V}$ e um \textit{produto por escalar} $\cdot: \mathbb{K}\times\mathbb{V}\to\mathbb{V}$ tal que:
    \begin{enumerate}
        \item Para cada par de vetores $\vb{v},\vb{u}\in\mathbb{V}$, associa-se um elemento de soma $(\vb{v} + \vb{u}) \in \mathbb{V}$ com as seguintes propriedades:
            \begin{enumerate}[(a)]
                \item Comutatividade: a soma é comutativa para todo $\vb{v},\vb{u}\in\mathbb{V}:\ \vb{v} + \vb{u} = \vb{u} + \vb{v}$;
                \item Associatividade: a soma é associativa para todo $\vb{v},\vb{u},\vb{w}\in\mathbb{V}:\ \vb{v} + (\vb{u} + \vb{w}) = (\vb{v} + \vb{u}) + \vb{w}$;
                \item Vetor nulo: existe um único vetor $\vb{0}$ denominado vetor nulo, tal que para todo $\vb{v}\in \mathbb{V}:\ \vb{v} + \vb{0} = \vb{v}$;
                \item Vetor simétrico: Para cada $\vb{v}\in\mathbb{V}$, existe um único elemento $-\vb{v}\in\mathbb{V}$ tal que $\vb{v} + (-\vb{v}) = \vb{0}$.
            \end{enumerate}
        \item Para cada par $\alpha \in\mathbb{K}$ e $\vb{v}\in\mathbb{V}$, existe um vetor associado $\alpha\cdot\vb{v}\in\mathbb{V}$, chamado de produto por escalar, com as seguintes propriedades:
            \begin{enumerate}[(a)]
                \item Associatividade: o produto por escalares é associativo para todo $\alpha,\beta\in\mathbb{K}$ e $\vb{v}\in\mathbb{V}:\ \alpha\cdot(\beta\cdot\vb{v}) = (\alpha\cdot\beta)\cdot\vb{v}$;
                \item Elemento unitário: existe um elemento unitário $1\in\mathbb{K}$ tal que para todo $\vb{v}\in\mathbb{V}:\ 1\cdot\vb{v} = \vb{v}$;
                \item Distributividade vetorial: para todo $\alpha\in\mathbb{K}$ e $\vb{v},\vb{u}\in\mathbb{K}$, o produto por escalares é distributivo em relação à soma de vetores: $\alpha\cdot(\vb{v} + \vb{u}) = \alpha\cdot\vb{v} + \alpha\cdot\vb{u}$;
                \item Distributividade escalar: para todo $\alpha,\beta\in\mathbb{K}$ e $\vb{v}\in\mathbb{V}$, o produto por escalares é ditributivo em relação à soma de escalares: $(\alpha + \beta)\cdot\vb{v} = \alpha\cdot\vb{v} + \beta\cdot\vb{v}$.
            \end{enumerate}
    \end{enumerate}
\end{definition}

Para termos um espaço vetorial euclidiano é necessário que o espaço vetorial seja finito, real e tenha nele definido um produto interno. Um primeiro conceito fundamental num espaço vetorial $\mathbb{V}$ é sua base, o conjunto de vetores linearmente dependente $\{\vb{v}_1, \vb{v}_2 \dots, \vb{v}_n\} \in \mathbb{V}$ que gera esse espaço:
    \begin{equation*}
        \vb{v} = a_1 \vb{v}_1 + a_2 \vb{v}_2 + \dots + a_n\vb{v}_n,\ \ \vb{v} \in \mathbb{V}, \ \{a_1, a_2,\dots, a_n\} \in \mathbb{R}
    \end{equation*}

Um exemplo usual dessas bases se da quando consideramos um espaço euclidiano $\mathbb{R}^3$ e a representação de um vetor $\vb{v} \in \mathbb{R}^3$ qualquer neste espaço. Comumente representamos $\vb{v}$ conforme os sistemas de coordenadas (a base) que nos são convenientes, como por exemplo as coordenadas:
    \begin{center}
        \tcbsidebyside[
            bicolor,
            colframe=white,
            colback=white,
            before skip = 6mm,
            after skip = 6mm,
            colbacklower=myLColor!10,
            coltitle=myDColor,
            fonttitle=\bfseries\sffamily,
            sidebyside adapt=right]{
                \begin{align*}
                    \text{Cartesianas: }& \vb{v} = v_{x}\vb{e_{x}} + v_{y}\vb{e_{y}} + v_{z}\vb{e_{z}} \\ \\
                    \text{Cilíndricas: }& \vb{v} = v_{r}\vb{e_{r}} + v_{\phi}\vb{e_{\phi}} + v_{z}\vb{e_{z}} \\ \\
                    \text{Esféricas: }& \vb{v} = v_{r}\vb{e_{r}} + v_{\theta}\vb{e_{\theta}} + v_{\phi}\vb{e_{\phi}} \\
                \end{align*}
        }{%
                \begin{tikzpicture}[scale = 1.0]
                    \draw[->] (0,0,0) -- (2,0,0) node[right]{$\vb{e}_{j}$}; % x
                    \draw[->] (0,0,0) -- (0,2,0) node[right]{$\vb{e}_{k}$}; % y
                    \draw[->] (0,0,0) -- (0,0,2) node[left]{$\vb{e}_{i}$};  % z
                    
                    \draw[->,myLColor] (0,0,0) -- (1.2,1.5,1) node[right]{$\vb{v}$};
                    
                    \draw[dotted] (1.2,1.5,1) -- (1.2,0,1) -- (0,0,0);
                    \draw[dotted] (1.2,0,1) -- (1.2,0,0);
                    \draw[dotted] (1.2,0,1) -- (0,0,1);
                    \draw[dotted] (1.2,1.5,1) -- (0,1.5,0);
                \end{tikzpicture}%
                }
    \end{center}

Vale lembrar que independente do sistema de coordenadas escolhido, o vetor $\vb{v}$ é sempre o mesmo, a base forma apenas uma representação deste vetor.

Nos exemplos acima, as bases utilizadas foram $\{\vb{e_{x}}, \vb{e_{y}}, \vb{e_{z}}\}$ (cartesiana), $\{\vb{e_{r}}, \vb{e_{\phi}}, \vb{e_{z}}\}$ (cilíndricas) e $\{\vb{e_{r}}, \vb{e_{\theta}}, \vb{e_{\phi}}\}$ (esféricas), os vetores que as compõem são convenientemente ortonormais, denotando-os genericamente por $\vb{e}_{i}$, isso significa que suas normas são $1$ e que são ortogonais entre si:
    \begin{equation*}
        \norm{\vb{e}_i} = 1 \hspace{1cm} \& \hspace{1cm} 
        \vb{e}_{i} \perp \vb{e}_{k}
    \end{equation*}

Essas bases de representação são essencialmente o que conhecemos como versores, que variam para cada sistema de coordenadas. Bases ortonormais são mais fáceis de se trabalhar que uma base arbitrária devido às propriedades citadas acima.

Dado as vantagens contempladas, suponha agora que tenhamos uma base ortonormal de vetores de um espaço euclidiano genérico $\mathbb{V}$: $\{\vb{e}_{i}\} \in \mathbb{V}$ de dimensão $\mathrm{dim}(\mathbb{V}) = n$, podemos representar um vetor $\vb{v} \in \mathbb{V}$ qualquer assim como nos exemplos em $\mathbb{R}^3$ como uma combinação linear\footnote{Utilizaremos a notação $\{\vb{e}_{i}\}$ para expressar o conjunto de vetores $\vb{e}_{i}$ que compõe a base.}:
    \begin{equation}\label{vetor}
        \vb{v} = \sum_{i=1}^n v_{i}\vb{e}_{i}
    \end{equation}

Podemos também definir neste exemplo o produto interno $\langle \vb{v} , \vb{u}\rangle$ entre dois vetores arbitrários $\vb{v} \in \mathbb{V}$ e $\vb{u} \in \mathbb{V}$ como o produto escalar usual (o mesmo usado em $\mathbb{R}^3$):
\begin{equation*}
    \langle \vb{v} , \vb{u}\rangle = \vb{v} \cdot \vb{u}
\end{equation*}
em que temos por definição que $\langle\vb{v},\vb{u}\rangle = 0$ se $\vb{v}\perp\vb{u}$. 

Podemos constatar algumas propriedades\footnote{Reservaremos as demonstrações das propriedades para o capítulo seguinte, agora já se estendendo para um espaço vetorial complexo.} referentes ao espaço vetorial $\mathbb{V}$. Sendo $\mathbb{V}$ um espaço vetorial com um produto escalar, podemos definir nele uma norma através deste.

\begin{definition}
\label{def: norm}
    Uma norma $\norm{\ \cdot\ }$ é uma função $\mathbb{V}\to\mathbb{R}$ com as seguintes propriedades:
    \begin{itemize}
        \item Para todo $\vb{v}\in\mathbb{V}$, tem-se que $\norm{\vb{v}} \geqslant 0$;
        \item A norma $\norm{\vb{v}} = 0 \Leftrightarrow \vb{v} = \vb{0}$;
        \item Para qualquer $\alpha\in\mathbb{C}$ e qualquer $\vb{v}\in\mathbb{B}$, vale $\norm{\alpha \vb{v}} = \abs{\alpha} \norm{\vb{v}}$;
        \item Para quaisquer $\vb{v},\vb{u}\in\mathbb{V}$, vale $\norm{\vb{v}+\vb{u}} \leqslant \norm{\vb{v}} + \norm{\vb{u}}$.
    \end{itemize}
\end{definition}

\begin{myitemize}
    \item Para o espaço vetorial $\mathbb{V}$ munido de um produto escalar, podemos definir uma norma da forma $\norm{\vb{v}} = \sqrt{\expval{\vb{v},\vb{v}}}$;

    \item Através da norma, podemos normalizar o vetor $\vb{v}$, tal que $\norm{\tilde{\vb{v}}} = 1$, onde $\tilde{\vb{v}} = \dfrac{\vb{v}}{\norm{\vb{v}}}$ é o vetor normalizado;

    \item Através do produto escalar, podemos projetar o vetor $\vb{v}$ sobre um vetor da base $\vb{e}_{i}$, tal que obtemos a componente de $\vb{v}$ nesta compotente, ou seja: $\expval*{\vb{v},\vb{e}_{i}} = v_{i}$.
\end{myitemize}
Perceba que as duas últimas propriedades quando aplicadas aos vetores da base (em que $\norm{\vb{e}_i} = 1$ e $\vb{e}_{i} \perp \vb{e}_{k} \implies 0$) nos leva ao que denominamos como a função condicional delta de Kronecker\footnote{Leopold Kronecker (1823--1891).}
\begin{equation*}
    \langle \vb{e}_i, \vb{e}_j \rangle = \delta_{ij} = \begin{cases}
        1,\ \textrm{se} \ i = j\\ 
        0, \ \textrm{se} \ i \neq j
    \end{cases}
\end{equation*}

Além da representação que estamos usando, podemos tratar dos vetores do espaço $\mathbb{V}$ na forma matricial, isto é, para $\vb{v},\vb{u}\in \mathbb{V}$:
    \begin{equation*}
    \vb{v} = 
        \begin{bmatrix}
            v_{1} \\
            v_{2} \\
            \vdots \\
            v_n
        \end{bmatrix} \hspace{1cm} \& \hspace{1cm}
    \vb{u} = 
        \begin{bmatrix}
            u_{1} \\
            u_{2} \\
            \vdots \\
            u_n
        \end{bmatrix}
\end{equation*}
em que o produto interno passa a ser expresso por meio da operação de transposição $\mathtt{T}$:
\begin{equation*}
    \langle\vb{v},\vb{u}\rangle = \vb{v}^\mathtt{T}\cdot \vb{u} =
    \begin{bmatrix}
        v_1 & v_2 & \cdots & v_n
    \end{bmatrix}
    \cdot
    \begin{bmatrix}
        u_1 \\
        u_2 \\
        \vdots \\
        u_n
    \end{bmatrix}
\end{equation*}


Essas informações preliminares nos serão úteis ao passarmos para espaços vetoriais complexos e em específico o espaço de Hilbert. Antes de chegarmos no espaço de Hilbert propriamente dito, vamos definir a chamada \textit{notação de Dirac} assim como o espaço vetorial com que trabalhamos nesta notação.

































% As \textit{bases de representação} são essencialmente o que conhecemos como versores, que variam para cada sistema de coordenadas. Mais genericamente eles são escritos como $\vb{e}_{i}$, de modo que, \textit{bases de representação} satisfazem
%     \begin{equation*}
%         |\vb{e}_i| = 1 \qquad \& \qquad 
%         \vb{e}_{i} \perp \vb{e}_{k}.
%     \end{equation*}

% De modo geral, representamos um vetor arbitrário por uma combinação linear de suas componentes na base de representação que estamos tratando, ou seja
%     \begin{equation}\label{eq: vbtor}
%         \vb{v} = \sum_{i} v_{i}\vb{e}_{i}
%     \end{equation}

% Com isso em mente, temos que definir o produto escalar entre dois vetores arbitrários $\vb{v}$ e $\vb{U}$, de modo que é necessário satisfazermos:
%     \begin{myitemize}
%         \item Se $\vb{v}\parallel\vb{U}$: $\langle\vb{v},\vb{U}\rangle = \abs{\vb{v}}\abs{\vb{U}}$;
%         \item Se $\vb{v}\perp\vb{U}$: $\langle\vb{v},\vb{U}\rangle = 0$.
%     \end{myitemize}

% Dessa forma, temos que nestes dois casos extremos, o produto entre os versores nada mais é do que um delta de Kronecker\footnote{Leopold Kronecker (1823--1891).} $\delta_{ik}$, definido por
%     \begin{equation}\label{eq: Kronecker's delta}
%         \delta_{ik} = 
%         \begin{cases}
%             0 &,\ \text{se }i\neq k,\\
%             1 &,\ \text{se }i = k.
%         \end{cases}
%     \end{equation}

% Então utilizando a forma dos vetores descrita na eq. \eqref{eq: vbtor}, temos que
%     \begin{align*}
%         \langle\vb{v},\vb{U}\rangle &\eq 
%         \left( \sum_{i}v_{i}\vb{e}_{i} \right)\cdot
%         \left( \sum_{k}u_{k}\vb{e}_{k} \right) = \sum_{i,k}v_{i}u_{k}(\vb{e}_{i}\cdot\vb{e}_{k}) \\
%         &\eq \sum_{i,k}v_{i}u_{k}\delta_{ik} = \sum_{i}v_{i}u_{i}
%     \end{align*}

% Agora se $\vb{v}$ for um dos elementos da base, ou seja $\vb{v} = \vb{e}_{k}$, o produto escalar fica
% \begin{align*}
%     \langle\vb{v},\vb{U}\rangle &\eq \vb{e}_{k}\cdot\sum_{i}u_{i}\vb{e}_{i} =
%     \sum_{i} u_{i}(\vb{e}_{i}\cdot\vb{e}_{k}) \\
%     &\eq \sum_{i}u_{i}\delta_{ik} = u_{k}
% \end{align*}

% Ou seja, se $\vb{v}$ for o elemento de base $k$, conseguimos encontrar a componente $k$ do vetor arbitrário $\vb{U}$.
% \begin{equation*}
%     \langle\vb{k},\vb{U}\rangle = u_{k}
% \end{equation*}

% Além dessa representação, podemos escrever o produto escalar como um produto de matrizes, tal que as matrizes são os vetores coluna:
% \begin{equation*}
%     \vb{v} = 
%         \begin{pmatrix}
%             v_{1} \\
%             v_{2} \\
%             \vdots
%         \end{pmatrix} \hspace{1cm} \& \hspace{1cm}
%     \vb{U} = 
%         \begin{pmatrix}
%             u_{1} \\
%             u_{2} \\
%             \vdots
%         \end{pmatrix}
% \end{equation*}

% Porém, para que o produto seja feito corretamente, precisamos que um dos vetores seja em linha, logo o produto escalar fica:
% \begin{equation*}
%     \langle\vb{v},\vb{U}\rangle = 
%     \begin{pmatrix}
%         v_{1} & v_{2} & \cdots
%     \end{pmatrix}
%     \begin{pmatrix}
%         u_{1} \\
%         u_{2} \\
%         \vdots
%     \end{pmatrix} = \vb{v}^{\mathtt{T}}\vb{U} = \sum_{i}v_{i}u_{i}
% \end{equation*}

% Com essas informações, podemos mudar de um simples espaço euclidiano $\mathbb{R}^3$ para espaços multidimensionais e complexos que irão definir o que são espaços de Hilbert.

% \begin{note}{}
% Para simplificar as notações, vamos definir a notação de Dirac para vetores, de tal forma que eles serão, a partir deste ponto do texto, as formas de escrever vetores em espaços de Hilbert.
% \end{note}

% A notação de Dirac consiste basicamente de duas formas simples de expressar algo que pode ser muito complexo de se analisar, além de permitir algumas manipulações muito mais facilmente do que se utilizássemos a notação usual de vetores.

% Temos então os vetores que podem ser escritos como \textit{ket's} $\ket{\cdot}$ e que podem ser escritos como \textit{bra's} $\bra{\cdot}$. Essas formas de escrever se traduzem na notação usual como sendo
%     \begin{equation*}
%         \ket{v} \equiv \vb{v} \qquad \& \qquad  \bra{v} \equiv (\vb{v}^{\ast})^{\mathtt{T}}
%     \end{equation*}

% Note que se $\vb{v}\in\mathbb{R}^{n}$, então $\bra{v} = \vb{v}^{\mathtt{T}}$. Para mostrar que essas forma são equivalentes a representação usual de vetores, considere $\vb{v},\vb{w}\in\mathbb{R}^{n}$. Trabalhando com as componentes de cada um deles, pode se constatar que
%     \begin{equation*}
%         \vb{v} = \sum_{i}(\vb{v}^{\mathtt{T}}\cdot\vb{e}_{i})\vb{e}_{i} = \sum_{i} \vb{e}_{i}(\vb{e}_{i}^{\mathtt{T}}\cdot\vb{v})
%     \end{equation*}
% de modo que o produto escalar na notação de Dirac é dado por
%     \begin{equation*}
%         \vb{v}^{\mathtt{T}}\cdot\vb{w} \equiv \braket{v}{w}
%     \end{equation*}
% ou seja
%     \begin{equation*}
%         \vb{v} = \sum_{i}\ket{e_{i}}\braket{e_{i}}{v} \overset{\ast}{=} \ket{v} \Rightarrow \sum_{i}\braket{v}{e_{i}}\bra{e_{i}} = \bra{v}
%     \end{equation*}

% \begin{note}{}
%     Veremos mais a frente, da definição do operador de projeção \eqref{eq: projection operador} que $\displaystyle\sum_{i}\ket{e_{i}}\bra{e_{i}}$ consiste de uma matriz unitária na base que se está trabalhando, e por isso resta apenas o $\ket{v}$ na tradução de notação.
% \end{note}

% Além disso, os vetores de base devem satisfazer a condição de ortonormalidade, tal que:
%     \begin{equation*}
%         \braket{e_{i}}{e_{j}} = \delta_{ij}
%     \end{equation*}

% Com isso, podemos verificar que o produto escalar entre dois vetores na notação de Dirac retorna ao resultado da notação usual quando ambos os vetores são completamente reais.
%     \begin{align*}
%         \vb{v}^{\mathtt{T}}\cdot\vb{w} &\eq 
%         \left(\sum_{i}\braket{v}{e_{i}}\bra{e_{i}}\right)
%         \cdot
%         \left(\sum_{j}\ket{e_{j}}\braket{e_{j}}{w}\right) \\
%         &\eq 
%         \sum_{i,j}\braket{v}{e_{i}}\braket{e_{i}}{e_{j}}\braket{e_{j}}{w} \\
%         &\eq \sum_{i,j}\delta_{ij}\braket{v}{e_{i}}\braket{e_{j}}{w} \\
%         &\eq \sum_{i}\braket{v}{e_{i}}\braket{e_{i}}{w} \equiv \sum_{j}\braket{v}{e_{j}}\braket{e_{j}}{w} \\
%         &\eq \braket{v}{w}
%     \end{align*}

% Com isso, a norma de um vetor é dada por:
%     \begin{equation*}
%         \vb{v}\cdot\vb{v} = \braket{v}{v} \Rightarrow \norm{\vb{v}} = \sqrt{\braket{v}{v}}
%     \end{equation*}

% % \subsubsection{Espaço de Hilbert}

% Podemos então definir agora o que é um espaço de Hilbert $\mathscr{H}$ a partir de 4 proposições fundamentais:

% \begin{myitemize}
%     \item A norma de um vetor $\vb{\Psi}=\ket{\psi}\in\mathscr{H}$ deve ser bem determinada (finita) e real, ou seja
%         \begin{equation*}
%             \norm{\vb{\Psi}}\in\mathbb{R} \hspace{1cm} \& \hspace{1cm}
%             \norm{\vb{\Psi}}<\infty
%         \end{equation*}
        
%     \item O produto escalar de dois vetores $\vb{\Psi}\in\mathscr{H}$ e $\vb{\Phi}\in\mathscr{H}$ é bem definido (finito) e possui valor em $\mathbb{C}$;
        
%     \item O conjunto dos vetores nesse espaço vetorial é \textit{completo}, ou seja, toda sequência de Cauchy $\{\varphi_{n}\}_{n\in\mathbb{N}}\in\mathscr{H}$ converge para um elemento do próprio $\mathscr{H}$. Isto é, a sequência $\{\varphi_{n}\}_{n\in\mathbb{N}}$ é tal que $\norm{\varphi_{n} - \varphi_{\ell}}\rightarrow 0$ quando $n$ e $\ell$ tendem à infinito.
    
%     \item Um vetor qualquer de $\mathscr{H}$ pode ser descrito como uma combinação linear de outros vetores do mesmo espaço de Hilbert $\mathscr{H}$.
% \end{myitemize}

% \begin{note}{}
%     De modo geral, um estado quântico é um vetor pertencente a um dado espaço de Hilbert $\mathscr{H}$.
% \end{note}

% Sejam dois vetores $\vb{\Psi}, \vb{\Phi}\in\mathscr{H}$, podemos definir uma base $\vb{e_{i}}\in\mathscr{H}$, tal que
%     \begin{equation*}
%         \vb{\psi} = \sum_{i}\psi_{i}\vb{e}_{i},\ \psi_{i}\in\mathbb{C} 
%         \qquad \& \qquad
%         \vb{\Phi} = \sum_{i}\phi_{i}\vb{e}_{i},\ \phi_{i}\in\mathbb{C}.
%     \end{equation*}

% Como $\norm{\vb{\Psi}}$ e $\norm{\vb{\Phi}}$ devem ser reais, o produto escalar $\vb{\Psi}^{\mathtt{T}}\cdot\vb{\Psi}$ e o $\vb{\Phi}^{\mathtt{T}}\cdot\vb{\Phi}$ devem ser reais e positivos. Portanto introduzimos o complexo conjugado no vetor transposto para que possamos utilizar a notação de Dirac. Como as componentes $\psi_{i},\phi_{i}\in\mathbb{C}$, garantimos que:
%     \begin{equation*}
%         (\vb{\Psi}^{\ast})^{\mathtt{T}}\cdot\vb{\Psi} = \braket{{\psi}}{{\psi}} = \sum_{i} \psi^{\ast}_{i} \psi_{i}\in \mathbb{R}\quad \& \quad 
%         (\vb{\Phi}^{\ast})^{\mathtt{T}}\cdot\vb{\Phi} = \braket{\phi}{\phi} = \sum_{i} \phi^{\ast}_{i} \phi_{i}\in \mathbb{R},
%     \end{equation*}
% já para o produto entre os dois vetores no espaço de Hilbert, temos que o resultado deve obrigatoriamente ser complexo, o que faz com que a notação de Dirac caia como uma luva, pois enquanto um dos vetores deve ser um conjugado complexo transposto, o outro vai ser apenas complexo, garantindo que 
%     \begin{equation*}
%         (\vb{\Psi}^{\ast})^{\mathtt{T}}\cdot\vb{\Phi} = \braket{\psi}{\phi} = \sum_{i}\psi_{i}^{\ast}\phi_{i}\in \mathbb{C} \Rightarrow \braket{\psi}{\phi} \neq \braket{\phi}{\psi},
%     \end{equation*}
% ou seja, a ordem do produto entre dois vetores do mesmo espaço de Hilbert é importante e influencia diretamente no resultado que se pretende obter.

% \begin{note}{}
%     O termo $(\Psi^{\ast})^\mathtt{T} = \bra{\psi}$ corresponde ao conjunto de todos os covetores (uma transformação linear que mapeia vetores a escalares) que formam um subespaço de um espaço vetorial dual.
% \end{note}

% Para simplificar a notação, definiu-se o símbolo ``dagger'' $\dagger$ como sendo o \textbf{conjugado complexo transposto}, tal que o produto escalar se reduz a
%     \begin{equation*}
%         \bra{\psi} = \vb{\Psi}^{\dagger} = 
%         \begin{pmatrix}
%             \psi_{1}^{\ast} & \psi_{2}^{\ast} & \cdots
%         \end{pmatrix}.
%     \end{equation*}

% Então escrevemos o produto escalar de dois vetores de estado em um espaço de Hilbert $\mathscr{H}$ como sendo simplesmente
%     \begin{answer*}
%         \braket{\psi}{\phi} \in \mathbb{C}.
%     \end{answer*}

% Como consequência, temos que a norma de um vetor em um espaço de Hilbert é dado por $\norm{\vb{\Psi}}^2 = \braket{\psi}{\psi} := \psi^2$. Além disso, a forma de escrever o produto escalar em espaços de Hilbert nos permite escrever
%     \begin{equation*}
%         \braket{\psi}{\phi} = \braket{\phi}{\psi}^{\ast}.
%     \end{equation*}

% Analogamente aos vetores no espaço euclidiano $\mathbb{R}^n$, para determinar a componente $k$ de um vetor de estado, basta que seja feito o produto entre um vetor de base e próprio vetor de estado.

% \begin{example}\label{exemple 1.2}
%     Sejam $\ket{\psi} = \ket{e_{k}}$ e $\ket{\phi} = \displaystyle\sum_{i}\phi_{i}\ket{e_{i}}$. É fácil constatar com os argumentos anteriores que podemos escrever
%         \begin{equation}\label{eq: k-th element}
%             \braket{e_{k}}{\phi} =
%             \bra{e_{k}}\sum_{i}\phi_{i}\ket{e_{i}} = 
%             \sum_{i}\phi_{i}\braket{e_{k}}{e_{i}} = \sum_{i}\phi_{i}\delta_{ki} = \phi_{k}
%         \end{equation}
% \end{example}

            \subsection{Notação de Dirac e o espaço de bra-kets}
                Em 1939, foi publicado um breve artigo de \textcite{Dirac-notation}\footnote{Paul Adrien Maurice Dirac (1902--1984).} introduzindo uma nova notação para facilitar a forma em que as equações na mecânica quântica são escritas. Esta notação foi denomida por ele como “notação de bracket”, e é comumente denotada por “notação de Dirac”.

A notação de Dirac é de extrema importância para descrevermos a mecânica quântica com eficiência e de forma intuitiva e diante disto construiremos agora dois espaços vetoriais complexos de dimensão arbitrária que chamaremos de “espaço de kets” $\mathscr{V}$ (como veremos será especificada de acordo com o sistema físico) e “espaço de bras” $\mathscr{V}^{\ast}$ (o motivo desta notação ficará claro adiante). Descreveremos nestes espaços vetores na forma de \textit{kets}, denotados por $\ket{\ \cdot\ }$ e vetores na forma de \textit{bras}, denotados por $\bra{\ \cdot\ }$. 
\begin{definition}
    \label{def: ket space}
    Seja $\mathscr{V}$ um espaço vetorial de dimensão arbitrária sobre $\mathbb{C}$. O conjunto de elementos deste espaço são vetores ket da forma $\ket{v}\in\mathscr{V}$, com as seguintes propriedades:
    \begin{enumerate}
        \item Dois kets podem ser somados para formar um novo ket: para todo $\ket{v},\ket{u}\in\mathscr{V}$, existe um único elemento $\ket{w}\in\mathscr{V}$ tal que $\ket{v} + \ket{u} = \ket{w}$;
        \item A multiplicação por escalar forma um novo ket: para todo $\ket{v}\in\mathscr{V}$ e $\alpha\in\mathbb{C}$, existe um único elemento $\ket{u}\in\mathscr{V}$ tal que $\ket{u} = \alpha\ket{v}$;
        \item Existe um ket nulo: para todo $\ket{v}\in\mathscr{V}$, quando multiplicado por $\alpha = 0$, retorna o ket nulo, que denotaremos simplesmente por $\vb{0}$.
    \end{enumerate}
\end{definition}

Dentro do espaço de kets, existem kets particulares que são de suma importância para toda teoria que são os chamados “autokets” $\{\ket*{a^{(1)}}, \ket*{a^{(2)}},...\} \equiv \{\ket*{a^{(i)}}\}$, que veremos mais a frente de onde surgem, responsáveis por gerar todo o espaço. São essencialmente os conhecidos autovetores da álgebra linear.

Construido o espaço de kets $\mathscr{V}$, somos capazes de construir o espaço de bras $\mathscr{V}^{\ast}$, que é basicamente um espelho de $\mathscr{V}$, porém com algumas características particulares que devem ser enfatizadas para melhor compreensão do que estamos fazendo.

O espaço de bras é formalmente denominado como o “espaço dual do espaço de kets”, isto é, vai existir uma correspondência um-pra-um entre $\mathscr{V}$ e $\mathscr{V}^{\ast}$, chamada \textit{correspondência dual} que pode ser vista sem qualquer rigor da seguinte forma: sendo $\ket{v}$ um elemento de $\mathscr{V}$ e $\alpha\in\mathbb{C}$, o correspondente dual do ket $\alpha\ket{v}\in\mathscr{V}$ é o bra $\alpha^{\ast}\bra{v}\in\mathscr{V}^{\ast}$. Com esta ideia em mente, a construção do espaço de bras se torna a seguinte:
\begin{definition}
    \label{def: bra space}
    Seja $\mathscr{V}$ um espaço vetorial de dimensão arbitrária sobre $\mathbb{C}$. Seja então $\mathscr{V}^{\ast}$ o espaço dual de $\mathscr{V}$, cujos elementos desse espaço são vetores bra da forma $\bra{v}\in\mathscr{V}^{\ast}$, com as seguintes propriedades:
    \begin{enumerate}
        \item Dois bras podem ser somados para formar um novo bra: para todo $\bra{v},\bra{u}\in\mathscr{V}^{\ast}$, existe um único elemento $\bra{w}\in\mathscr{V}^{\ast}$ tal que $\bra{v} + \bra{u} = \bra{w}$;
        \item A multiplicação por escalar forma um novo bra: para todo $\bra{v}\in\mathscr{V}^{\ast}$ e $\alpha\in\mathbb{C}$, existe um único elemento $\bra{u}\in\mathscr{V}^{\ast}$ tal que $\bra{u} = \alpha\bra{v}$;
        \item Existe um bra nulo: para todo $\bra{v}\in\mathscr{V}^{\ast}$, quando multiplicado por $\alpha = 0$, retorna o bra nulo, que denotaremos simplesmente por $\vb{0}$.
    \end{enumerate}
\end{definition}
onde dentro deste espaço temos os “autobras” $\{\bra*{a^{(i)}}\}$, responsáveis por gerar todo o espaço. A título de comparação, ou seja, as correspondências duais entre $\mathscr{V}$ e $\mathscr{V}^{\ast}$, temos
    \begin{equation}\label{eq: dual correspondence brackets}
        \alpha\ket{v} + \beta\ket{u} \leftrightarrow \alpha^{\ast}\bra{v} + \beta^{\ast}\bra{u}
    \end{equation}

Dizemos que o bra $\bra{v}$ é a versão adjunta complexa (ou o hermitiano) do ket $\ket{v}$, sendo a operação de transposição seguida por conjugação complexa das componentes de $\ket{v}$ expressada por meio da notação \textit{dagger} ($\dagger$):
\begin{equation*}
    \bra{v} = (\ket{v}^*)^\mathtt{T} \equiv \ket{v}^\dagger
\end{equation*}

Note que se $\mathscr{V}$ for construido sobre $\mathbb{R}$, então o adjunto complexo se torna apenas a operação de transposição. 
% Ao trabalharmos com o espaço de kets $\mathscr{V}$, estamos lidando com um espaço vetorial linear, no qual precisamos satisfazer a Definição~\ref{def: vector space}, isto é, um conjunto de vetores $\ket{v},\ket{u},...\in\mathscr{V}$ e um conjunto de escalares $\alpha,\beta,...\in\mathbb{C}$ que satisfazem as regras \textit{1} e \textit{2} da definição.

Por fim, definimos um produto interno entre dois vetores, um $\ket{u}\in\mathscr{V}$ e um  $\bra{v}\in\mathscr{V}^{\ast}$. Na notação de Dirac, escrevemos o bra à esquerda e o ket à direita, tal que
    \begin{equation}
    \label{inner product in Dirac notation}
        \braket{v}{u} \coloneqq (\bra{v})\cdot(\ket{u})
    \end{equation}
em que este produto é em geral um número complexo

Vale ressaltar, que da mesma maneira que para um espaço vetorial nos reais, temos que se um ket $\ket{u}\in\mathscr{V}$ e um bra $\bra{v}\in\mathscr{V}^{\ast}$ respeitarem
    \begin{equation*}
        \braket{v}{u} = 0
    \end{equation*}
estes são ditos ortogonais.

Podemos da mesma forma que fizemos na seção anterior (para $\vb{v}$ na base de vetores $\{\vb{e}_i\}$), expandir os vetores $\ket{v}\in\mathscr{V}$ e $\bra{v}\in\mathscr{V}^{\ast}$ em uma base $\{\ket{e_i}\}$ e seu correspondente dual $\{\bra{e_{i}}\}$ (sendo da mesma forma uma base ortonormal, $\braket{e_i}{e_k} = \delta_{ik}$):
    \begin{equation}
    \label{ket expansion in ortonormal basis}
        \ket{v} = \sum_{i} v_{i} \ket{e_i} 
    \end{equation}
e
    \begin{equation}
        \label{bra expansion in ortonormal basis}
        \bra{v} = (\ket{v})^\dagger = \sum_{i} v_{i}^{\ast}\bra{e_{i}}.
    \end{equation}
    
    % \begin{note}{}
    %     De uma maneira mais formal, o termo $\vb{v}^{\dagger} = \bra{v}$ corresponde ao conjunto de todos os covetores (uma transformação linear que mapeia vetores a escalares) que formam um subespaço de um espaço vetorial dual. Para mais detalhes consulte o apêndice ao final do capítulo.
    % \end{note}
    
    Usemos as expansões na base $\{\ket{e_i}\}$ para demonstrar quatro propriedades facilmente deriváveis, que nos serão úteis ao longo do texto e nos permitirão praticar um pouco a nova notação:
    \begin{properties}
        \label{proper: projection under base vector}
        Dado um  espaço de kets $\mathscr{V}$ e seu espaço dual $\mathscr{V}^{\ast}$, sobre os vetores da base $\{\ket{e_{i}}\}$ e os correspondentes duais $\{\bra{e_{i}}\}$, podemos projetar tanto $\ket{v}\in\mathscr{V}$ quanto $\bra{v}\in\mathscr{V}^{\ast}$ nestas bases, tal que 
            \begin{equation*}
                \braket*{e_{i}}{v} = v_{i} \hsp \braket{v}{e_{i}} = v_{i}^{\ast}
            \end{equation*}
    \end{properties}
    \begin{proof}
        Atuando com um elemento da base ortonormal $\bra{e_j}$ pela esquerda em (\ref{ket expansion in ortonormal basis}):
        \begin{equation*}
            \braket{e_j}{v} = \sum_i v_i \braket{e_j}{e_i} = \sum_i v_i \delta_{ij} = v_j
        \end{equation*}
        e da mesma forma atuando com $\ket{e_j}$ pela direita em (\ref{bra expansion in ortonormal basis}):
        \begin{equation*}
            \braket{v}{e_j} = \sum_i v_i^*\braket{e_{i}}{e_j} = \sum_i v_i^* \delta_{ij} = v_j^*
        \end{equation*}
    \end{proof}

    \begin{properties}
        \label{proper: anti-commutation}
        Dado um espaço de kets $\mathscr{V}$ e seu espaço dual $\mathscr{V}^{\ast}$, o produto escalar entre $\ket{v}\in\mathscr{V}$ e $\bra{u}\in\mathscr{V}^{\ast}$ é anti-comutativo, ou seja 
            \begin{equation*}
                \braket{u}{v} = \braket{v}{u}^{\ast}
            \end{equation*}
    \end{properties}
    \begin{proof}
        Aqui podemos utilizar o resultado anterior, considerando primeiramente vetor $\bra{u}$ atuando sobre (\ref{ket expansion in ortonormal basis}):
        \begin{equation*}
            \braket{u}{v} = \sum_{i} v_{i} \braket{u}{e_{i}} = \sum_{i} v_{i} u_{i}^{\ast}
        \end{equation*}
        e da mesma forma atuando $\ket{u}$ sobre (\ref{bra expansion in ortonormal basis}):
        \begin{equation*}
            \braket{v}{u} = \sum_{i} v_{i}^{\ast} \braket{e_{i}}{u} = \sum_{i} v_{i}^{\ast} u_{i} = \left(\sum_{i} v_{i} u_{i}\right)^{\ast}
        \end{equation*}
        sendo portanto
        \begin{equation*}
            \braket{v}{u} = \braket{u}{v}^{\ast}
        \end{equation*}

        Perceba que poderíamos chegar nessa conclusão abrindo a notação de Dirac nos vetores de notação usual
        \begin{equation*}
            \braket{u}{v} = (\vb{u}^{\ast})^\mathtt{T}\vb{v} = (\vb{u}^{\mathtt{T}} \vb{v}^{\ast})^{\ast} = [(\vb{v}^{\ast})^{\mathtt{T}}\vb{u}]^{\ast} = \braket{v}{u}^{\ast}
        \end{equation*}
    \end{proof}

    \begin{properties}
        \label{proper: scalar product geq 0}
        Dado um espaço de kets $\mathscr{V}$ e seu espaço dual $\mathscr{V}^{\ast}$, para cada vetor $\ket{v}\in\mathscr{V}$, vale que
            \begin{equation*}
                \braket*{v} \geqslant 0
            \end{equation*}
        onde $\braket{v} = 0 \Leftrightarrow \ket{v} = \vb{0}$. Propriedade esta conhecida como métrica definida positiva.
    \end{properties}
    \begin{proof}
        Utilizando as formas \eqref{ket expansion in ortonormal basis} e \eqref{bra expansion in ortonormal basis}, temos:
            \begin{align*}
                \braket{v}{v} &\eq \left(\sum_{i} v_{i}^{\ast} \bra{e_{i}} \right) \left(\sum_{i} v_{i} \ket{e_{i}} \right) 
                = \sum_{i, j} v_{i}^{\ast} v_{j} \braket{e_{i}}{e_{j}} \\
                &\eq \sum_{i, j} v_{i}^{\ast} v_{j} \delta_{ij} 
                = \sum_{i} v_{i}^{\ast} v_{i}^{\phantom\ast} 
                = \sum_{i} \abs{v_{i}}^{2} \geq 0
            \end{align*}
    \end{proof}

    \begin{properties}
        \label{proper: inner product is real}
        Dado um espaço de kets $\mathscr{V}$ e seu espaço dual $\mathscr{V}^{\ast}$, o produto interno de todo $\ket{v}\in\mathscr{V}$ com si mesmo é sempre um número real.
    \end{properties}
    \begin{proof}
        A partir de $\braket{u}{v} = \braket{v}{u}^{\ast}$:
        \begin{equation*}
            \braket{v}{v} = \braket{v}{v}^{\ast}
        \end{equation*}
        o que coincide com a definição de um número real.
    \end{proof}

Note que a partir destas propriedades, podemos definir a norma $\norm{\ket{v}}$ do vetor $\ket{v}$ de tal forma que seu resultado seja real e positivo (analogamente à norma euclidiana $\norm{\vb{v}} = \sqrt{\expval{\vb{v},\vb{v}}}$). Através da Propriedade~\ref{proper: inner product is real}, temos a norma definida por
\begin{equation}
\label{eq: definition of norm in Dirac notation}
    \braket{v}{v} \in \mathbb{R} \Rightarrow \norm{\ket{v}} \coloneqq \sqrt{\braket{v}}
\end{equation}

Atente-se para o fato de que não escolhemos $\braket{v}{v}$ de forma a não assumir valores complexos e ser positivo despropositadamente, definimos o produto como \eqref{eq: definition of norm in Dirac notation} de maneira que essa fosse uma consequência. Caso definíssemos o produto como fizemos anteriormente para os reais, isto é $\braket{u}{v}\in\mathbb{C}$, haveria brecha para que os valores não fossem reais. O motivo da escolha do valor da norma ser real e positivo ficará claro mais a frente, ao associarmos a esta grandeza o conceito de probabilidade.

Por fim, da mesma forma que para um espaço vetorial real, se tivermos uma norma 
\begin{equation*}
    \norm{\ket{v}} = \sqrt{\braket{v}{v}} = 1,
\end{equation*}
dizemos que o vetor está normalizado e analogamente à versão real é possível construir um vetor normalizado $\ket{\tilde{v}}\in\mathscr{V}$ a partir de sua versão original $\ket{v}\in\mathscr{V}$ (não normalizada) ao dividirmos este pela norma (caso não estejamos falando de um vetor nulo $\ket{v} = \vb{0}$):
    \begin{equation*}
        \ket{\Tilde{v}} = \frac{1}{\sqrt{\braket{v}{v}}}\ket{v} \implies \norm{\ket{\tilde{v}}} = 1
    \end{equation*}

            \subsection{Definição do espaço de Hilbert}
                Para definirmos o que chamamos de espaço de Hilbert $\mathscr{H}$ partimos do espaço de kets $\mathscr{V}$ (e seu dual) da última seção, isto é, um espaço vetorial linear complexo cuja dimensão é arbitrária. Impomos os seguintes postulados para que $\mathscr{V}$ seja um espaço de Hilbert $\mathscr{H}$.
            
\begin{definition}
    \label{def: Hilbert space}
    Seja $\mathscr{H}$ um espaço vetorial sobre $\mathbb{C}$ com as características da Definição~\ref{def: ket space}, e cujo seu dual $\mathscr{H}^{\ast}$ satisfaz a Definição~\ref{def: bra space}. Sejam $\ket{\psi},\ket{\phi}\in\mathscr{H}$ quaisquer. Para que $\mathscr{H}$ seja um espaço de Hilbert, as seguintes propriedades devem ser satisfeitas:
    \begin{enumerate}
        \item A norma $\norm{\ket{\psi}} \coloneqq \sqrt{\braket{\psi}}$ deve ser real e bem definida (finita);
        \item O produto interno $\braket{\psi}{\phi}$ deve ser bem definido (finito), positivo e possuir valores em $\mathbb{C}$;
        \item Para cada $\ket{\psi}\in\mathscr{H}$, existe uma sequência de Cauchy $\{\ket{\psi_{n}}\}_{n\in\mathbb{N}}\in\mathscr{H}$ tal que para todo $\varepsilon > 0$, existe pelo menos um elemento $\ket{\psi_{k}}$ da sequência que satisfaz $\norm{\ket{\psi} - \ket{\psi_{k}}} < \varepsilon$;
        \item O conjunto dos elementos de $\mathscr{H}$ é um espaço vetorial completo, ou seja, toda sequência de Cauchy $\{\ket{\psi_{n}}\}_{n\in\mathbb{N}}\in\mathscr{H}$ converge para um elemento do próprio $\mathscr{H}$. 
    \end{enumerate}
\end{definition}

Tendo definido o espaço de Hilbert, podemos esboçar um pouco da sua importância para a construção matemática da mecânica quântica. Ao tratarmos da mecânica clássica, todas as propriedades do estado de um sistema de $n$-partículas são completamente determinadas através de um único ponto $P = (x_{1},...,x_{3n};p_{1},...,p_{3n})$ em um espaço de fase $\Omega$ de dimensão $6n$, onde $x_{i}$ e $p_{i}$ são as coordenadas e os momentos das partículas, respectivamente. No entanto, na mecânica quântica essa abordagem não é válida. 

Tratando-se de um estado quântico, ele possui características únicas que diferem-se de $P$. Um fato é que o estado quântico não está diretamente ligado à realidade, isto é, dada uma expressão matemática para descrever o estado (como por exemplo uma função de onda, que veremos mais a frente), não teremos um objeto com significado físico, mas sim teremos uma ferramenta para fazer predições estatísticas sobre o comportamento do estado quântico. Pode não ser tão claro a primeira vista o que de fato queremos transmitir, mas esta é uma das características fundamentais da mecânica quântica! Os estados devem pertencer a um espaço arbitrário, que satisfaz condições únicas e que melhor descreve as propriedades e comportamentos dos sistemas quânticos, e este espaço é o espaço de Hilbert.
% Ainda acho que não consegui explicar o motivo de usarmos espaços de Hilbert...

\begin{note}{}
    Chamaremos, a partir de agora, os vetores $\ket{\psi}\in\mathscr{H}$ de \textit{estados quânticos}.
\end{note}

\begin{note}{}
    A grosso modo, uma sequência de Cauchy é toda sequência $\{a_{n}\}_{n\in\mathbb{N}}$, tal que para dois termos arbitrários da sequência $a_{k}$ e $a_{m}$, vale que $\displaystyle\lim_{k,m\to \infty}\norm{a_{k} - a_{m}}\to 0$ (repare aqui que a notação de norma define a distância entre os elementos, se os elementos forem reais, $\norm{a_{k} - a_{m}} = \abs{a_{k} - a_{m}}$). 
    
    Um exemplo de sequência de Cauchy é a sequência $\qty{\frac{1}{n}}_{n\in\mathbb{N}}$, em que 
    \begin{equation*}
        \abs{\frac{1}{2} - \frac{1}{3}} = \frac{1}{6} > 
        \abs{\frac{1}{3} - \frac{1}{4}} = \frac{1}{12} > ... > 
        \lim_{k,m\to\infty}\abs{\frac{1}{k} - \frac{1}{m}} = 0
    \end{equation*}
\end{note}

Nos atentaremos ao longo desta seção principalmente para o postulado fundamentais (a), dado sua importância física (para mais detalhes sobre as demais condições leia o apêndice ao final do capítulo). O postulado (a) nos diz que $\braket{\psi}{\psi}$ além de pertencer aos reais, deve ser finita e não negativa, isto é 
\begin{equation*}
    0 \leq \braket{\psi}{\psi} < \infty. 
\end{equation*}
Essa restrição, essencialmente sobre a norma de $\ket{\psi}$, nos permite atribuir, do ponto de vista físico,  uma interpretação probabilística da mecânica quântica. Matematicamente temos que as probabilidades $\{P_n\}$ ($n =1, 2,\dots, \ N)$ de eventos $\{\varepsilon_n\}$ ocorrerem devem respeitar duas propriedades \footnote{Note que podemos estender essas propriedades para o caso em que $N\to \infty$.}:
\begin{myitemize}
    \item Todas as probabilidades devem ser positivas e maiores ou iguais a $0$ e menor ou igual a $1$:
    \begin{equation*}
        0\leq P_n \leq 1, \ \forall\ n
    \end{equation*}
    \item A soma de todas as probabilidades deve ser $1$ ($100\%$):
    \begin{equation*}
        \sum_i^{N}P_n = 1
    \end{equation*}
\end{myitemize}
Como $\norm{\ket{\psi}} = \sqrt{\braket{\psi}{\psi}}$ é estritamente positiva e finita, podemos relaciona-la ao conceito de probabilidade dado que seja feita uma normalização apropriada. Sendo $\ket{\psi^\prime} = \ket{\psi}/\sqrt{\braket{\psi}{\psi}}$ a versão normalizada de $\ket{\psi}$ (em que $\ket{\psi}$ não admite o estado nulo) teremos que
\begin{equation*}
    \braket{\psi'}{\psi'} = 1 \implies \norm{\ket{\psi^\prime}}= \sqrt{\braket{\psi^\prime}{\psi^\prime}} = 1
\end{equation*}
Se expandirmos o produto $\braket{\psi'}{\psi'}$ conforme visto na última seção (sendo $\{\ket{e_i}\}\in \mathscr{H}$ uma base ortonormal),
\begin{equation*}
    \ket{\psi'} = \sum_{i}\psi_{i}'\ket{e_{i}}
\end{equation*}
podemos escrever o produto $\braket{\psi'}{\psi'}$ como
\begin{equation*}
    \braket{\psi'}{\psi'} = \sum_i \abs{\psi_i'}^2 = 1
\end{equation*}
como a soma dos termos $\abs{\psi_i'}^2$ deve somar $1$, temos aqui que cada elemento $\abs{\psi_i'}^2$ representa\footnote{Em linguagem estatística o termo não representa a probabilidade em si, mas sim a densidade de probabilidade discreta.} uma probabilidade do estado $\ket{\psi'}$ ser um estado em particular (analogamente à definição de probabilidade comentada anteriormente nesta seção). Conforme o que vimos na seção anterior
\begin{equation*}
    \braket{e_i}{\psi'} = \psi_i' \implies \abs{\braket{e_i}{\psi'}}^2 = \abs{\psi_i'}^2
\end{equation*}
logo os possíveis estados de $\ket{\psi'}$ são os elementos da base $\{\ket{e_1}, \ket{e_2}, \dots\}$ e a probabilidade de $\ket{\psi'}$ atingir um estado $\ket{e_i}$ é dada pela projeção de $\ket{\psi'}$ sobre o elemento $\ket{e_i}$. Na expansão do estado $\ket{\psi'}$ os coeficientes $\psi_i'$ estão relacionados às probabilidades de atingir o estado $i$ e $\ket{e_i}$ é o possível estado.

Perceba que ainda não estamos tratando da evolução de $\ket{\psi}$ no tempo, os estados $\{\ket{e_i}\}$ formam todas as possibilidades do estado $\ket{\psi}$ dentro de um instante fixo. Essa é a ideia de superposição da mecânica quântica.

\begin{note}{}
        Note que caso a norma não fosse finita não poderíamos normalizar o estado dado que 
        \begin{equation*}
            \ket{\psi^\prime} = \frac{\ket{\psi}}{\sqrt{\braket{\psi}{\psi}}}
        \end{equation*}
        não seria definido caso a norma fosse infinita. Se fossem admitidos números complexos ou negativos por parte de $\braket{\psi}{\psi}$, perderíamos a interpretação probabilística da norma quadrada de $\ket{\psi'}$. 
\end{note}

Estamos por enquanto tratando da interpretação probabilística de maneira superficial. Estamos tratando somente de estados discretos, precisamos ainda nos abranger aos estados contínuos. Não associamos ainda um sentido físico à esses possíveis estados, para isso precisamos introduzir os conceitos de operadores, autovalores e autoestados. Esta seção tem o propósito de unicamente associar a norma a uma interpretação probabilística, que nos servirá de alicerce quando tratarmos do assunto aliado a um sentido físico. Finalizaremos está seção com um exemplo puramente ilustrativo de como pensamos em termos de probabilidade a partir de um estado $\ket{\psi}$ aplicado a uma moeda justa (sem trapaças em sua constituição):

\begin{example}[Estados de uma moeda (versão discreto)]{}
    % Conferir nas notas do Tablet

    Realizaremos um exemplo de caráter puramente didático de forma a esboçar a relação do vetor de estado à ideia de probabilidade. Considere o experimento de um arremesso de uma moeda justa. Consideraremos que cada face desse nosso "sistema" representa um estado possível. Imagine um cenário em que os dois possíveis estados estivessem de fato em superposição após o arremesso, não sendo portanto possível distinguir em qual lado a moeda caiu. 
    
    Como sabemos, a probabilidade do evento cair cara é de $50\%$ e o de cair corôa também $50\%$. Por praticidade, utilizaremos a base canônica como a base ortonormal, isto é, os estados
    \begin{equation*}
    \{\ket{e_i}\} = \{\ket{e_1}, \ket{e_2}\}
    \left\{
    \begin{bmatrix}
        1 \\
        0
    \end{bmatrix}, \ 
    \begin{bmatrix}
        0 \\
        1
    \end{bmatrix}
    \right\}
    \end{equation*}
    como deve ser de conhecimento do leitor, essa base é ortonormal. Sendo o vetor de estado dado por 
    \begin{equation*}
        \ket{\psi} = 
        \begin{bmatrix}
        \psi_1 \\
        \psi_2
        \end{bmatrix}
    \end{equation*}
    temos que a projeção do vetor da base $\ket{e_1}$ sobre o vetor de estado resulta em
    \begin{equation*}
        \braket{e_1}{\psi} = 
        \begin{bmatrix}
            1 & 0
        \end{bmatrix}
        \cdot
        \begin{bmatrix}
        \psi_1 \\
        \psi_2
        \end{bmatrix}
        = \psi_1
    \end{equation*}
    como dito anteriormente ao longo desta seção, associamos as probabilidades de cada estado à grandeza $\abs{\psi_i}^2$, sendo assim (note que a possibilidade negativa é descartada):
    \begin{equation*}
        \abs{\psi_1}^2 = \abs{\psi_2}^2 = \frac{1}{2} \implies \psi_1 = \psi_2 = \frac{1}{\sqrt{2}}
    \end{equation*}
    Temos portanto que o vetor de estado dessa nossa versão quântica da moeda seria
    \begin{equation*}
        \ket{\psi} = \sum_i \psi_i \ket{e_i} = \frac{1}{\sqrt{2}}\ket{e_1} + \frac{1}{\sqrt{2}}\ket{e_2}
    \end{equation*}
    em que $\ket{e_1}$ seleciona um dos estados (digamos o cara por exemplo) cuja probabilidade é fornecida pelo coeficiente $\psi_1$ pois $\abs{\psi_1}^2 = 1/2$ e o estado $\ket{e_2}$ seleciona o outro (digamos que o corôa) cuja probabilidade é fornecida pelo coeficiente $\psi_2$ pois $\abs{\psi_2}^2 = 1/2$. Note que como já associamos diretamente as probabilidades à $\abs{\psi_1}^2$ e $\abs{\psi_2}^2$, não havia necessidade de normalização, já partimos desde o princípio de que $\abs{\psi_1}^2 + \abs{\psi_2}^2 = 1$, o que não é a maneira como resolvemos os problemas de mecânica quântica, usualmente a probabilidade de cada estado será o que descobriremos ao final e necessitaremos realizar uma normalização.
    
    Vale ressaltar que uma moeda pertencente ao nosso ambiente macroscópico não pode representar um sistema quântico, dado que de forma mecanicista poderíamos prever o movimento a partir das condições iniciais de arremesso e obter com certeza qual dos lados ela teria caído. Ao desenvolvermos conceitos fundamentais como operadores nos próximos capítulos, seremos capazes de resolver um problema de caráter mais realista em mecânica quântica.
\end{example}

            \subsection{Bases contínuas}
                \input{Partes/Parte 2/Capítulos/Seções/Subseções/Bases contínuas}

                \subsubsection{Produto escalar em estados contínuos}
                    \input{Partes/Parte 2/Capítulos/Seções/Subseções/Subsubseção/Produto escalar em estados contínuos}

                \subsubsection{Vetor de estado e projeção para estados contínuos}
                    \input{Partes/Parte 2/Capítulos/Seções/Subseções/Subsubseção/Vetor de estado e projeção para estados contínuos}

                \subsubsection{Introdução ao espaço de posições e a função de onda}
                    \input{Partes/Parte 2/Capítulos/Seções/Subseções/Subsubseção/Introdução ao espaço de posições e a função de onda.tex}

                \subsubsection{Condição de quadrado integrável}
                    \input{Partes/Parte 2/Capítulos/Seções/Subseções/Subsubseção/Condição de quadrado integrável.tex}

        \section{Mudança de base}
            Muitas vezes queremos mudar as bases de um vetor, como por exemplo passar de coordenadas cartesianas para cilíndricas ou esféricas, e para isso utilizávamos expressões que relacionavam as coordenadas, porém para bases gerais, as coisas não são tão simples assim.

Suponha que conhecemos um vetor de estado $\ket{\psi}$ em uma base ortonormal $\ket{e_{i}}$, ou seja
    \begin{equation*}
        \ket{\psi} = \sum_{i}\psi_{i}\ket{e_{i}} \qquad \& \qquad 
        \braket{e_{i}}{e_{j}} = \delta_{ij}
    \end{equation*}
e queremos expressar esse mesmo vetor em uma outra base ortonormal $\ket{b_{j}}$ tal que:
    \begin{equation*}
        \ket{\psi} = \sum_{j}\beta_{j}\ket{b_{j}}
    \end{equation*}
    
\textbf{A questão é:} Como calcular os elementos $\beta_{i}$? E para responder isso, precisamos introduzir o conceito de \textit{operador de projeção}.

\begin{definition}{}\label{def: projection operator}
    (\textbf{Operador de projeção}) Dada uma base $\ket{e_{i}}$ qualquer, definimos o operador de projeção $\hat{\mathcal{P}}$ como sendo
        \begin{answer}\label{eq: projection operador}
            \hat{\mathcal{P}} = \sum_{i}\ket{e_{i}}\bra{e_{i}}
        \end{answer}
\end{definition}

Note que aqui temos um produto \Acomment{ket--bra} e não um \Acomment{bra--ket}, isso muda completamente a quantidade resultante. No caso do produto \Acomment{bra--ket} como já vimos, isso nos retorna uma quantidade escalar, já o produto \Acomment{ket--bra} irá nos retornar uma matriz quadrada com a dimensão dos vetores de base. Esse produto é denominado \textit{produto direto entre vetores}.

\begin{example}\label{exemple 1.3}
    Considere o sistema de coordenadas cartesianas no espaço euclidiano $\mathbb{R}^3$. Neste caso os vetores de base são:
        \begin{equation*}
            \ket{e_{1}} = 
            \begin{pmatrix}
                1 \\
                0 \\
                0
            \end{pmatrix} \qquad \& \qquad 
            \ket{e_{2}} = 
            \begin{pmatrix}
                0 \\
                1 \\
                0
            \end{pmatrix} \qquad \& \qquad 
            \ket{e_{3}} = 
            \begin{pmatrix}
                0 \\
                0 \\
                1
            \end{pmatrix}
        \end{equation*}
        
    Com isso, podemos calcular $\hat{\mathcal{P}}$ diretamente:
        \begin{align*}
            \hat{\mathcal{P}} &\eq \sum_{i}\ket{e_{i}}\bra{e_{i}} \\
            &\eq \begin{pmatrix}
                1 \\
                0 \\
                0
            \end{pmatrix} 
            \begin{pmatrix}
                1 & 0 & 0
            \end{pmatrix} + 
            \begin{pmatrix}
                0 \\
                1 \\
                0
            \end{pmatrix} 
            \begin{pmatrix}
                0 & 1 & 0
            \end{pmatrix} + 
            \begin{pmatrix}
                0 \\
                0 \\
                1
            \end{pmatrix} 
            \begin{pmatrix}
                0 & 0 & 1
            \end{pmatrix} \\
            &\eq 
            \begin{pmatrix}
                1 & 0 & 0 \\
                0 & 0 & 0 \\
                0 & 0 & 0
            \end{pmatrix} +
            \begin{pmatrix}
                0 & 0 & 0 \\
                0 & 1 & 0 \\
                0 & 0 & 0
            \end{pmatrix} +
            \begin{pmatrix}
                0 & 0 & 0 \\
                0 & 0 & 0 \\
                0 & 0 & 1
            \end{pmatrix} \\
            &\eq \boldone
        \end{align*}
\end{example}

Ou seja, o operador de projeção é sempre uma matriz identidade, de modo que:
    \begin{equation*}
        \boldone\ket{\psi} = \ket{\psi} \qquad \& \qquad
        \bra{\psi}\boldone = \bra{\psi}
    \end{equation*}

Voltemos então ao problema de transformar as componentes da base $\ket{e_{i}}$ para base $\ket{b_{i}}$. O fato da matriz de projeção ser a identidade, nos permite escrever que:
    \begin{equation*}
        \hat{\mathcal{P}} = \sum_{j} \ket{b_{j}}\bra{b_{j}} = \boldone
    \end{equation*}

Como conhecemos $\ket{\psi}$ na base $\ket{e_{i}}$, o mais concreto seria começar por ele.
    \begin{align*}
        \ket{\psi} &\eq \sum_{i}\psi_{i}\ket{e_{i}} \\
        &\eq \sum_{i}\psi_{i}{\color{myLColor}\boldone}\ket{e_{i}} = 
        \sum_{i}\psi_{i}{\color{myLColor!50}{\sum_{j}\ket{b_{j}}\bra{b_{j}}}}\ket{e_{i}}\\
        &\eq \sum_{i,j}\psi_{i}\ket{b_{j}}\braket{b_{j}}{e_{i}} = 
        \sum_{j}{\color{myLColor!50}\sum_{i}\psi_{i}\braket{b_{j}}{e_{i}}}\ket{b_{j}}
    \end{align*}

\begin{note}{}
    Adicionar a matriz de projeção à expressão de $\ket{\psi}$ implica em dizermos que estamos projetando o elemento $j$ do vetor de base $\ket{e_{i}}$ no vetor de base $\ket{b_{j}}$. 
\end{note}

Dessa forma, como queremos escrever $\ket{\psi} = \displaystyle\sum_{j}\beta_{j}\ket{b_{j}}$, podemos comparar os resultados 
    \begin{equation*}
        \ket{\psi} = \sum_{j}\beta_{j}\ket{b_{j}} \qquad \& \qquad 
        \ket{\psi} = \sum_{j}\sum_{i}\psi_{i}\braket{b_{j}}{e_{i}}\ket{b_{j}}
    \end{equation*}
e concluir que:
    \begin{answer*}
        \beta_{j} = \sum_{i}\psi_{i}\braket{b_{j}}{e_{i}}
    \end{answer*}

Expandindo um pouco essa ideia, existem vetores de estado que podem ser escritos em bases de dimensão infinita, isto é, a base torna-se um contínuo de vetores de base, de modo que:
    \begin{equation*}
        \ket{\psi} = \sum_{i}\psi_{i}\ket{e_{i}} \rightarrow 
        \ket{\psi} = \int\psi(\xi)\ket{\xi}\dd\xi
    \end{equation*}

\begin{example}\label{exemple 1.4}
Um exemplo importante que envolvem bases contínuas, é a \Acomment{base de autoestados de posição}, tal que dado um vetor de base $\ket{x}\in\{\ket{x}\}$, que designará um vetor posição e outro vetor de base $\ket{x'}\in\{\ket{x}\}$, a base de autoestados satisfaz que:
    \begin{equation*}
        \braket{x'}{x} = \delta(x' - x) \rightarrow {\color{myLColor}\text{Delta de Dirac}}
    \end{equation*}

\begin{note}{Delta de Dirac}
    Uma das propriedades principais da Delta de Dirac se dá a partir de integrais:
        \begin{equation*}
            \int_{-\infty}^{\infty} f(x')\delta(x-x')\dd x' = f(x)
        \end{equation*}
    
    Outra propriedade importante é que para $x=0$, sendo a função delta par, $\delta(x-x') = \delta(x')$, temos que:
        \begin{equation*}
            \delta(x') = 0,\ \forall x'\neq 0
        \end{equation*}

    Para mais detalhes acerca deste recurso matemático, recomendamos a leitura disponível em \textcite{Arfken,Barata38}
\end{note}

Dessa forma, qual seria o valor de um vetor de estado $\ket{\psi}$ qualquer na componente $\ket{x}$ da base $\{\ket{x}\}$? Constata-se facilmente que
    \begin{align*}
        \braket{x}{\psi} &\eq \bra{x}\int\psi(x')\ket{x'}\dd{x'} \\
        &\eq \int \psi(x')\braket{x}{x'}\dd{x'} \\
        &\eq \int \psi(x')\delta(x-x')\dd{x'} = \psi(x)    
    \end{align*}
\end{example}

Uma consequência super importante que desenvolvemos neste simples exemplo é o fato de que para determinar a função de onda de uma partícula, ou de um sistema de partículas, em uma base qualquer de autoestados $\{\ket{x}\}$, basta calcular
    \begin{answer}\label{eq: wavefunction eigenstate base}
        \psi(x) = \braket{x}{\psi}
    \end{answer}

Temos nesse exemplo a ideia de operador intrínseca no problema, vamos então definir o que são e por que eles se relacionam com este exemplo.

\begin{note}{}
    Antes de prosseguir, vale salientar que no Apêndice \ref{apendice A} há uma discussão mais aprofundada sobre espaços de Hilbert e algumas de suas propriedades essenciais para mecânica quântica.
\end{note}

        \section{Operadores em Mecânica Quântica}
                Finalmente definiremos o que são operadores e o porque dessa ferramenta ser tão útil no desenvolvimento de toda teoria quântica. Grande parte de sua importância se dá pelo fato de que muita informação pode ser armazenada em um único operador e mesmo assim nenhuma propriedade essencial do sistema estudado é perdida quando desenvolvemos as contas com eles.
    
    \begin{definition}\label{def: operators}
        
        (\textbf{Operadores}) Um operador é um objeto matemático que age sobre o vetor de estado do sistema e produz outro vetor de estado. Para ser preciso, se denotarmos um operador por $\hat{\mathcal{A}}$ e $\ket{\psi}$ um elemento do espaço de Hilbert do sistema, então
            \begin{answer*}
                \hat{\mathcal{A}}\ket{\psi} = \ket{\phi}
            \end{answer*}
        onde o vetor de estado $\ket{\phi}$ também pertence ao mesmo espaço de Hilbert.
    \end{definition}
    
    É importante notar que $\bra{\phi'}$ não necessariamente é o vetor dual de $\ket{\phi}$! É fácil ver que isso é verdade:
    
    \begin{example}\label{example 1.5}
        Consideremos o vetor de estado $\ket{\psi}$ e o operador $\hat{\mathcal{A}}\in\mathbb{R}^2$ como sendo
            \begin{equation*}
                \ket{\psi} = \begin{pmatrix}
                    x \\ y
                \end{pmatrix} \qquad \& \qquad 
                \hat{\mathcal{A}} = \begin{pmatrix}
                    a & b \\ c & d
                \end{pmatrix}.
            \end{equation*}
        
        Temos então
            \begin{equation*}
                \hat{\mathcal{A}}\ket{\psi} = 
                    \begin{pmatrix}
                        a & b \\ c & d
                    \end{pmatrix}
                    \begin{pmatrix}
                        x \\ y
                    \end{pmatrix} = 
                    \begin{pmatrix}
                        ax + by \\ cx + dy
                    \end{pmatrix} \coloneqq \ket{\phi},
            \end{equation*}
        de modo que transformamos o vetor coluna $\ket{\psi}$ em outro vetor coluna $\ket{\psi}$. De forma análoga
            \begin{equation*}
                \bra{\psi}\hat{\mathcal{A}} = 
                \begin{pmatrix}
                    x^{\ast} & y^{\ast}
                \end{pmatrix}
                \begin{pmatrix}
                    a & b \\ c & d
                \end{pmatrix} = 
                \begin{pmatrix}
                    x^{\ast}a + y^{\ast}c & x^{\ast}b + y^{\ast}d    
                \end{pmatrix} \coloneqq \bra{\phi'}
            \end{equation*}
        em que também transformamos um \Acomment{bra} em outro \Acomment{bra}. Note então que o dual de $\ket{\phi}$ é
            \begin{equation*}
                \bra{\phi} = 
                \begin{pmatrix}
                    a^{\ast}x^{\ast} + b^{\ast}y^{\ast} & c^{\ast}x^{\ast} + d^{\ast}y^{\ast}
                \end{pmatrix},
            \end{equation*}
        portanto, podemos concluir que
            \begin{equation*}
                \bra{\phi} \neq \bra{\phi'}.
            \end{equation*}
    \end{example}
    
    Dessa forma, vemos que $\bra{\phi}\hat{\mathcal{A}}$ não é o vetor dual de $\hat{\mathcal{A}}\ket{\psi}$. Podemos mostrar então que o vetor dual de $\hat{\mathcal{A}}\ket{\psi}$ é dado por $\bra{\psi}\hat{\mathcal{A}}^{\dagger}$.
    \begin{note}{}
        Muitas vezes o termo $\hat{\mathcal{A}}^{\dagger}$ é denotado por \Acomment{operador adjunto de $\hat{\mathcal{A}}$}.
    \end{note}
    
    Como os operadores são matrizes, temos que as propriedades básicas dos operadores são as mesmas das matrizes ou seja
    \begin{myitemize}
        \item A soma de operadores é comutativa e associativa. $\textrm{Dados }\hat{\mathcal{A}}, \hat{\mathcal{B}}, \hat{\mathcal{C}}$:
            \begin{equation*}
                \hat{\mathcal{A}} + \hat{\mathcal{B}} + \hat{\mathcal{C}} = (\hat{\mathcal{A}} + \hat{\mathcal{B}}) + \hat{\mathcal{C}} = \hat{\mathcal{A}} + (\hat{\mathcal{B}} + \hat{\mathcal{C}}) = \hat{\mathcal{C}} + \hat{\mathcal{A}} + \hat{\mathcal{B}}
            \end{equation*}
        \item O produto é associativo, mas em geral não é comutativo. Dados $\hat{\mathcal{A}},\hat{\mathcal{B}},\hat{\mathcal{C}}$:
            \begin{equation*}
                \hat{\mathcal{A}}\hat{\mathcal{B}}\hat{\mathcal{C}} = (\hat{\mathcal{A}} \hat{\mathcal{B}}) \hat{\mathcal{C}} = \hat{\mathcal{A}} (\hat{\mathcal{B}} \hat{\mathcal{C}}) \neq \hat{\mathcal{C}} \hat{\mathcal{A}} \hat{\mathcal{B}}
            \end{equation*}
    \end{myitemize}
    
    O fato dos operadores não serem comutativos, nos induz a introduzir uma operação que nos mostra se os operadores são ou não comutativos, ou seja, se a operação der zero, diremos que eles comutam.
    
    \begin{definition}\label{def: commutator}
        (\textbf{Comutador}) Dados dois operadores $\hat{\mathcal{A}}$ e $\hat{\mathcal{B}}$, definimos o \textit{comutador} como sendo
            \begin{answer}\label{eq: commutator}
                [\hat{\mathcal{A}},\hat{\mathcal{B}}] = \hat{\mathcal{A}}\hat{\mathcal{B}} - \hat{\mathcal{B}}\hat{\mathcal{A}}
            \end{answer}
    \end{definition}

            \subsection{Medidas em Mecânica Quântica}
                    Em mecânica clássica, temos que os eventos são determinísticos, de modo que conhecendo as variáveis do problema, podemos prever o resultado final do evento, ou seja, existe sempre uma equação que rege o problema. No mundo quântico a situação muda. Não exitem equações que determinem exatamente o resultado final de um evento. Em mecânica quântica, as medidas são tomadas de forma completamente \textit{probabilística}.
    
    Em 1954, Born\footnote{Max Born (1882--1970).} publicou um artigo, \textcite{Born}, onde propôs que a probabilidade de ocorrência de um evento quântico está intrinsecamente atrelada ao módulo quadrado da função de onda $\psi(x)$, isto é, a probabilidade de um dado evento $X$ ocorrer num intervalo $[x,x+\dd{x}]$, onde $\dd{x}$ é o tamanho do detector, dado um certo conjunto de informações $\mathcal{I}$ é dada por:
        \begin{equation*}
            \prob{X\in[x,x+\dd{x}]}{\mathcal{I}}\dd{x} = \abs{\psi(x)}^2\dd{x}
        \end{equation*}
    
    Utilizando essa ideia, dizemos que para variáveis discretas, o valor esperado de uma dada quantidade $x$ é:
        \begin{equation*}
            \expval*{x} = \sum_{i}x_{i}\prob{x_{i}}{\mathcal{I}}
        \end{equation*}
    
    Já para variáveis contínuas, analisamos a probabilidade em intervalos infinitesimais, de modo que:
        \begin{equation*}
            \expval*{x} = \int x\prob{x}{\mathcal{I}}\dd{x} = \int x\abs{\psi(x)}^2\dd{x}
        \end{equation*}
    
    Como podemos abrir $\abs{\psi(x)}^2 = \psi^{\ast}(x)\psi(x)$ e comutá-los da maneira que quisermos, chegamos em:
        \begin{equation*}
            \expval*{x} = \int x\psi^{\ast}(x)\psi(x)\dd{x} = \int\psi^{\ast}(x)x\psi(x)\dd{x}
        \end{equation*}
    
    Que é um resultado já conhecido em cursos básicos de física quântica, mais especificamente, recomenda-se a leitura do Cap. 5 de \textcite{Eisberg}. Além da posição, existem outros observáveis importantes em mecânica quântica como momento ou energia, tal que é útil escrevermos o valor esperado de um observável num estado qualquer como sendo:
        \begin{answer}\label{eq: expected value of an observable}
            \expval*{\mathcal{\hat{A}}} = \bra{\psi}\hat{\mathcal{A}}\ket{\psi}
        \end{answer}
    
    Uma vez que queremos o valor esperado $\expval*{\hat{\mathcal{A}}}$, precisamos determinar quais são os elementos da matriz $\hat{\mathcal{A}}$, o que pode ser feito a partir da introdução operadores de projeção:
        \begin{align*}
            \boldone\hat{\mathcal{A}}\boldone &\eq 
            \sum_{i}\ket{e_{i}}\bra{e_{i}}\hat{\mathcal{A}}\sum_{j}\ket{e_{j}}\bra{e_{j}} \\
            &\eq \sum_{i,j}\ket{e_{i}}\bra{e_{i}}\hat{\mathcal{A}}\ket{e_{j}}\bra{e_{j}}
        \end{align*}
    
    Definindo o elemento de matriz $A_{ij}\coloneqq \bra{e_{i}}\hat{\mathcal{A}}\ket{e_{j}}$, obtemos que: 
        \begin{answer}\label{eq: matrix form of an operator}
            \hat{\mathcal{A}} = \boldone\hat{\mathcal{A}}\boldone = \sum_{i,j}\bra{e_{i}}\hat{\mathcal{A}}\ket{e_{j}}\ket{e_{i}}\bra{e_{j}} = \sum_{i,j}A_{ij}\ket{e_{i}}\bra{e_{j}}
        \end{answer}
    
    \begin{example}\label{example 1.6}
        Para visualizar o resultado acima, considere o espaço euclidiano $\mathbb{R}^2$ de tal forma que:
            \begin{align*}
                \hat{\mathcal{A}} &\eq 
                A_{11}\ket{e_{1}}\bra{e_{1}} + 
                A_{12}\ket{e_{1}}\bra{e_{2}} +
                A_{21}\ket{e_{2}}\bra{e_{1}} + 
                A_{22}\ket{e_{2}}\bra{e_{2}} \\
                &\eq 
                A_{11}\begin{bmatrix}
                    1 & 0 \\ 0 & 0
                \end{bmatrix} + 
                A_{12}\begin{bmatrix}
                    0 & 1 \\ 0 & 0
                \end{bmatrix} +
                A_{21}\begin{bmatrix}
                    0 & 0 \\ 1 & 0 
                \end{bmatrix} +
                A_{22}\begin{bmatrix}
                    0 & 0 \\ 0 & 1
                \end{bmatrix} \\
                &\eq \begin{bmatrix}
                    A_{11} & A_{12} \\ A_{21} & A_{22}
                \end{bmatrix}
            \end{align*}
    \end{example}
    
    De cara não é fácil ver que escrever o valor esperado na forma \eqref{eq: expected value of an observable} é o mesmo que escrever a conhecida forma integral. No entanto, para ver isso melhor, podemos analisar por exemplo o operador posição $\hat{x}$, de modo que consideramos o operador de projeção de uma base contínua como sendo:
        \begin{equation*}
            \boldone = \int \ket{x}\bra{x}\dd{x}
        \end{equation*}
    
    Com isso, temos que:
        \begin{align*}
            \expval*{\hat{x}} &\eq \bra{\psi}\hat{x}\ket{\psi} \\
            &\eq \bra{\psi}\boldone \hat{x}\boldone\ket{\psi} \\
            &\eq \bra{\psi}
            {{\color{myLColor!50}\int\ket{x}\bra{x}\dd{x}}}\ 
            \hat{x} 
            {{\color{myLColor!50}\int\ket{x'}\bra{x'}\dd{x'}}}
            \ket{\psi} \\
            &\eq \int\int \braket{\psi}{x}\bra{x}\hat{x}\ket{x'}\braket{x'}{\psi}\dd{x'}\dd{x}
        \end{align*}
    
    Note que $\braket{\psi}{x}$ corresponde à função de onda conjugada, ou seja $\psi^{\ast}(x)$. Já $\braket{x'}{\psi}$ é a função de onda em $x'$, ou seja $\psi(x')$, portanto:
        \begin{align*}
            \expval*{\hat{x}} &\eq \int\int \psi^{\ast}(x)\bra{x}\hat{x}\ket{x'}\psi(x')\dd{x}'\dd{x} \\
            &\eq \int\int \psi^{\ast}(x)\hat{x}\braket{x}{x'}\psi(x')\dd{x'} \dd{x} \\
            &\eq \int\int \psi^{\ast}(x)\hat{x}\delta(x-x')\psi(x')\dd{x'}\dd{x}
        \end{align*}
    
    Separando as integrais em $\dd{x}'$ e $\dd{x}$:
        \begin{equation*}
            \expval*{\hat{x}} = \int\psi^{\ast}(x)\hat{x}\underbrace{\left[\int\psi(x')\delta(x-x')\dd{x}'\right]}_{\psi(x)}\dd{x}
        \end{equation*}
    
    Concluímos então que:
        \begin{answer}\label{eq: expectated value of x}
                \expval*{\hat{x}} = \int\psi^{\ast}(x)\hat{x}\psi(x)\dd{x} = \bra{\psi}\hat{x}\ket{\psi}
        \end{answer}
    
    % \subsection{Operadores Hermitianos}
    
    % Um fato importante sobre os observáveis é que o valor esperado de um operador que se relaciona diretamente a ele deve sempre ser um número real, isto é, para um dado evento $A$ que está associado a um operador $\hat{\mathcal{A}}$, temos que:
    %     \begin{equation*}
    %         \expval*{\hat{\mathcal{A}}}^{\ast} = \expval*{\hat{\mathcal{A}}} \in\mathbb{R}
    %     \end{equation*}
    
    % Como consequência, podemos escrever:
    %     \begin{align*}
    %         \expval*{\hat{\mathcal{A}}}^{\ast} = (\bra{\psi}\hat{\mathcal{A}}\ket{\psi})^{\ast} &= \bra{\psi}^{\ast}\hat{\mathcal{A}}^{\ast}\ket{\psi}^{\ast} \\
    %         &= (\hat{\mathcal{A}}^{\ast}\ket{\psi}^{\ast})^{\mathtt{T}}(\bra{\psi}^{\ast})^{\mathtt{T}} \\
    %         &= \bra{\psi}\hat{\mathcal{A}}^{\dagger}\ket{\psi}
    %     \end{align*}
    %     \begin{align*}
    %         \expval*{\hat{\mathcal{A}}} = \bra{\psi}\hat{\mathcal{A}}\ket{\psi}
    %     \end{align*}
    
    % Mas então, como $\expval*{\hat{\mathcal{A}}}^{\ast} = \expval*{\hat{\mathcal{A}}}$, podemos concluir que quando um operador está associado a um observável, vale que:
    %     \begin{equation}\label{operadores hermitianos}
    %         \resp{
    %             \hat{\mathcal{A}}^{\dagger} = \hat{\mathcal{A}}
    %         }   
    %     \end{equation}
    
    % Esses operadores são denominados \textit{operadores hermitianos} e são de suma importância para a realização de uma medida em mecânica quântica, de modo que operadores hermitianos são condição necessária para caracterizar um observável.
    
    % Vamos supor então que iremos realizar um experimento qualquer no qual pretendemos medir um observável relacionado a um operador $\hat{\mathcal{A}}$, de tal forma que conseguimos determinar o valor esperado desse operador ($\expval*{\hat{\mathcal{A}}} = a$). Nesse experimento, obtemos que todas as medidas são equivalentes ao valor esperado, isto implica diretamente que a variância do valor esperado de $\hat{\mathcal{A}}$ é zero, ou seja:
    %     \begin{equation*}
    %         \sigma^2 = \expval*{(\hat{\mathcal{A}} - \expval*{\hat{\mathcal{A}}})^2} = 0 \Rightarrow \bra{\psi}(\hat{\mathcal{A}} - \expval*{\hat{\mathcal{A}}})^2\ket{\psi} = 0
    %     \end{equation*}
    
    % Expandindo essa relação, temos que:
    %     \begin{equation*}
    %         \bra{\psi}(\hat{\mathcal{A}} - \expval*{\hat{\mathcal{A}}})^{\dagger}(\hat{\mathcal{A}} - \expval*{\hat{\mathcal{A}}})\ket{\psi} = 0
    %     \end{equation*}
    
    % De tal forma que podemos definir:
    %     \begin{equation*}
    %         \bra{\phi} \coloneqq \bra{\psi}(\hat{\mathcal{A}} - \expval*{\hat{\mathcal{A}}})^{\dagger} \hspace{1cm} \& \hspace{1cm} 
    %         \ket{\phi} \coloneqq (\hat{\mathcal{A}} - \expval*{\hat{\mathcal{A}}})\ket{\psi}
    %     \end{equation*}
    
    % Portanto, para que a equação seja verdadeira:
    %     \begin{equation*}
    %         \bra{\phi} = 0 \hspace{1cm} \textrm{ou} \hspace{1cm} \ket{\phi} = 0
    %     \end{equation*}
    
    % O que implica diretamente em:
    %     \begin{equation*}
    %         (\hat{\mathcal{A}} - \expval*{\hat{\mathcal{A}}})\ket{\psi} = 0 \Rightarrow 
    %         (\hat{\mathcal{A}} - a)\ket{\psi} = 0 \Rightarrow \hat{\mathcal{A}}\ket{\psi} = a\ket{\psi}
    %     \end{equation*}
    %     \begin{equation*}
    %         \bra{\psi}(\hat{\mathcal{A}} - \expval*{\hat{\mathcal{A}}})^{\dagger} = 0 \Rightarrow 
    %         \bra{\psi}(\hat{\mathcal{A}} - a)^{\dagger} = 0 \Rightarrow 
    %         \bra{\psi}\hat{\mathcal{A}}^{\dagger} = \bra{\psi}a^{\ast}
    %     \end{equation*}
    
    % Ou seja, independente do caso, caímos em uma equação de autovetores e autovalores, de modo que isso ocorre somente quanto $\ket{\psi}$ for um autovetor de $\hat{\mathcal{A}}$ e $a$ um autovalor de $\hat{\mathcal{A}}$. 
    
    % Resumindo, se a variância de um observável for nula (zero) o estado do sistema é um autoestado do operador relacionado ao observável implicando que:
    %     \begin{equation}\label{eq autoestados}
    %         \resp{
    %             \hat{\mathcal{A}}\ket{\psi} = a\ket{\psi}
    %         }
    %     \end{equation}
    
    % \begin{theorem}{}{}
    %     Os autovetores (autoestados) de um operador hermitiano são sempre ortogonais entre si. De modo que o conjunto de autoestados podem ser usados como uma base.
    % \end{theorem}
    
    % \begin{proof}
    %     Seja $\hat{\mathcal{A}}$ um operador hermitiano com dois autovalores $a_{1}$ e $a_{2}$, de modo que ele possui dois autoestados $\ket{1}$ e $\ket{2}$. Tomemos então os seguintes resultados:
    %         \begin{equation}\label{1}
    %             \bra{1}\underbrace{\hat{\mathcal{A}}\ket{2}}_{a_{2}\ket{2}} = \bra{1}a_{2}\ket{2} = a_{2}\braket{1}{2}
    %         \end{equation}
            
    %         \begin{equation}\label{2}
    %             \bra{2}\underbrace{\hat{\mathcal{A}}\ket{1}}_{a_{1}\ket{1}} = \bra{2}a_{1}\ket{1} = a_{1}\braket{2}{1}
    %         \end{equation}
    %     \begin{note}{}
    %         Aplicando o fato de ser hermitiano: $\overset{h}{=}$.
    %     \end{note}
        
    %     Calculando então o conjugado transposto de (\ref{1}):
    %         \begin{equation}\label{3}
    %             (\bra{1}\hat{\mathcal{A}}\ket{2})^{\ast} = \bra{2}\hat{\mathcal{A}}^{\dagger}\ket{1} \overset{h}{=} \bra{2}\hat{\mathcal{A}}\ket{1} = a_{1}\braket{2}{1}
    %         \end{equation}
        
    %     Vemos então que se $\hat{\mathcal{A}}$ for hermitiano:
    %         \begin{equation*}
    %             (\bra{1}\hat{\mathcal{A}}\ket{2})^{\ast} = \bra{2}\hat{\mathcal{A}}\ket{1}
    %         \end{equation*}
        
    %     Então aplicando (\ref{1}) em (\ref{3}):
    %         \begin{equation*}
    %             (a_{2}\braket{1}{2})^{\ast} = a_{1}\braket{2}{1} \Rightarrow 
    %             a_{2}\braket{2}{1} = a_{1}\braket{2}{1}
    %         \end{equation*}
    %         \begin{equation*}
    %             (a_{2} - a_{1})\braket{2}{1} = 0
    %         \end{equation*}
        
    %     Portanto, se $a_{2}\neq a_{1}$, temos obrigatoriamente que $\braket{2}{1} = 0$, ou seja os autoestados são ortogonais.
        
        
    % \end{proof}
    
    % Vamos supor agora que tenhamos um estado $\ket{\phi}$ e um observável relacionado a um dado operador $\hat{\mathcal{A}}$ de tal forma que temos um conjunto de autoestados $\ket{a_{i}}$ e autovalores $a_{i}$. Escrevendo $\ket{\phi}$ em termos dos autoestados:
    %     \begin{equation*}
    %         \ket{\phi} = \sum_{i}\phi_{i}\ket{a_{i}}
    %     \end{equation*}
    
    % Com isso, queremos saber como determinar $\expval*{\hat{\mathcal{A}}}$ no estado $\ket{\phi}$:
    %     \begin{align*}
    %         \expval*{\hat{\mathcal{A}}} &= \bra{\phi}\hat{\mathcal{A}}\ket{\phi} 
    %         = \sum_{i}\phi_{i}^{\ast}\mel**{a_{i}}{\hat{\mathcal{A}}\sum_{j}\phi_{j}}{a_{j}} \\
    %         &= \sum_{i,j}\phi_{i}^{\ast}\phi_{j}\bra{a_{i}}\hat{\mathcal{A}}\ket{a_{j}} 
    %         = \sum_{i,j}\phi_{i}^{\ast}\phi_{j}\bra{a_{i}}a_{j}\ket{a_{j}} \\
    %         &= \sum_{i,j}\phi_{i}^{\ast}\phi_{j}a_{j}\braket{a_{i}}{a_{j}} 
    %         = \sum_{i,j}\phi_{i}^{\ast}\phi_{j}a_{j}\delta_{ij} \\
    %         &= \sum_{i}\phi_{i}^{\ast}\phi_{i}a_{i}
    %     \end{align*}
    %     \begin{equation*}
    %         \resp{
    %             \expval*{\hat{\mathcal{A}}} = \sum_{i}\abs{\phi_{i}}^2a_{i}
    %         }
    %     \end{equation*}
        
    % Imaginemos então que o estado $\ket{\phi}$ esteja normalizado, ou seja $\braket{\phi}{\phi} = 1$:
    %     \begin{align*}
    %         \braket{\phi}{\phi} &= \sum_{i}\phi_{i}^{\ast}\mel**{a_{i}}{\sum_{j}\phi_{j}}{a_{j}} \\
    %         &= \sum_{i,j}\phi_{i}^{\ast}\phi_{j}\braket{a_{i}}{a_{j}} \\
    %         &= \sum_{i,j}\phi_{i}^{\ast}\phi_{j}\delta_{ij} \\
    %         &= \sum_{i}\phi_{i}^{\ast}\phi_{i}
    %     \end{align*}
    %     \begin{equation*}
    %         \resp{
    %             \braket{\phi}{\phi} = 1 = \sum_{i}\abs{\phi_{i}}^2
    %         }
    %     \end{equation*}
    
    % Isso nos dá uma ideia de média ponderada pelas probabilidades. Voltando diretamente ao processo de medidas, temos que quando fazemos uma única medida, vamos obter uma resposta dentre os possíveis valores de um observável, que são os \textit{autovalores}. Ou seja, o estado $\ket{\psi}$ se transforma no autoestado correspondente ao autovalor obtido.
    
    % \subsection{Medidas simultâneas em mecânica quântica}
    
    % \begin{example}
    %     Vamos supor que tenhamos um operador $\hat{\mathcal{A}}$ que representa um observável e possui dois autoestados atribuídos a ele:
    %         \begin{equation*}
    %             \ket{1}\rightarrow a_{1} \hspace{1cm} \& \hspace{1cm} 
    %             \ket{2}\rightarrow a_{2} 
    %         \end{equation*}
        
    %     Seja também o vetor de estado $\ket{\psi}$ conhecido e dado por uma combinação dos autoestados, tal que:
    %         \begin{equation*}
    %             \ket{\psi} = \dfrac{1}{\sqrt{5}}(2\ket{1} + \ket{2})
    %         \end{equation*}
        
    %     Escrevendo $\ket{\psi}$ como um somatório podemos explicitar as componentes desse vetor de estado:
    %         \begin{equation*}
    %             \ket{\psi} = \sum_{i}\psi_{i}\ket{i} \Rightarrow 
    %             \begin{cases}
    %             \begin{aligned}
    %                 \psi_{1} &= \dfrac{2}{\sqrt{5}} \\
    %                 \psi_{2} &= \dfrac{1}{\sqrt{5}}
    %             \end{aligned}
    %             \end{cases}
    %         \end{equation*}
        
    %     Com tudo isso, queremos saber o valor esperado do operador $\hat{\mathcal{A}}$ relacionado ao observável de interesse. Para isso, devemos primeiro saber se o vetor de estado está normalizado.
    %         \begin{align*}
    %             \braket{\psi}{\psi} &= 
    %             \dfrac{1}{\sqrt{5}}(2\bra{1} + \bra{2})\dfrac{1}{\sqrt{5}}(2\ket{1}+\ket{2}) \\
    %             &= \dfrac{1}{5}(4\braket{1}{1} + 2\braket{1}{2} + 2\braket{2}{1} + \braket{2}{2})
    %         \end{align*}
        
    %     Dado então que os vetores de base $\ket{1}$ e $\ket{2}$ são ortonormais, temos que os produtos escalares $\braket{1}{2} = \braket{2}{1} = 0$, além de que $\braket{1}{1} = \braket{2}{2} = 1$, portanto:
    %         \begin{equation*}
    %             \braket{\psi}{\psi} = \dfrac{1}{5}(4+1) = 1
    %         \end{equation*}
        
    %     Logo o vetor de estado está normalizado e não serão necessárias constantes para normalizá-lo. Calculemos então o valor esperado $\hat{\mathcal{A}}$:
    %         \begin{align*}
    %             \expval*{\hat{\mathcal{A}}} &= 
    %             \bra{\psi}\hat{\mathcal{A}}\ket{\psi} 
    %             = \dfrac{1}{\sqrt{5}}(2\bra{1} + \bra{2})\hat{\mathcal{A}}\dfrac{1}{\sqrt{5}}(2\ket{1} + \ket{2}) \\
    %             &= \dfrac{1}{5}(4\bra{1}\hat{\mathcal{A}}\ket{1} + 2\bra{1}\hat{\mathcal{A}}\ket{2} + 2\bra{2}\hat{\mathcal{A}}\ket{1} + \bra{2}\hat{\mathcal{A}}\ket{2})
    %         \end{align*}
        
    %     Mas como $\ket{1}$ e $\ket{2}$ são vetores de base que possuem autovalores relacionados, temos que:
    %         \begin{equation*}
    %             \hat{\mathcal{A}}\ket{1} = a_{1}\ket{1} \hspace{1cm} \& \hspace{1cm} 
    %             \hat{\mathcal{A}}\ket{2} = a_{2}\ket{2}
    %         \end{equation*}
        
    %     Usando isso na expressão acima, temos:
    %         \begin{align*}
    %             \expval*{\hat{\mathcal{A}}} &= 
    %             \dfrac{1}{5}(4\bra{1}a_{1}\ket{1} + 2\bra{1}a_{2}\ket{2} + 2\bra{2}a_{1}\ket{1} + \bra{2}a_{2}\ket{2}) \\
    %             &= 
    %             \dfrac{1}{5}(4a_{1}\braket{1}{1} + 2a_{2}\braket{1}{2} + 2a_{1}\braket{2}{1} + a_{2}\braket{2}{2})
    %         \end{align*}
    %         \begin{equation*}
    %             \resp{
    %                 \expval*{\hat{\mathcal{A}}} = \dfrac{1}{5}(4a_{1} + a_{2})
    %             }
    %         \end{equation*}
        
    %     Vemos então uma ideia de média ponderada entre os valores possíveis do operador. A grosso modo, podemos dizer que a cada 5 medidas feitas, 4 serão relacionadas ao autoestado $\ket{1}$ que possui como autovalor $a_{1}$ e 1 deles relacionado ao autoestado $\ket{2}$ que possui $a_{2}$ como autovalor.
    % \end{example}
    
    % \begin{note}{}
    %     Após a realização de uma medida, o vetor de estado $\ket{\psi}$ se torna um dos autoestados possíveis, de modo que se fizermos uma medida logo após a primeira medida, obteremos o mesmo resultado que está relacionado com o mesmo autoestado.
    % \end{note}
    
    % \begin{example}
    %     Vamos supor que tenhamos como base $\mathscr{B} = \{\ket{1},\ket{2}\}$, que é uma base ortonormal. Além disso, sabemos por algum motivo a forma do operador $\hat{\mathcal{B}}$ relacionado a um observável qualquer, tal que:
    %         \begin{equation*}
    %             \hat{\mathcal{B}} = 
    %             \begin{bmatrix}
    %                 0   &   1   \\
    %                 1   &   0   
    %             \end{bmatrix}
    %         \end{equation*}
        
    %     Queremos então saber que são os autoestados do sistema, de modo a satisfazer as equações:
    %         \begin{equation*}
    %             \hat{\mathcal{B}}\ket{b_{1}} = b_{1}\ket{b_{1}} \hspace{1cm} \& \hspace{1cm}
    %             \hat{\mathcal{B}}\ket{b_{2}} = b_{2}\ket{b_{2}}
    %         \end{equation*}
        
    %     De modo geral, para determinar os autovalores fazemos:
    %         \begin{equation*}
    %             \det(\hat{\mathcal{B}} - \mathfrak{b}\boldone) = 0
    %         \end{equation*}
    %     onde $\mathfrak{b}$ representa o conjunto de autovalores possíveis. Portanto:
    %         \begin{equation*}
    %             \abs{\begin{bmatrix}
    %                 0   &   1   \\
    %                 1   &   0
    %             \end{bmatrix} - 
    %             \begin{bmatrix}
    %                 \mathfrak{b}   &   0   \\
    %                 0   &   \mathfrak{b}
    %             \end{bmatrix}} = 0 \Rightarrow 
    %             \begin{vmatrix*}[r]
    %                 -\mathfrak{b}  &   1   \\
    %                 1   &   -\mathfrak{b}  
    %             \end{vmatrix*} = 0 \Rightarrow 
    %             \mathfrak{b}^2 - 1 = 0
    %         \end{equation*}
    %         \begin{equation*}
    %             \resp{b_{1} = 1} \hspace{1cm} \& \hspace{1cm}
    %             \resp{b_{2} = -1}
    %         \end{equation*}
        
    %     Para o autoestado relacionado a $b_{1}=1$ é então:
    %         \begin{equation*}
    %             \hat{\mathcal{B}}\ket{b_{1}} = 1\ket{b_{1}} \Rightarrow 
    %             \begin{bmatrix}
    %                 0   &   1   \\
    %                 1   &   0
    %             \end{bmatrix}
    %             \begin{bmatrix}
    %                 c_{1} \\
    %                 c_{2}
    %             \end{bmatrix} = 1
    %             \begin{bmatrix}
    %                 c_{1} \\
    %                 c_{2}
    %             \end{bmatrix} \Rightarrow 
    %             \begin{cases}
    %                 0\cdot c_{1} + 1\cdot c_{2} = c_{1} \\
    %                 1\cdot c_{1} + 0\cdot c_{2} = c_{2}
    %             \end{cases}
    %         \end{equation*}
    %         \begin{equation*}
    %             c_{1} = c_{2} \equiv k_{1}
    %         \end{equation*}
        
    %     Então de modo geral o autoestado relacionado a $b_{1}$ pode ser escrito como sendo:
    %         \begin{equation*}
    %             \ket{b_{1}} = k_{1}
    %             \begin{bmatrix}
    %                 1 \\ 1
    %             \end{bmatrix}
    %         \end{equation*}
        
    %     Afim de fazermos uma base ortonormal com os autoestados, podemos impor a normalização de modo que:
    %         \begin{equation*}
    %             \braket{b_{1}}{b_{1}} = 1 \Leftrightarrow k_{1}^{\ast}
    %             \begin{bmatrix}
    %                 1 & 1
    %             \end{bmatrix} k_{1}
    %             \begin{bmatrix}
    %                 1 \\ 1
    %             \end{bmatrix} = 1 \Rightarrow 2\abs{k_{1}}^2 = 1 \Rightarrow \abs{k_{1}} = \dfrac{1}{\sqrt{2}}
    %         \end{equation*} 
        
    %     Sendo assim, o autoestado $\ket{b_{1}}$ é dado simplesmente por:
    %         \begin{equation*}
    %             \resp{
    %                 \ket{b_{1}} = \dfrac{1}{\sqrt{2}}
    %                 \begin{bmatrix}
    %                     1 \\ 1
    %                 \end{bmatrix}
    %             }
    %         \end{equation*}
        
    %     Analogamente para $b_{2} = -1$, temos que:
    %         \begin{equation*}
    %             \hat{\mathcal{B}}\ket{b_{2}} = -1\ket{b_{2}} \Rightarrow 
    %             \begin{bmatrix}
    %                 0   &   1   \\
    %                 1   &   0
    %             \end{bmatrix}
    %             \begin{bmatrix}
    %                 c_{3} \\
    %                 c_{4}
    %             \end{bmatrix} = -1
    %             \begin{bmatrix}
    %                 c_{3} \\
    %                 c_{4}
    %             \end{bmatrix} \Rightarrow 
    %             \begin{cases}
    %                 0\cdot c_{3} + 1\cdot c_{4} = -c_{3} \\
    %                 1\cdot c_{3} + 0\cdot c_{4} = -c_{4}
    %             \end{cases}
    %         \end{equation*}
    %         \begin{equation*}
    %             c_{3} = -c_{4} \equiv k_{2}
    %         \end{equation*}
        
    %     Portanto:
    %         \begin{equation*}
    %             \ket{b_{2}} = k_{2}
    %             \begin{bmatrix*}[r]
    %                 1 \\ -1
    %             \end{bmatrix*}
    %         \end{equation*}
        
    %     Normalizando:
    %         \begin{equation*}
    %             \braket{b_{2}}{b_{2}} = 1 \Leftrightarrow k_{2}^{\ast}
    %             \begin{bmatrix}
    %                 1 & -1
    %             \end{bmatrix} k_{2}
    %             \begin{bmatrix*}[r]
    %                 1 \\ -1
    %             \end{bmatrix*} = 1 \Rightarrow 2\abs{k_{2}}^2 = 1 \Rightarrow \abs{k_{2}} = \dfrac{1}{\sqrt{2}}
    %         \end{equation*}
        
    %     Logo, o autoestado $\ket{b_{2}}$ é dado por:
    %         \begin{equation*}
    %             \resp{
    %                 \ket{b_{2}} = \dfrac{1}{\sqrt{2}}
    %                 \begin{bmatrix*}[r]
    %                     1 \\ -1
    %                 \end{bmatrix*}
    %             }
    %         \end{equation*}
        
    %     \begin{note}{}
    %         Podemos escrever tanto o operador $\hat{\mathcal{B}}$ quanto os autoestados $\ket{b_{1}}$ e $\ket{b_{2}}$ na forma de combinações lineares de \textit{bra's} e \textit{ket's}, tal que:
    %             \begin{equation*}
    %                 \hat{\mathcal{B}} = \ket{2}\bra{1} + \ket{1}\bra{2}
    %             \end{equation*}
    %             \begin{equation*}
    %                 \ket{b_{1}} = \dfrac{1}{\sqrt{2}}(\ket{1} + \ket{2}) \hspace{1cm} \& \hspace{1cm}
    %                 \ket{b_{2}} = \dfrac{1}{\sqrt{2}}(\ket{1} - \ket{2})
    %             \end{equation*}
    %     \end{note}
        
    % \end{example}
    
    % Dados os exemplos acima, podemos fazer uma análise simples sobre o processo de medida em mecânica quântica. Olhando para o Exemplo 5, temos os autoestados $\ket{1}$ e $\ket{2}$ cujos autovalores relacionados são respectivamente $a_{1}$ e $a_{2}$, já no Exemplo 6 temos os autoestados $\ket{b_{1}}$ e $\ket{b_{2}}$ compostos por uma combinação linear entre $\ket{1}$ e $\ket{2}$, e possuem autovalores $b_{1}$ e $b_{2}$.
    
    % Vamos então supor que exista um aparelho que meça o operador $\hat{\mathcal{B}}$, de modo que ele nos retorna um dos autovalores do operador:
    %     \begin{figure}[H]
    %         \centering
    %         \includegraphics{Figuras/Operators measurement.pdf}
    %         \caption{Representação esquemática de como um operador atua em sob um estado quântico.}
    %         \label{Operators measurement}
    %     \end{figure}
    
    % Então se obtivermos $b_{1}=1$, teremos associado o autoestado $\ket{b_{1}}$ que é composto pelos vetores $\ket{1}$ e $\ket{2}$, e isto nos indica que não podemos medir os dois operadores simultaneamente, pois quando fazemos a medida de $\hat{\mathcal{B}}$, não conseguimos determinar qual dos autoestados de $\hat{\mathcal{A}}$ teremos, isto pelo simples motivo de que $\ket{b_{1}}$ depende de $\ket{1}$ \textit{\textbf{e}} $\ket{2}$. 
    
    % Da mesma forma, podemos manipular $\ket{b_{1}}$ e $\ket{b_{2}}$ para isolarmos $\ket{1}$ e $\ket{2}$, tal que estes dependerão de $\ket{b_{1}}$ e $\ket{b_{2}}$, o que gera uma espécie de looping quando tentamos medir os dois operadores ao mesmo tempo.
    
    % Sabemos que com base em um vetor de estado $\ket{\psi}$, podemos calcular os valores esperados dos operadores $\hat{\mathcal{A}}$ e $\hat{\mathcal{B}}$ como sendo:
    %     \begin{equation*}
    %         \expval*{\hat{\mathcal{A}}} = \bra{\psi}\hat{\mathcal{A}}\ket{\psi} \hspace{1cm} \& \hspace{1cm}
    %         \expval*{\hat{\mathcal{B}}} = \bra{\psi}\hat{\mathcal{B}}\ket{\psi}
    %     \end{equation*}
    
    % Com isso, medir as variâncias $\sigma_{\hat{\mathcal{A}}}^2$ e $\sigma_{\hat{\mathcal{B}}}^{2}$ fica inteiramente determinada, já que:
    %     \begin{equation*}
    %         \sigma_{\hat{\mathcal{A}}}^2 = \expval*{(\hat{\mathcal{A}} - \expval*{\hat{\mathcal{A}}})^2} \hspace{1cm} \& \hspace{1cm} 
    %         \sigma_{\hat{\mathcal{B}}}^2 = \expval*{(\hat{\mathcal{B}} - \expval*{\hat{\mathcal{B}}})^2}
    %     \end{equation*}
    
    % As variâncias vão nos dizes o quão boas são as medidas, se comparadas com o valor esperado, dessa forma, calcular o produto entre as variâncias nos fornecerá informações extras sobre as medidas. Antes disso, temos:
    %     \begin{align*}
    %         \sigma_{\hat{\mathcal{A}}}^2 &= \expval*{(\hat{\mathcal{A}} - \expval*{\hat{\mathcal{A}}})^2} \\
    %         &= \bra{\psi}(\hat{\mathcal{A}} - \expval*{\hat{\mathcal{A}}})^2\ket{\psi} \\
    %         &= \bra{\psi}(\hat{\mathcal{A}} - \expval*{\hat{\mathcal{A}}})^{\dagger}(\hat{\mathcal{A}} - \expval*{\hat{\mathcal{A}}})\ket{\psi} \\
    %         &= \braket{f}{f}
    %     \end{align*}
    % onde $\ket{f} \coloneqq (\hat{\mathcal{A}} - \expval*{\hat{\mathcal{A}}})\ket{\psi}$. Analogamente para $\sigma_{\hat{\mathcal{B}}}^2$:
    %     \begin{align*}
    %         \sigma_{\hat{\mathcal{B}}}^2 &= \expval*{(\hat{\mathcal{B}} - \expval*{\hat{\mathcal{B}}})^2} \\
    %         &= \bra{\psi}(\hat{\mathcal{B}} - \expval*{\hat{\mathcal{B}}})^2\ket{\psi} \\
    %         &= \bra{\psi}(\hat{\mathcal{B}} - \expval*{\hat{\mathcal{B}}})^{\dagger}(\hat{\mathcal{B}} - \expval*{\hat{\mathcal{B}}})\ket{\psi} \\
    %         &= \braket{g}{g}
    %     \end{align*}
    % onde $\ket{g} \coloneqq (\hat{\mathcal{B}} - \expval*{\hat{\mathcal{B}}})\ket{\psi}$. Então o produto entre as variâncias pode ser escrito simplesmente por:
    %     \begin{equation*}
    %         \sigma_{\hat{\mathcal{A}}}^2\sigma_{\hat{\mathcal{B}}}^2 = \braket{f}{f}\braket{g}{g}
    %     \end{equation*}
    
    % Porém, se não soubermos o vetor de estado $\ket{\psi}$, não conseguimos determinar as variâncias, tampouco o produto entre elas, dessa forma, utilizaremos a \textit{desigualdade de Cauchy-Schwarz}, dada por:
    %     \begin{equation}\label{desigualdade de CS}
    %         \resp{\braket{\alpha}{\alpha}\braket{\beta}{\beta} \geqslant \abs{\braket{\alpha}{\beta}}^2}
    %     \end{equation}
    %     \begin{proof}
    %         Dados dois vetores quaisquer $\ket{\alpha}$ e $\ket{\beta}$, podemos construir um vetor $\ket{\gamma}$ utilizando Gram-Schmidt tal que:
    %             \begin{equation*}
    %                 \ket{\gamma} = \ket{\beta} - \dfrac{\braket{\alpha}{\beta}}{\braket{\alpha}{\alpha}}\ket{\alpha}
    %             \end{equation*}
            
    %         De modo que impomos $\braket{\alpha}{\alpha}>0$. Calculando então o produto escalar $\braket{\beta}{\gamma}$, temos:
    %             \begin{align*}
    %                 \braket{\beta}{\gamma} &= 
    %                 \bra{\beta}\left(
    %                     \ket{\beta} - \dfrac{\braket{\alpha}{\beta}}{\braket{\alpha}{\alpha}}\ket{\alpha}
    %                 \right) \\
    %                 &= \braket{\beta}{\beta} - \dfrac{\braket{\alpha}{\beta}}{\braket{\alpha}{\alpha}}\braket{\beta}{\alpha} \\
    %                 &= \braket{\beta}{\beta} - \dfrac{\braket{\alpha}{\beta}}{\braket{\alpha}{\alpha}}(\braket{\alpha}{\beta})^{\dagger} \\
    %                 &= \braket{\beta}{\beta} - \dfrac{\abs{\braket{\alpha}{\beta}}^2}{\braket{\alpha}{\alpha}}
    %             \end{align*}
            
    %         Sendo então $\braket{\beta}{\gamma}\geqslant0$, podemos multiplicar ambos os lados por $\braket{\alpha}{\alpha}$ e concluir que:
    %             \begin{equation*}
    %                 \braket{\beta}{\gamma} \braket{\alpha}{\alpha} \geqslant 0 
    %             \end{equation*}
            
    %         Portanto:
    %             \begin{equation*}
    %                 \braket{\alpha}{\alpha}\braket{\beta}{\beta} \geqslant\abs{\braket{\alpha}{\beta}}^2
    %             \end{equation*}
    %     \end{proof}
    
    % Então aplicando (\ref{desigualdade de CS}):
    %     \begin{align*}
    %         \sigma_{\hat{\mathcal{A}}}^2\sigma_{\hat{\mathcal{B}}}^2 = 
    %         \braket{f}{f}\braket{g}{g} &\geqslant \abs{\braket{f}{g}}^2 \\
    %         &= \abs{\bra{\psi}(\hat{\mathcal{A}} - \expval*{\hat{\mathcal{A}}})^{\dagger}(\hat{\mathcal{B}} - \expval*{\hat{\mathcal{B}}})\ket{\psi}}^2 \\
    %         &= \abs{\bra{\psi}(\hat{\mathcal{A}} - \expval*{\hat{\mathcal{A}}})(\hat{\mathcal{B}} - \expval*{\hat{\mathcal{B}}})\ket{\psi}}^2 \\ 
    %         &= \abs{\bra{\psi} \hat{\mathcal{A}}\hat{\mathcal{B}} - \expval*{\hat{\mathcal{A}}}\hat{\mathcal{B}} - \expval*{\hat{\mathcal{B}}}\hat{\mathcal{A}} + \expval*{\hat{\mathcal{A}}}\expval*{\hat{\mathcal{B}}} \ket{\psi}}^2 \\
    %         &= \abs{
    %         \bra{\psi}\hat{\mathcal{A}}\hat{\mathcal{B}}\ket{\psi} - 
    %         \expval*{\hat{\mathcal{A}}}\bra{\psi}\hat{\mathcal{B}}\ket{\psi} - 
    %         \expval*{\hat{\mathcal{B}}}\bra{\psi}\hat{\mathcal{A}}\ket{\psi} +
    %         \expval*{\hat{\mathcal{A}}}\expval*{\hat{\mathcal{B}}}\braket{\psi}{\psi}
    %         }^2 \\
    %         &= \abs{
    %         \bra{\psi}\hat{\mathcal{A}}\hat{\mathcal{B}} - \expval*{\hat{\mathcal{A}}}\expval*{\hat{\mathcal{B}}}\ket{\psi} - 
    %         \expval*{\hat{\mathcal{A}}}\expval*{\hat{\mathcal{B}}} + 
    %         \expval*{\hat{\mathcal{B}}}\expval*{\hat{\mathcal{A}}}
    %         }^2 \\
    %         &= \abs{\bra{\psi}\hat{\mathcal{A}}\hat{\mathcal{B}} - \expval*{\hat{\mathcal{A}}}\expval*{\hat{\mathcal{B}}}\ket{\psi}}^2
    %     \end{align*}
    
    % Seja então $\hat{\mathcal{Z}} \coloneqq \hat{\mathcal{A}}\hat{\mathcal{B}} - \expval*{\hat{\mathcal{A}}}\expval*{\hat{\mathcal{B}}}$ um novo operador, tal que ele não é necessariamente hermitiano, ou seja, $\expval*{\hat{\mathcal{Z}}}\in\mathbb{C}$:
    %     \begin{equation*}
    %         \expval*{\hat{\mathcal{Z}}} = \mathfrak{Re}[\expval*{\hat{\mathcal{Z}}}] + i\mathfrak{Im}[\expval*{\hat{\mathcal{Z}}}]
    %     \end{equation*}
    
    % Temos portanto que:
    %     \begin{equation*}
    %          \sigma_{\hat{\mathcal{A}}}^2\sigma_{\hat{\mathcal{B}}}^2 \geqslant \abs{\bra{\psi}\hat{\mathcal{Z}}\ket{\psi}}^2 = \abs{\expval*{\hat{\mathcal{Z}}}}^2 \geqslant \abs{\mathfrak{Im}[\expval*{\hat{\mathcal{Z}}}]}^2
    %     \end{equation*}
    
    % Lembrando que a parte imaginária de um número complexo $z$ pode ser escrita como sendo:
    %     \begin{equation*}
    %         \mathfrak{Im}[z] = \dfrac{1}{2i}(z - z^{\ast})
    %     \end{equation*}
    
    % Portanto:
    %     \begin{align*}
    %         \sigma_{\hat{\mathcal{A}}}^2\sigma_{\hat{\mathcal{B}}}^2 &\geqslant 
    %         \abs{\dfrac{1}{2i}(\expval*{\hat{\mathcal{Z}}} - \expval*{\hat{\mathcal{Z}}^{\ast}})}^2 \\
    %         &= \abs{\dfrac{1}{2i}(\bra{\psi}\hat{\mathcal{A}}\hat{\mathcal{B}} - \expval*{\hat{\mathcal{A}}}\expval*{\hat{\mathcal{B}}} - \hat{\mathcal{B}}^{\dagger}\hat{\mathcal{A}}^{\dagger} +\expval*{\hat{\mathcal{A}}}\expval*{\hat{\mathcal{B}}}\ket{\psi})}^2 \\
    %         &= \abs{\dfrac{1}{2i}\bra{\psi}\hat{\mathcal{A}}\hat{\mathcal{B}} - \hat{\mathcal{B}}\hat{\mathcal{A}} \ket{\psi}}^2 \\
    %         &= \abs{
    %             \dfrac{1}{2i}\bra{\psi}[\hat{\mathcal{A}},\hat{\mathcal{B}}]\ket{\psi}
    %         }^2 \\
    %         &= \dfrac{1}{2^2}\abs{\expval*{[\hat{\mathcal{A}},\hat{\mathcal{B}}]}}^2
    %     \end{align*}
    
    % Logo, tirando a raiz quadrada em ambos os lados, obtemos o \textit{princípio da incerteza entre dois operadores}:
    %     \begin{equation}\label{principio da incerteza operadores}
    %         \resp{
    %             \sigma_{\hat{\mathcal{A}}}\sigma_{\hat{\mathcal{B}}} \geqslant \dfrac{1}{2}\abs{\expval*{[\hat{\mathcal{A}},\hat{\mathcal{B}}]}}
    %         }
    %     \end{equation}
    
    % Esse resultado é importante devido ao fato de que caso os operadores comutem entre si, existe a possibilidade de medirmos simultaneamente os observáveis relacionados. No entanto, se não comutarem, eles nunca poderão ser medidos de forma simultânea, mesmo que o aparato de medida seja o mais tecnológico de todos. Essa restrição é intrínseca à relação de comutação entre os operadores e nada pode mudar isso.

    % Para a situação de precisão infinita temos um importante teorema a destacar:
    % \begin{theorem}{}{}
    %     Dados dois operadores $\mathcal{\hat{A}}$ e $\mathcal{\hat{B}}$ que comutam entre si:
    %     \begin{equation*}
    %         [\mathcal{\hat{A}},\mathcal{\hat{B}}] = 0
    %     \end{equation*}
    %     Sempre existe uma base de autoestados $\{\ket{\psi}\}$ comum tanto para $\mathcal{\hat{A}}$ quanto para $\mathcal{\hat{B}}$.
    % \end{theorem}
    % \begin{proof}
    %     Seja $\{\ket{\psi}\}$ uma base de autoestados de $\mathcal{\hat{A}}$ sem autovalores degenerados, isto é,
    %     \begin{equation*}
    %         \mathcal{\hat{A}}\ket{\psi} = a\ket{\psi}
    %     \end{equation*}
    %     Como $\mathcal{\hat{A}}$ e $\mathcal{\hat{B}}$ comutam, podemos escrever
    %     \begin{equation*}
    %         \mathcal{\hat{A}}\mathcal{\hat{B}} = \mathcal{\hat{B}}\mathcal{\hat{A}}
    %     \end{equation*}
    %     Podemos escrever portanto
    %     \begin{equation*}
    %         \mathcal{\hat{A}}\mathcal{\hat{B}}\ket{\psi} = \mathcal{\hat{B}}\mathcal{\hat{A}}\ket{\psi} = \mathcal{\hat{B}}\left(a\ket{\psi}\right) = a\mathcal{\hat{B}}\ket{\psi}
    %     \end{equation*}
    %     \begin{equation*}
    %         \mathcal{\hat{A}}(\mathcal{\hat{B}}\ket{\psi}) = a(\mathcal{\hat{B}}\ket{\psi})
    %     \end{equation*}
    %     ou seja, $\mathcal{\hat{B}}\ket{\psi}$ é um autoestado de $\mathcal{\hat{A}}$ associado ao mesmo autovalor $a$. Isso significa que, dado que $a$ não é degenerado, $\mathcal{\hat{B}}\ket{\psi}$ deve ser o mesmo autoestado que $\ket{\psi}$, o que só pode ocorrer se um for o outro a menos de uma constante $b$:
    %     \begin{equation*}
    %         \mathcal{\hat{B}}\ket{\psi} = b\ket{\psi}
    %     \end{equation*}
    %     o que constitui uma equação de autovalores, concluindo nossa demonstração.
    % \end{proof}
        
            \subsection{Operadores hermitianos}
                    Um fato importante sobre os observáveis é que o valor esperado de um operador que se relaciona diretamente a ele deve sempre ser um número real, isto é, para um dado evento $A$ que está associado a um operador $\hat{\mathcal{A}}$, temos que:
        \begin{equation*}
            \expval*{\hat{\mathcal{A}}}^{\ast} = \expval*{\hat{\mathcal{A}}} \in\mathbb{R}
        \end{equation*}
    
    Como consequência, podemos escrever:
        \begin{align*}
            \expval*{\hat{\mathcal{A}}}^{\ast} = (\bra{\psi}\hat{\mathcal{A}}\ket{\psi})^{\ast} &\eq \bra{\psi}^{\ast}\hat{\mathcal{A}}^{\ast}\ket{\psi}^{\ast} \\
            &\eq (\hat{\mathcal{A}}^{\ast}\ket{\psi}^{\ast})^{\mathtt{T}}(\bra{\psi}^{\ast})^{\mathtt{T}} \\
            &\eq \bra{\psi}\hat{\mathcal{A}}^{\dagger}\ket{\psi}
        \end{align*}
        \begin{align*}
            \expval*{\hat{\mathcal{A}}} = \bra{\psi}\hat{\mathcal{A}}\ket{\psi}
        \end{align*}
    
    Mas então, como $\expval*{\hat{\mathcal{A}}}^{\ast} = \expval*{\hat{\mathcal{A}}}$, podemos concluir que quando um operador está associado a um observável, vale que:
        \begin{answer}\label{eq: operadores hermitianos}
            \hat{\mathcal{A}}^{\dagger} = \hat{\mathcal{A}}
        \end{answer}
    
    Esses operadores são denominados \textit{operadores hermitianos} e são de suma importância para a realização de uma medida em mecânica quântica, de modo que operadores hermitianos são condição necessária para caracterizar um observável.
    
    Vamos supor então que iremos realizar um experimento qualquer no qual pretendemos medir um observável relacionado a um operador $\hat{\mathcal{A}}$, de tal forma que conseguimos determinar o valor esperado desse operador ($\expval*{\hat{\mathcal{A}}} = a$). Nesse experimento, obtemos que todas as medidas são equivalentes ao valor esperado, isto implica diretamente que a variância do valor esperado de $\hat{\mathcal{A}}$ é zero, ou seja:
        \begin{equation*}
            \sigma^2 = \expval*{(\hat{\mathcal{A}} - \expval*{\hat{\mathcal{A}}})^2} = 0 \Rightarrow \bra{\psi}(\hat{\mathcal{A}} - \expval*{\hat{\mathcal{A}}})^2\ket{\psi} = 0
        \end{equation*}
    
    Expandindo essa relação, temos que:
        \begin{equation*}
            \bra{\psi}(\hat{\mathcal{A}} - \expval*{\hat{\mathcal{A}}})^{\dagger}(\hat{\mathcal{A}} - \expval*{\hat{\mathcal{A}}})\ket{\psi} = 0
        \end{equation*}
    
    De tal forma que podemos definir:
        \begin{equation*}
            \bra{\phi} \coloneqq \bra{\psi}(\hat{\mathcal{A}} - \expval*{\hat{\mathcal{A}}})^{\dagger} \qquad \& \qquad 
            \ket{\phi} \coloneqq (\hat{\mathcal{A}} - \expval*{\hat{\mathcal{A}}})\ket{\psi}
        \end{equation*}
    
    Portanto, para que a equação seja verdadeira:
        \begin{equation*}
            \bra{\phi} = 0 \qquad \textrm{ou} \qquad \ket{\phi} = 0
        \end{equation*}
    
    O que implica diretamente em:
        \begin{equation*}
            (\hat{\mathcal{A}} - \expval*{\hat{\mathcal{A}}})\ket{\psi} = 0 \Rightarrow 
            (\hat{\mathcal{A}} - a)\ket{\psi} = 0 \Rightarrow \hat{\mathcal{A}}\ket{\psi} = a\ket{\psi}
        \end{equation*}
        \begin{equation*}
            \bra{\psi}(\hat{\mathcal{A}} - \expval*{\hat{\mathcal{A}}})^{\dagger} = 0 \Rightarrow 
            \bra{\psi}(\hat{\mathcal{A}} - a)^{\dagger} = 0 \Rightarrow 
            \bra{\psi}\hat{\mathcal{A}}^{\dagger} = \bra{\psi}a^{\ast}
        \end{equation*}
    
    Ou seja, independente do caso, caímos em uma equação de autovetores e autovalores, de modo que isso ocorre somente quanto $\ket{\psi}$ for um autovetor de $\hat{\mathcal{A}}$ e $a$ um autovalor de $\hat{\mathcal{A}}$. 
    
    Resumindo, se a variância de um observável for nula (zero) o estado do sistema é um autoestado do operador relacionado ao observável implicando que:
        \begin{answer}\label{eq: autoestados}
                \hat{\mathcal{A}}\ket{\psi} = a\ket{\psi}
        \end{answer}
    
    \begin{theorem}{Ortogonalidade de operadores hermitianos}{}
        Seja $\hat{\mathcal{A}}$ um operador hermitiano arbitrário, com autovalores $a_{1},a_{2}\in\mathbb{R}$ e dois autovetores (autoestados) $\ket{1},\ket{2}\in\mathscr{H}$, onde $\mathscr{H}$ é um espaço de Hilbert arbitrário. Os autoestados deste operador são sempre ortogonais, ou seja
            \begin{equation*}
                \braket{1}{2} = \braket{2}{1} = 0
            \end{equation*}
    \end{theorem}
    
    \begin{proof}
        Tomemos então os seguintes resultados:
            \begin{equation}\label{eq: 1}
                \bra{1}\underbrace{\hat{\mathcal{A}}\ket{2}}_{a_{2}\ket{2}} = \bra{1}a_{2}\ket{2} = a_{2}\braket{1}{2}
            \end{equation}
            
            \begin{equation}\label{eq: 2}
                \bra{2}\underbrace{\hat{\mathcal{A}}\ket{1}}_{a_{1}\ket{1}} = \bra{2}a_{1}\ket{1} = a_{1}\braket{2}{1}
            \end{equation}
        \begin{note}{}
            Aplicando o fato de ser hermitiano: $\overset{h}{=}$.
        \end{note}
        
        Calculando então o conjugado transposto de \eqref{eq: 1}:
            \begin{equation}\label{eq: 3}
                (\bra{1}\hat{\mathcal{A}}\ket{2})^{\ast} = \bra{2}\hat{\mathcal{A}}^{\dagger}\ket{1} \overset{h}{=} \bra{2}\hat{\mathcal{A}}\ket{1} = a_{1}\braket{2}{1}
            \end{equation}
        
        Vemos então que se $\hat{\mathcal{A}}$ for hermitiano:
            \begin{equation*}
                (\bra{1}\hat{\mathcal{A}}\ket{2})^{\ast} = \bra{2}\hat{\mathcal{A}}\ket{1}
            \end{equation*}
        
        Então aplicando \eqref{eq: 1} em \ref{eq: 3}:
            \begin{equation*}
                (a_{2}\braket{1}{2})^{\ast} = a_{1}\braket{2}{1} \Rightarrow 
                a_{2}\braket{2}{1} = a_{1}\braket{2}{1}
            \end{equation*}
            \begin{equation*}
                (a_{2} - a_{1})\braket{2}{1} = 0
            \end{equation*}
        Caso $a_{2} = a_{1}$, temos um caso trivial, pois se isso ocorrer teremos que $\ket{1} = \ket{2}$, que claramente é um caso irrelevante.
        Portanto, se $a_{2}\neq a_{1}$, temos obrigatoriamente que $\braket{2}{1} = 0$, ou seja os autoestados são ortogonais.
    \end{proof}

    \begin{corollary}\label{cor: base de autoestados}
        O conjunto de autoestados $\{\ket{i}\}$, com $i=1,2$, pode ser utilizado como uma base
    \end{corollary}
    
    Vamos supor agora que tenhamos um estado $\ket{\phi}$ e um observável relacionado a um dado operador $\hat{\mathcal{A}}$ de tal forma que temos um conjunto de autoestados $\ket{a_{i}}$ e autovalores $a_{i}$. Escrevendo $\ket{\phi}$ em termos dos autoestados:
        \begin{equation*}
            \ket{\phi} = \sum_{i}\phi_{i}\ket{a_{i}}
        \end{equation*}
    
    Com isso, queremos saber como determinar $\expval*{\hat{\mathcal{A}}}$ no estado $\ket{\phi}$:
        \begin{align*}
            \expval*{\hat{\mathcal{A}}} &\eq \bra{\phi}\hat{\mathcal{A}}\ket{\phi} 
            = \sum_{i}\phi_{i}^{\ast}\bra{a_{i}}\hat{\mathcal{A}}\sum_{j}\phi_{j}\ket{a_{j}} \\
            &\eq \sum_{i,j}\phi_{i}^{\ast}\phi_{j}\bra{a_{i}}\hat{\mathcal{A}}\ket{a_{j}} 
            = \sum_{i,j}\phi_{i}^{\ast}\phi_{j}\bra{a_{i}}a_{j}\ket{a_{j}} \\
            &\eq \sum_{i,j}\phi_{i}^{\ast}\phi_{j}a_{j}\braket{a_{i}}{a_{j}} 
            = \sum_{i,j}\phi_{i}^{\ast}\phi_{j}a_{j}\delta_{ij} \\
            &\eq \sum_{i}\phi_{i}^{\ast}\phi_{i}a_{i}
        \end{align*}
        \begin{answer*}
            \expval*{\hat{\mathcal{A}}} = \sum_{i}\abs{\phi_{i}}^2a_{i}
        \end{answer*}
        
    Imaginemos então que o estado $\ket{\phi}$ esteja normalizado, ou seja $\braket{\phi}{\phi} = 1$:
        \begin{align*}
            \braket{\phi}{\phi} &\eq \sum_{i}\phi_{i}^{\ast}\bra{a_{i}}\sum_{j}\phi_{j}\ket{a_{j}} \\
            &\eq \sum_{i,j}\phi_{i}^{\ast}\phi_{j}\braket{a_{i}}{a_{j}} \\
            &\eq \sum_{i,j}\phi_{i}^{\ast}\phi_{j}\delta_{ij} \\
            &\eq \sum_{i}\phi_{i}^{\ast}\phi_{i}
        \end{align*}
        \begin{answer*}
            \braket{\phi}{\phi} = 1 = \sum_{i}\abs{\phi_{i}}^2
        \end{answer*}
    
    Isso nos dá uma ideia de média ponderada pelas probabilidades. Voltando diretamente ao processo de medidas, temos que quando fazemos uma única medida, vamos obter uma resposta dentre os possíveis valores de um observável, que são os \textit{autovalores}. Ou seja, o estado $\ket{\psi}$ se transforma no autoestado correspondente ao autovalor obtido.
    

            \subsection{Medidas simultâneas}
                    Para iniciar o assunto sobre a realização de medidas simultâneas em mecânica quântica, comecemos com um simples exemplo
    
    \begin{example}
        Vamos supor que tenhamos um operador $\hat{\mathcal{A}}$ que representa um observável e possui dois autoestados atribuídos a ele:
            \begin{equation*}
                \ket{1}\rightarrow a_{1} \qquad \& \qquad 
                \ket{2}\rightarrow a_{2} 
            \end{equation*}
        
        Seja também o vetor de estado $\ket{\psi}$ conhecido e dado por uma combinação dos autoestados, tal que:
            \begin{equation*}
                \ket{\psi} = \dfrac{1}{\sqrt{5}}(2\ket{1} + \ket{2})
            \end{equation*}
        
        Escrevendo $\ket{\psi}$ como um somatório podemos explicitar as componentes desse vetor de estado:
            \begin{equation*}
                \ket{\psi} = \sum_{i}\psi_{i}\ket{i} \Rightarrow 
                \begin{cases}
                \begin{aligned}
                    \psi_{1} &= \dfrac{2}{\sqrt{5}} \\
                    \psi_{2} &= \dfrac{1}{\sqrt{5}}
                \end{aligned}
                \end{cases}
            \end{equation*}
        
        Com tudo isso, queremos saber o valor esperado do operador $\hat{\mathcal{A}}$ relacionado ao observável de interesse. Para isso, devemos primeiro saber se o vetor de estado está normalizado.
            \begin{align*}
                \braket{\psi}{\psi} &= 
                \dfrac{1}{\sqrt{5}}(2\bra{1} + \bra{2})\dfrac{1}{\sqrt{5}}(2\ket{1}+\ket{2}) \\
                &= \dfrac{1}{5}(4\braket{1}{1} + 2\braket{1}{2} + 2\braket{2}{1} + \braket{2}{2})
            \end{align*}
        
        Dado então que os vetores de base $\ket{1}$ e $\ket{2}$ são ortonormais, temos que os produtos escalares $\braket{1}{2} = \braket{2}{1} = 0$, além de que $\braket{1}{1} = \braket{2}{2} = 1$, portanto:
            \begin{equation*}
                \braket{\psi}{\psi} = \dfrac{1}{5}(4+1) = 1
            \end{equation*}
        
        Logo o vetor de estado está normalizado e não serão necessárias constantes para normalizá-lo. Calculemos então o valor esperado $\hat{\mathcal{A}}$:
            \begin{align*}
                \expval*{\hat{\mathcal{A}}} &\eq 
                \bra{\psi}\hat{\mathcal{A}}\ket{\psi} 
                = \dfrac{1}{\sqrt{5}}(2\bra{1} + \bra{2})\hat{\mathcal{A}}\dfrac{1}{\sqrt{5}}(2\ket{1} + \ket{2}) \\
                &\eq \dfrac{1}{5}(4\bra{1}\hat{\mathcal{A}}\ket{1} + 2\bra{1}\hat{\mathcal{A}}\ket{2} + 2\bra{2}\hat{\mathcal{A}}\ket{1} + \bra{2}\hat{\mathcal{A}}\ket{2})
            \end{align*}
        
        Mas como $\ket{1}$ e $\ket{2}$ são vetores de base que possuem autovalores relacionados, temos que:
            \begin{equation*}
                \hat{\mathcal{A}}\ket{1} = a_{1}\ket{1} \qquad \& \qquad 
                \hat{\mathcal{A}}\ket{2} = a_{2}\ket{2}
            \end{equation*}
        
        Usando isso na expressão acima, temos:
            \begin{align*}
                \expval*{\hat{\mathcal{A}}} &\eq 
                \dfrac{1}{5}(4\bra{1}a_{1}\ket{1} + 2\bra{1}a_{2}\ket{2} + 2\bra{2}a_{1}\ket{1} + \bra{2}a_{2}\ket{2}) \\
                &\eq 
                \dfrac{1}{5}(4a_{1}\braket{1}{1} + 2a_{2}\braket{1}{2} + 2a_{1}\braket{2}{1} + a_{2}\braket{2}{2})
            \end{align*}
            \begin{answer*}
                    \expval*{\hat{\mathcal{A}}} = \dfrac{1}{5}(4a_{1} + a_{2})
            \end{answer*}
        
        Vemos então uma ideia de média ponderada entre os valores possíveis do operador. A grosso modo, podemos dizer que a cada 5 medidas feitas, 4 serão relacionadas ao autoestado $\ket{1}$ que possui como autovalor $a_{1}$ e 1 deles relacionado ao autoestado $\ket{2}$ que possui $a_{2}$ como autovalor.
    \end{example}
    
    \begin{note}{}
        Após a realização de uma medida, o vetor de estado $\ket{\psi}$ se torna um dos autoestados possíveis, de modo que se fizermos uma medida logo após a primeira medida, obteremos o mesmo resultado que está relacionado com o mesmo autoestado.
    \end{note}
    
    \begin{example}
        Vamos supor que tenhamos como base $\mathscr{B} = \{\ket{1},\ket{2}\}$, que é uma base ortonormal. Além disso, sabemos por algum motivo a forma do operador $\hat{\mathcal{B}}$ relacionado a um observável qualquer, tal que:
            \begin{equation*}
                \hat{\mathcal{B}} = 
                \begin{bmatrix}
                    0   &   1   \\
                    1   &   0   
                \end{bmatrix}
            \end{equation*}
        
        Queremos então saber que são os autoestados do sistema, de modo a satisfazer as equações:
            \begin{equation*}
                \hat{\mathcal{B}}\ket{b_{1}} = b_{1}\ket{b_{1}} \qquad \& \qquad
                \hat{\mathcal{B}}\ket{b_{2}} = b_{2}\ket{b_{2}}
            \end{equation*}
        
        De modo geral, para determinar os autovalores fazemos:
            \begin{equation*}
                \det(\hat{\mathcal{B}} - \mathfrak{b}\boldone) = 0
            \end{equation*}
        onde $\mathfrak{b}$ representa o conjunto de autovalores possíveis. Portanto:
            \begin{equation*}
                \abs{\begin{bmatrix}
                    0   &   1   \\
                    1   &   0
                \end{bmatrix} - 
                \begin{bmatrix}
                    \mathfrak{b}   &   0   \\
                    0   &   \mathfrak{b}
                \end{bmatrix}} = 0 \Rightarrow 
                \begin{vmatrix*}[r]
                    -\mathfrak{b}  &   1   \\
                    1   &   -\mathfrak{b}  
                \end{vmatrix*} = 0 \Rightarrow 
                \mathfrak{b}^2 - 1 = 0
            \end{equation*}
            \begin{minipage}{0.45\linewidth}
                \begin{answer*}
                    b_{1} = 1
                \end{answer*}
            \end{minipage}
            \begin{minipage}{0.45\linewidth}
                \begin{answer*}
                    b_{2} = -1
                \end{answer*}
            \end{minipage}\medskip
        
        Para o autoestado relacionado a $b_{1}=1$ é então:
            \begin{equation*}
                \hat{\mathcal{B}}\ket{b_{1}} = 1\ket{b_{1}} \Rightarrow 
                \begin{bmatrix}
                    0   &   1   \\
                    1   &   0
                \end{bmatrix}
                \begin{bmatrix}
                    c_{1} \\
                    c_{2}
                \end{bmatrix} = 1
                \begin{bmatrix}
                    c_{1} \\
                    c_{2}
                \end{bmatrix} \Rightarrow 
                \begin{cases}
                    0\cdot c_{1} + 1\cdot c_{2} = c_{1} \\
                    1\cdot c_{1} + 0\cdot c_{2} = c_{2}
                \end{cases}
            \end{equation*}
            \begin{equation*}
                c_{1} = c_{2} \equiv k_{1}
            \end{equation*}
        
        Então de modo geral o autoestado relacionado a $b_{1}$ pode ser escrito como sendo:
            \begin{equation*}
                \ket{b_{1}} = k_{1}
                \begin{bmatrix}
                    1 \\ 1
                \end{bmatrix}
            \end{equation*}
        
        Afim de fazermos uma base ortonormal com os autoestados, podemos impor a normalização de modo que:
            \begin{equation*}
                \braket{b_{1}}{b_{1}} = 1 \Leftrightarrow k_{1}^{\ast}
                \begin{bmatrix}
                    1 & 1
                \end{bmatrix} k_{1}
                \begin{bmatrix}
                    1 \\ 1
                \end{bmatrix} = 1 \Rightarrow 2\abs{k_{1}}^2 = 1 \Rightarrow \abs{k_{1}} = \dfrac{1}{\sqrt{2}}
            \end{equation*} 
        
        Sendo assim, o autoestado $\ket{b_{1}}$ é dado simplesmente por:
            \begin{answer*}
                \ket{b_{1}} = \dfrac{1}{\sqrt{2}}
                \begin{bmatrix}
                    1 \\ 1
                \end{bmatrix}
            \end{answer*}
        
        Analogamente para $b_{2} = -1$, temos que:
            \begin{equation*}
                \hat{\mathcal{B}}\ket{b_{2}} = -1\ket{b_{2}} \Rightarrow 
                \begin{bmatrix}
                    0   &   1   \\
                    1   &   0
                \end{bmatrix}
                \begin{bmatrix}
                    c_{3} \\
                    c_{4}
                \end{bmatrix} = -1
                \begin{bmatrix}
                    c_{3} \\
                    c_{4}
                \end{bmatrix} \Rightarrow 
                \begin{cases}
                    0\cdot c_{3} + 1\cdot c_{4} = -c_{3} \\
                    1\cdot c_{3} + 0\cdot c_{4} = -c_{4}
                \end{cases}
            \end{equation*}
            \begin{equation*}
                c_{3} = -c_{4} \equiv k_{2}
            \end{equation*}
        
        Portanto:
            \begin{equation*}
                \ket{b_{2}} = k_{2}
                \begin{bmatrix*}[r]
                    1 \\ -1
                \end{bmatrix*}
            \end{equation*}
        
        Normalizando:
            \begin{equation*}
                \braket{b_{2}}{b_{2}} = 1 \Leftrightarrow k_{2}^{\ast}
                \begin{bmatrix}
                    1 & -1
                \end{bmatrix} k_{2}
                \begin{bmatrix*}[r]
                    1 \\ -1
                \end{bmatrix*} = 1 \Rightarrow 2\abs{k_{2}}^2 = 1 \Rightarrow \abs{k_{2}} = \dfrac{1}{\sqrt{2}}
            \end{equation*}
        
        Logo, o autoestado $\ket{b_{2}}$ é dado por:
            \begin{answer*}
                \ket{b_{2}} = \dfrac{1}{\sqrt{2}}
                \begin{bmatrix*}[r]
                    1 \\ -1
                \end{bmatrix*}
            \end{answer*}
        
        \begin{note}{}
            Podemos escrever tanto o operador $\hat{\mathcal{B}}$ quanto os autoestados $\ket{b_{1}}$ e $\ket{b_{2}}$ na forma de combinações lineares de \textit{bra's} e \textit{ket's}, tal que:
                \begin{equation*}
                    \hat{\mathcal{B}} = \ket{2}\bra{1} + \ket{1}\bra{2}
                \end{equation*}
                \begin{equation*}
                    \ket{b_{1}} = \dfrac{1}{\sqrt{2}}(\ket{1} + \ket{2}) \qquad \& \qquad
                    \ket{b_{2}} = \dfrac{1}{\sqrt{2}}(\ket{1} - \ket{2})
                \end{equation*}
        \end{note}
        
    \end{example}
    
    Dados os exemplos acima, podemos fazer uma análise simples sobre o processo de medida em mecânica quântica. Olhando para o Exemplo 5, temos os autoestados $\ket{1}$ e $\ket{2}$ cujos autovalores relacionados são respectivamente $a_{1}$ e $a_{2}$, já no Exemplo 6 temos os autoestados $\ket{b_{1}}$ e $\ket{b_{2}}$ compostos por uma combinação linear entre $\ket{1}$ e $\ket{2}$, e possuem autovalores $b_{1}$ e $b_{2}$.
    
    Vamos então supor que exista um aparelho que meça o operador $\hat{\mathcal{B}}$, de modo que ele nos retorna um dos autovalores do operador:
        \begin{figure}[H]
            \centering
            \begin{tikzpicture}
                \draw[->](0,0) -- (1.5,0) node[midway, above]{$\ket{\psi}=?$};
                \draw[fill = myLColor] (1.5,-0.5) rectangle (2.5,0.5) node[midway,myLLColor]{$\hat{\mathcal{B}}$};
                \draw[->] (2.5,0.25) -- (4,0.25) node[midway, above]{$b_{1}=+1$};
                \draw[->] (2.5,-0.25) -- (4,-0.25) node[midway, below]{$b_{2}=-1$};
            \end{tikzpicture}
            \caption{Representação esquemática de como um operador atua em sob um estado quântico.}
            \label{Operators measurement}
        \end{figure}
    
    Então se obtivermos $b_{1}=1$, teremos associado o autoestado $\ket{b_{1}}$ que é composto pelos vetores $\ket{1}$ e $\ket{2}$, e isto nos indica que não podemos medir os dois operadores simultaneamente, pois quando fazemos a medida de $\hat{\mathcal{B}}$, não conseguimos determinar qual dos autoestados de $\hat{\mathcal{A}}$ teremos, isto pelo simples motivo de que $\ket{b_{1}}$ depende de $\ket{1}$ \textit{\textbf{e}} $\ket{2}$. 
    
    Da mesma forma, podemos manipular $\ket{b_{1}}$ e $\ket{b_{2}}$ para isolarmos $\ket{1}$ e $\ket{2}$, tal que estes dependerão de $\ket{b_{1}}$ e $\ket{b_{2}}$, o que gera uma espécie de looping quando tentamos medir os dois operadores ao mesmo tempo.
    
    Sabemos que com base em um vetor de estado $\ket{\psi}$, podemos calcular os valores esperados dos operadores $\hat{\mathcal{A}}$ e $\hat{\mathcal{B}}$ como sendo:
        \begin{equation*}
            \expval*{\hat{\mathcal{A}}} = \bra{\psi}\hat{\mathcal{A}}\ket{\psi} \qquad \& \qquad
            \expval*{\hat{\mathcal{B}}} = \bra{\psi}\hat{\mathcal{B}}\ket{\psi}
        \end{equation*}
    
    Com isso, medir as variâncias $\sigma_{\hat{\mathcal{A}}}^2$ e $\sigma_{\hat{\mathcal{B}}}^{2}$ fica inteiramente determinada, já que:
        \begin{equation*}
            \sigma_{\hat{\mathcal{A}}}^2 = \expval*{(\hat{\mathcal{A}} - \expval*{\hat{\mathcal{A}}})^2} \qquad \& \qquad 
            \sigma_{\hat{\mathcal{B}}}^2 = \expval*{(\hat{\mathcal{B}} - \expval*{\hat{\mathcal{B}}})^2}
        \end{equation*}
    
    As variâncias vão nos dizes o quão boas são as medidas, se comparadas com o valor esperado, dessa forma, calcular o produto entre as variâncias nos fornecerá informações extras sobre as medidas. Antes disso, temos:
        \begin{align*}
            \sigma_{\hat{\mathcal{A}}}^2 &\eq \expval*{(\hat{\mathcal{A}} - \expval*{\hat{\mathcal{A}}})^2} \\
            &\eq \bra{\psi}(\hat{\mathcal{A}} - \expval*{\hat{\mathcal{A}}})^2\ket{\psi} \\
            &\eq \bra{\psi}(\hat{\mathcal{A}} - \expval*{\hat{\mathcal{A}}})^{\dagger}(\hat{\mathcal{A}} - \expval*{\hat{\mathcal{A}}})\ket{\psi} \\
            &\eq \braket{f}{f}
        \end{align*}
    onde $\ket{f} := (\hat{\mathcal{A}} - \expval*{\hat{\mathcal{A}}})\ket{\psi}$. Analogamente para $\sigma_{\hat{\mathcal{B}}}^2$:
        \begin{align*}
            \sigma_{\hat{\mathcal{B}}}^2 &\eq \expval*{(\hat{\mathcal{B}} - \expval*{\hat{\mathcal{B}}})^2} \\
            &\eq \bra{\psi}(\hat{\mathcal{B}} - \expval*{\hat{\mathcal{B}}})^2\ket{\psi} \\
            &\eq \bra{\psi}(\hat{\mathcal{B}} - \expval*{\hat{\mathcal{B}}})^{\dagger}(\hat{\mathcal{B}} - \expval*{\hat{\mathcal{B}}})\ket{\psi} \\
            &\eq \braket{g}{g}
        \end{align*}
    onde $\ket{g} := (\hat{\mathcal{B}} - \expval*{\hat{\mathcal{B}}})\ket{\psi}$. Então o produto entre as variâncias pode ser escrito simplesmente por:
        \begin{equation*}
            \sigma_{\hat{\mathcal{A}}}^2\sigma_{\hat{\mathcal{B}}}^2 = \braket{f}{f}\braket{g}{g}
        \end{equation*}
    
    Porém, se não soubermos o vetor de estado $\ket{\psi}$, não conseguimos determinar as variâncias, tampouco o produto entre elas, dessa forma, utilizaremos a \textit{desigualdade de Cauchy-Schwarz}, dada por:
        \begin{answer}\label{desigualdade de CS}
            \braket{\alpha}{\alpha}\braket{\beta}{\beta} \geqslant \abs{\braket{\alpha}{\beta}}^2
        \end{answer}
        \begin{proof}
            Dados dois vetores quaisquer $\ket{\alpha}$ e $\ket{\beta}$, podemos construir um vetor $\ket{\gamma}$ utilizando Gram-Schmidt tal que:
                \begin{equation*}
                    \ket{\gamma} = \ket{\beta} - \dfrac{\braket{\alpha}{\beta}}{\braket{\alpha}{\alpha}}\ket{\alpha}
                \end{equation*}
            
            De modo que impomos $\braket{\alpha}{\alpha}>0$. Calculando então o produto escalar $\braket{\beta}{\gamma}$, temos:
                \begin{align*}
                    \braket{\beta}{\gamma} &\eq 
                    \bra{\beta}\left(
                        \ket{\beta} - \dfrac{\braket{\alpha}{\beta}}{\braket{\alpha}{\alpha}}\ket{\alpha}
                    \right) \\
                    &\eq \braket{\beta}{\beta} - \dfrac{\braket{\alpha}{\beta}}{\braket{\alpha}{\alpha}}\braket{\beta}{\alpha} \\
                    &\eq \braket{\beta}{\beta} - \dfrac{\braket{\alpha}{\beta}}{\braket{\alpha}{\alpha}}(\braket{\alpha}{\beta})^{\dagger} \\
                    &\eq \braket{\beta}{\beta} - \dfrac{\abs{\braket{\alpha}{\beta}}^2}{\braket{\alpha}{\alpha}}
                \end{align*}
            
            Sendo então $\braket{\beta}{\gamma}\geqslant0$, podemos multiplicar ambos os lados por $\braket{\alpha}{\alpha}$ e concluir que:
                \begin{equation*}
                    \braket{\beta}{\gamma} \braket{\alpha}{\alpha} \geqslant 0 
                \end{equation*}
            
            Portanto:
                \begin{equation*}
                    \braket{\alpha}{\alpha}\braket{\beta}{\beta} \geqslant\abs{\braket{\alpha}{\beta}}^2
                \end{equation*}
        \end{proof}
    
    Então aplicando (\ref{desigualdade de CS}):
        \begin{align*}
            \sigma_{\hat{\mathcal{A}}}^2\sigma_{\hat{\mathcal{B}}}^2 = 
            \braket{f}{f}\braket{g}{g} &\geqslant \abs{\braket{f}{g}}^2 \\
            &\eq \abs{\bra{\psi}(\hat{\mathcal{A}} - \expval*{\hat{\mathcal{A}}})^{\dagger}(\hat{\mathcal{B}} - \expval*{\hat{\mathcal{B}}})\ket{\psi}}^2 \\
            &\eq \abs{\bra{\psi}(\hat{\mathcal{A}} - \expval*{\hat{\mathcal{A}}})(\hat{\mathcal{B}} - \expval*{\hat{\mathcal{B}}})\ket{\psi}}^2 \\ 
            &\eq \abs{\bra{\psi} \hat{\mathcal{A}}\hat{\mathcal{B}} - \expval*{\hat{\mathcal{A}}}\hat{\mathcal{B}} - \expval*{\hat{\mathcal{B}}}\hat{\mathcal{A}} + \expval*{\hat{\mathcal{A}}}\expval*{\hat{\mathcal{B}}} \ket{\psi}}^2 \\
            &\eq \Big|
            \bra{\psi}\hat{\mathcal{A}}\hat{\mathcal{B}}\ket{\psi} - 
            \expval*{\hat{\mathcal{A}}}\bra{\psi}\hat{\mathcal{B}}\ket{\psi} - 
            \expval*{\hat{\mathcal{B}}}\bra{\psi}\hat{\mathcal{A}}\ket{\psi} + \\
            &\noeq \expval*{\hat{\mathcal{A}}}\expval*{\hat{\mathcal{B}}}\braket{\psi}{\psi}
            \Big|^2 \\
            &\eq \abs{
            \bra{\psi}\hat{\mathcal{A}}\hat{\mathcal{B}} - \expval*{\hat{\mathcal{A}}}\expval*{\hat{\mathcal{B}}}\ket{\psi} - 
            \expval*{\hat{\mathcal{A}}}\expval*{\hat{\mathcal{B}}} + 
            \expval*{\hat{\mathcal{B}}}\expval*{\hat{\mathcal{A}}}
            }^2 \\
            &\eq \abs{\bra{\psi}\hat{\mathcal{A}}\hat{\mathcal{B}} - \expval*{\hat{\mathcal{A}}}\expval*{\hat{\mathcal{B}}}\ket{\psi}}^2
        \end{align*}
    
    Seja então $\hat{\mathcal{Z}} := \hat{\mathcal{A}}\hat{\mathcal{B}} - \expval*{\hat{\mathcal{A}}}\expval*{\hat{\mathcal{B}}}$ um novo operador, tal que ele não é necessariamente hermitiano, ou seja, $\expval*{\hat{\mathcal{Z}}}\in\mathbb{C}$:
        \begin{equation*}
            \expval*{\hat{\mathcal{Z}}} = \mathfrak{Re}[\expval*{\hat{\mathcal{Z}}}] + i\mathfrak{Im}[\expval*{\hat{\mathcal{Z}}}]
        \end{equation*}
    
    Temos portanto que:
        \begin{equation*}
             \sigma_{\hat{\mathcal{A}}}^2\sigma_{\hat{\mathcal{B}}}^2 \geqslant \abs{\bra{\psi}\hat{\mathcal{Z}}\ket{\psi}}^2 = \abs{\expval*{\hat{\mathcal{Z}}}}^2 \geqslant \abs{\mathfrak{Im}[\expval*{\hat{\mathcal{Z}}}]}^2
        \end{equation*}
    
    Lembrando que a parte imaginária de um número complexo $z$ pode ser escrita como sendo:
        \begin{equation*}
            \mathfrak{Im}[z] = \dfrac{1}{2i}(z - z^{\ast})
        \end{equation*}
    
    Portanto:
        \begin{align*}
            \sigma_{\hat{\mathcal{A}}}^2\sigma_{\hat{\mathcal{B}}}^2 &\geqs
            \abs{\dfrac{1}{2i}(\expval*{\hat{\mathcal{Z}}} - \expval*{\hat{\mathcal{Z}}^{\ast}})}^2 \\
            &\eq \abs{\dfrac{1}{2i}(\bra{\psi}\hat{\mathcal{A}}\hat{\mathcal{B}} - \expval*{\hat{\mathcal{A}}}\expval*{\hat{\mathcal{B}}} - \hat{\mathcal{B}}^{\dagger}\hat{\mathcal{A}}^{\dagger} +\expval*{\hat{\mathcal{A}}}\expval*{\hat{\mathcal{B}}}\ket{\psi})}^2 \\
            &\eq \abs{\dfrac{1}{2i}\bra{\psi}\hat{\mathcal{A}}\hat{\mathcal{B}} - \hat{\mathcal{B}}\hat{\mathcal{A}} \ket{\psi}}^2 \\
            &\eq \abs{
                \dfrac{1}{2i}\bra{\psi}[\hat{\mathcal{A}},\hat{\mathcal{B}}]\ket{\psi}
            }^2 \\
            &\eq \dfrac{1}{2^2}\abs{\expval*{[\hat{\mathcal{A}},\hat{\mathcal{B}}]}}^2
        \end{align*}
    
    Logo, tirando a raiz quadrada em ambos os lados, obtemos o \textit{princípio da incerteza entre dois operadores}:
        \begin{answer}\label{principio da incerteza operadores}
                \sigma_{\hat{\mathcal{A}}}\sigma_{\hat{\mathcal{B}}} \geqslant \dfrac{1}{2}\abs{\expval*{[\hat{\mathcal{A}},\hat{\mathcal{B}}]}}
        \end{answer}
    
    Esse resultado é importante devido ao fato de que caso os operadores comutem entre si, existe a possibilidade de medirmos simultaneamente os observáveis relacionados. No entanto, se não comutarem, eles nunca poderão ser medidos de forma simultânea, mesmo que o aparato de medida seja o mais tecnológico de todos. Essa restrição é intrínseca à relação de comutação entre os operadores e nada pode mudar isso.

    Para a situação de precisão infinita temos um importante teorema a destacar:
    \begin{theorem}{Base entre operadores}{}
        Dados dois operadores $\mathcal{\hat{A}}$ e $\mathcal{\hat{B}}$ que comutam entre si:
        \begin{equation*}
            [\mathcal{\hat{A}},\mathcal{\hat{B}}] = 0
        \end{equation*}
        Sempre existe uma base de autoestados $\{\ket{\psi}\}$ comum tanto para $\mathcal{\hat{A}}$ quanto para $\mathcal{\hat{B}}$.
    \end{theorem}
    \begin{proof}
        Seja $\{\ket{\psi}\}$ uma base de autoestados de $\mathcal{\hat{A}}$ sem autovalores degenerados, isto é,
        \begin{equation*}
            \mathcal{\hat{A}}\ket{\psi} = a\ket{\psi}
        \end{equation*}
        Como $\mathcal{\hat{A}}$ e $\mathcal{\hat{B}}$ comutam, podemos escrever
        \begin{equation*}
            \mathcal{\hat{A}}\mathcal{\hat{B}} = \mathcal{\hat{B}}\mathcal{\hat{A}}
        \end{equation*}
        Podemos escrever portanto
        \begin{equation*}
            \mathcal{\hat{A}}\mathcal{\hat{B}}\ket{\psi} = \mathcal{\hat{B}}\mathcal{\hat{A}}\ket{\psi} = \mathcal{\hat{B}}\left(a\ket{\psi}\right) = a\mathcal{\hat{B}}\ket{\psi}
        \end{equation*}
        \begin{equation*}
            \mathcal{\hat{A}}(\mathcal{\hat{B}}\ket{\psi}) = a(\mathcal{\hat{B}}\ket{\psi})
        \end{equation*}
        ou seja, $\mathcal{\hat{B}}\ket{\psi}$ é um autoestado de $\mathcal{\hat{A}}$ associado ao mesmo autovalor $a$. Isso significa que, dado que $a$ não é degenerado, $\mathcal{\hat{B}}\ket{\psi}$ deve ser o mesmo autoestado que $\ket{\psi}$, o que só pode ocorrer se um for o outro a menos de uma constante $b$:
        \begin{equation*}
            \mathcal{\hat{B}}\ket{\psi} = b\ket{\psi}
        \end{equation*}
        o que constitui uma equação de autovalores, concluindo nossa demonstração.
    \end{proof}

        \section{Operadores de posição e momento}
            Levando em conta que o conceito de posição e momento são quantidades claramente mensuráveis, de modo que se relacionam diretamente a observáveis, podemos atribuir a eles \textit{operadores hermitianos}, tal que cada um terá seu próprio conjunto de autoestados e autovalores relacionados e, pelo Corolário \ref{cor: base de autoestados}, esses autoestados irão constituir uma base ortonormal.

            \subsection{O operador de posição}
                    Denotemos o operador de posição por $\hat{x}$, em que ele é hermitiano e seus autoestados e autovalores satisfazem a equação:
        \begin{answer}\label{eq: autoestados e autovalores de x}
            \hat{x}\ket{x} = x\ket{x}
        \end{answer}
    
    É importante salientar que a base de autoestados é contínua, ou seja, cada elemento desta é contínuo, de modo que se usarmos a definição de posição da mecânica clássica, podemos nos locomover uma distância tão pequena quanto se queira que ainda estaremos no mesmo espaço, portanto as equações passarão do discreto para o contínuo:
        \begin{align*}
            \ket{\psi} = \sum_{i}\psi_{i}\ket{a_{i}} &\rightarrow \ket{\psi} = \int \psi(x')\ket{x'}\dd{x'} \\
            \psi_{i} = \braket{a_{i}}{\psi} &\rightarrow \psi(x) = \braket{x}{\psi} \\ \\
            \braket{a_{i}}{a_{j}} = \delta_{ij} &\rightarrow \braket{x'}{x''} = \delta(x'-x'')
        \end{align*}
    
    Podemos demonstrar a segunda equação de forma simples:
        \begin{align*}
            \braket{x}{\psi} &\eq \bra{x}\int\psi(x')\ket{x'}\dd{x'} \\
            &\eq \int\psi(x')\braket{x}{x'}\dd{x'} \\
            &\eq \int\psi(x')\delta(x-x')\dd{x'} = \psi(x)
        \end{align*}

    Mesmo sendo uma demonstração simples, note a relevância deste resultado. Essencialmente o que estamos dizendo é que a função de onda $\psi(x)$ representa a projeção de um estado $\ket{\psi}$ qualquer no estado $\ket{x}$. Isto nos permite tratar muitas situações, como veremos mais adiante, com mais simplicidade e clareza, tendo em mente que nem sempre é fácil lidar com as funções de onda de forma direta.
    
    Além disto, conseguimos tirar como consequência que $\abs{\psi(x)}^2\dd{x}$ se relaciona diretamente com a probabilidade de encontrar o estado $\ket{\psi}$ na posição $x$. Isso devido ao fato de que se o vetor de estado estiver normalizado:
        \begin{align*}
            \braket{\psi}{\psi} = 1 &\eq \int\psi^{\ast}(x')\bra{x'}\dd{x'} \int\psi(x'')\ket{x''}\dd{x'}' \\
            &\eq \int\int\psi^{\ast}(x')\psi(x'')\braket{x'}{x''}\dd{x'}\dd{x'}' \\
            &\eq \int\int\psi^{\ast}(x')\psi(x'')\delta(x'-x'')\dd{x'}\dd{x'}' \\
            &\eq \int\psi^{\ast}(x')\psi(x')\dd{x'} \\
            &\eq \int\abs{\psi(x')}^2\dd{x'}
        \end{align*}
    
    Logo, se $\bra{\psi} \mapsto \bra{x}$, temos a probabilidade de encontrar o estado $\ket{\psi}$ na posição $x$.
    

            \subsection{O operador de momento}

            \subsection{O comutador posição--momento}

        \section{Mudança de base de representação}

        \section{Evolução temporal}

            \subsection{Descrição de Schrödinger}

            \subsection{Descrição de Heisenberg}

        \section{Relação de incerteza entre energia--tempo}

        \addcontentsline{toc}{section}{\itshape Leitura complementar}
        \subsubsection*{Leitura complementar}

        \addcontentsline{toc}{section}{\itshape Exercícios}
        \subsubsection*{Exercícios}

    \chapter{\hspace{0.8cm}Aplicações dos conceitos básicos}

        \section{Experimento de Stern--Gerlach}
    
            \subsection{Operadores de dipolo magnético}
    
        \section{Detecção de neutrinos}
    
            \subsection{Experimento de Reines--Cowan (1953)}
    
            \subsection{Experimento de Homestake (1953)}
    
            \subsection{Super--K e SNO (2001)}
    
            \subsection{Mundo simplificado: 2 tipos de neutrinos}
    
        \section{O oscilador harmônico}
    
            \subsection{Oscilador harmônico clássico}
    
            \subsection{Oscilador harmônico quântico}
    
            \subsection{A função de onda}
    
        \addcontentsline{toc}{section}{\itshape Leitura complementar}
        \subsubsection*{Leitura complementar}
    
        \addcontentsline{toc}{section}{\itshape Exercícios}
        \subsubsection*{Exercícios}
    
\part{Fundamentos Intermediários}
    
    \chapter{\hspace{0.8cm}Mecânica Quântica em 3D}
    
    \chapter{\hspace{0.8cm}Sistemas de partículas idênticas}
    
    \chapter{\hspace{0.8cm}Teoria de perturbação}

        \section{Independente do tempo}

        \section{Dependente do tempo}

\part{Fundamentos Avançados}

    \chapter{\hspace{0.8cm}Mecânica quântica relativística}
        Ao passarmos por diversas partes da mecânica quântica, podemos finalmente tratar de uma das partes mais interessantes a ser estudada e que gera uma diversidade de descrições mais avançadas, como por exemplo a Teoria Quântica de Campos. Nesta capítulo, trataremos inicialmente de uma básica revisão dos conceitos de relatividade \textit{restrita}, seguida de uma dedução minuciosa da famosa equação de equação de Klein\footnote{Oskar Benjamin Klein (1894--1977).}--Fock\footnote{Vladimir Aleksandrovich Fock (1898--1974).}--Gordon\footnote{Walter Gordon (1893--1939).} para bósons relativísticos e em seguida, um desenvolvimento apropriado para descrição de férmions relativísticos, através da também famosa equação de Dirac\footnote{Paul Adrien Maurice Dirac (1902--1984).}

A relatividade restrita, desenvolvida em seus primórdios por \textcite{Einstein1}, é parte de um dos tópicos mais importantes da física, juntamente com a relatividade geral. Em essência, a relatividade restrita descreve a estrutura básica do espaço--tempo (onde o tempo e o espaço são equivalentes), tal que a teoria não depende da escolha de nenhum referencial \textit{inercial}, em que este fato se resume na chamadas \textit{transformações de Lorentz}\footnote{Hendrik Antoon Lorentz (1853--1928).}, no entanto, a mecânica quântica \textit{não}-relativística possui limitações que fazem com que tais requisitos não sejam satisfeitos. Um exemplo simples disso é o fato de que o tempo e o espaço são assimétricos, o que podemos ver na própria equação de Schrödinger, em que a derivada no tempo é de primeira ordem e a derivada no espaço é de segunda ordem.

A conciliação total entre a mecânica quântica e a relatividade restrita é a chamada ``teoria quântica de de campos'', ou simplesmente TQC, no entanto, este tópico exige uma maturidade física e matemática muito superior ao que de fato abordamos até agora, o que nos leva a querer abordar o assunto de uma forma menos radical, ou seja, a partir de mudanças sutis encontrar novas equações onde o espaço e o tempo sejam tratados de forma mais equilibrada. A construção destas novas equações acabam tendo consequências cruciais na teoria, como por exemplo o caso de antipartículas e a relação do spin com o momento magnético.

No entanto, antes de realmente tratarmos da parte relativística, é conveniente tratar os sistemas estudados em sistemas de unidades diferentes do SI. Isto não é feito por uma necessidade, mas sim pela utilidade, pois na prática, o desenvolvimento teórico se simplifica de forma imensurável ao tratá-lo com um sistema de unidades diferente.

        \section{Sistemas de unidades naturais}
            Em suma, podemos descrever as unidades naturais como uma representação numérica de grandezas, ou seja, este sistema vai estabelecer que as constantes físicas universais serão unitárias, adimensionais e independentes da escala humana, o que não é verdade para os sistemas de unidades SI, CGS, etc. Vale salientar que estamos tratando de \textit{sistemas}, ou seja, não existe apenas um único sistema de unidades naturais, de modo que cada um é utilizado dependendo do contexto físico em que se está trabalhando para simplificar as contas e conseguir absorver melhor como a teoria está de fato sendo desenvolvida.

Escolheremos em particular um sistema introduzido por \textcite{Planck4} e modificadas por \textcite{Sorkin} através de uma normalização diferente, onde temos o que foi nomeada como ``sistema de unidades naturais de Planck racionalizada'' (ou para fins de praticidade, SN), caracterizada por:
    \begin{answer*}
        \hbar = c = k_{\text{B}} = \varepsilon_{0} = 4\pi G = 1
    \end{answer*}

Neste sistema , a única unidade que de fato precisamos e utilizamos é o elétron-volt (eV), de modo que grande parte das unidades, como por exemplo a temperatura, serão expressas em unidades de energia. Com base nisto, as coisas se simplificam bastante, alguns exemplos disso são: a equação de Schrödinger
    \begin{equation*}
        -\dfrac{\hbar^2}{2m}\nabla^2\psi + V(\psi) = i\hbar\pdv{\psi}{t} \Rightarrow 
        -\dfrac{1}{2m}\nabla^2\psi + V(\psi) = i\pdv{\psi}{t}, 
    \end{equation*}
a relação entre massa e energia
    \begin{equation*}
        E^2 = m^2 c^4 + p^2 c^2 \Rightarrow E^2 = m^2 + p^2
    \end{equation*}
e até mesmo o princípio da incerteza
    \begin{equation*}
        \Delta x \Delta p \geqslant \dfrac{\hbar}{2} \Rightarrow 
        \Delta x \Delta p \geqslant \dfrac{1}{2}
    \end{equation*}

Além de ver como funcionam as equações neste sistema, é conveniente e importante determinar como são as unidades nele, o que não é muito complicado de se ver. Através da relação entre massa e energia, tem-se que como $[E] = \text{eV}$:
    \begin{equation*}
        E^2 = m^2 + p^2 \Rightarrow [E]^2 = [m]^2 + [p]^2 \Rightarrow [E] = [m] = [p] = \text{eV}
    \end{equation*}

De forma similar, para determinar a unidade de comprimento, pode-se utilizar o princípio da incerteza:
    \begin{equation*}
        \Delta x \Delta p \geqslant \dfrac{1}{2} \Rightarrow [x] [p] = \text{qtd. adimensional} \Rightarrow [x] = [p]^{-1} = \text{eV}^{-1}
    \end{equation*}
e através do princípio da incerteza entre energia e tempo, tem-se
    \begin{equation*}
        \Delta E \Delta p \geqslant \dfrac{1}{2} \Rightarrow [E] [t] = \text{qtd. adimensional} \Rightarrow [t] = [E]^{-1} = \text{eV}^{-1}
    \end{equation*}
ou seja, tempo e espaço têm a mesma dimensão! Tendo em mente que nesse sistema de unidades $c = 1$ é uma quantidade adimensional, toda e qualquer velocidade também será adimensional e em particular teremos
    \begin{equation*}
        0 \leqslant v \leqslant 1
    \end{equation*}

Por fim, é conveniente também determinar a dimensão da carga elétrica, tendo em mente que em mecânica quântica esta quantidade se faz sempre presente, e para isso, partimos da constante de estrutura fina, que é uma quantidade adimensional:
    \begin{equation*}
        \alpha = \dfrac{e^2}{4 \pi \epsilon_{0} \hbar c} \Rightarrow \alpha = \dfrac{e^2}{4\pi} \Rightarrow e = \sqrt{4\pi\alpha} \approx 0.303\ (\text{qtd. adimensional})
    \end{equation*}
implicando que no SN a carga elétrica é também uma quantidade adimensional. Nota-se claramente que coisas são simplificadas teoricamente, pois as constantes que eventualmente aparecem são sempre iguais a 1 e adimensionais, tornando a matemática mais simples de ser trabalhada, no entanto, há a necessidade de saber converter as unidades do SN para o SI, por exemplo, o que não é difícil de ser feito.
    \begin{example}
        Se quisermos determinar o fator de conversão do tempo entre o SN e o SI, podemos fazer uso da constante de Planck reduzida:
            \begin{equation*}
                \hbar = 1 = 0.658 \cdot 10^{-15}\ \mathrm{eV\cdot s} \Rightarrow 
                1\ \mathrm{eV}^{-1} = 0.658\cdot 10^{-15}\ \text{s}
            \end{equation*}
    \end{example}

    \begin{example}
        De forma análoga ao exemplo anterior, podemos determinar o fator de conversão da posição entre os dois sistemas:
            \begin{equation*}
                \hbar c = 1 = 0.658\cdot 10^{-15} \cdot 2.998\cdot 10^{8}\ \mathrm{eV\cdot m} = 1.972 \cdot 10^{-7}\ \mathrm{eV\cdot m}
            \end{equation*}
        implicando portanto em
            \begin{equation*}
                1\ \text{eV}^{-1} = 1.972\cdot 10^{-7}\ \text{m}
            \end{equation*}
    \end{example}

    \begin{example}
        Um exemplo prático de como as contas se simplificam pode ser visto a partir da determinação do estado fundamental do átomo de hidrogênio, isto por que a expressão para determinar as energias fica bem mais simples
            \begin{equation*}
                E_{n} = -\qty(\dfrac{\mathcal{Z}e^2}{4\pi\varepsilon_{0}})^2\dfrac{m_{e}}{2\hbar^2}\dfrac{1}{n^2} \Rightarrow E_{n} = -\qty(\dfrac{\mathcal{Z}e^2}{4\pi})^2\dfrac{m_{e}}{2n^2} \overset{\mathcal{Z}=1}{=} -\dfrac{e^4 m_{e}}{32\pi^2n^2}
            \end{equation*}

        Segue que para o estado fundamental ($n=1$), utilizamos a massa do elétron (0.511 MeV) e o valor da carga elétrica (0.303) para obter
            \begin{equation*}
                E_{1} = -\dfrac{(0.303)^4 \cdot 0.511 \cdot 10^{6}}{32 \pi^2} = -13.6\ \text{eV}
            \end{equation*}
    \end{example}

    A partir deste ponto, usaremos em sua grande maioria este sistema de unidades naturais afim de simplificar as equações e suas interpretações.

        \section{Recordando relatividade restrita}
            A relatividade restrita é um conceito da física que pode ser compactada em dois postulados principais
    \begin{myitemize}
        \item As leis da física são invariantes em relação a mudanças entre referenciais inerciais;
        \item A velocidade da luz independe da velocidade da fonte de luz.
    \end{myitemize}

O conteúdo destes postulados gera algumas consequências fundamentais para toda teoria, algumas delas são: o espaço e o tempo não são absolutos, o espaço pode ser contraído e o tempo dilatado, a simultaneidade de eventos é diferente dependendo do referencial, etc. E para entender melhor sobre esses postulados e suas consequências, é necessário relembrar sobre as transformações de Lorentz.

            \subsection{Transformações de Lorentz}
                Consideremos dois referenciais $S$ e $S'$ tais que $S'$ se move em relação à $S$ com velocidade $v = v_{z} = \beta$, onde $\beta \coloneqq \dfrac{v}{c}$, mas como $c=1$ no sistema de unidades naturais, temos apenas $\beta = v$. Lembrando que o fator de Lorentz é definido por
    \begin{equation*}
        \gamma = \dfrac{1}{\sqrt{1 - \beta^2}}
    \end{equation*}
temos que as transformações de Lorentz para este caso são 
    \begin{align*}
        t' &= \gamma(t - \beta z) &
        x' &= x &
        y' &= y &
        z' &= \gamma(z - \beta t)
    \end{align*}

Com estas equações, podemos construir uma matriz que será responsável por armazenar todas as quantidades relacionadas às contrações espaciais e dilatações temporais:
    \begin{equation}\label{eq: transformations}
        \begin{pmatrix}
            t' \\ x' \\ y' \\ z'
        \end{pmatrix} = 
        \begin{pmatrix}
            \gamma & 0 & 0 & -\beta \gamma \\
            0 & 1 & 0 & 0 \\
            0 & 0 & 1 & 0 \\
            -\beta \gamma & 0 & 0 & \gamma
        \end{pmatrix} 
        \begin{pmatrix}
            t \\ x \\ y \\ z
        \end{pmatrix}
    \end{equation}
de modo que a partir desta construção, definimos um quadrivetor contravariante na forma
    \begin{equation*}
        x^{\mu} \coloneqq(x^{0}, x^{1}, x^{2}, x^{3}) \equiv  (t, x, y, z) = (x^{0}, \vb{r})
    \end{equation*}
e a matriz na verdade é a forma matricial do tensor de Lorentz, o que escrevemos nesta situação por
    \begin{equation*}
        (\tensor{\Lambda}{^{\mu}_{\nu}}) \coloneqq 
        \begin{pmatrix}
            \gamma & 0 & 0 & -\beta \gamma \\
            0 & 1 & 0 & 0 \\
            0 & 0 & 1 & 0 \\
            -\beta \gamma & 0 & 0 & \gamma
        \end{pmatrix} 
    \end{equation*}

Com estas duas definições, podemos reescrever a equação \eqref{eq: transformations} em termos do quadrivetor e do tensor de Lorentz:
    \begin{equation*}
        x^{\prime \mu} = \sum_{\nu=0}^{3} \tensor{\Lambda}{^{\mu}_{\nu}}x^{\nu}
    \end{equation*}
o que ainda pode ser simplificado ao utilizar-se a notação de Einstein, que nos diz que se houver os mesmos índices covariantes e contravariantes, a somatória é subentendida na conta, ou seja
    \begin{equation*}
        x^{\prime \mu} = \sum_{\nu=0}^{3} \tensor{\Lambda}{^{\mu}_{\nu}}x^{\nu} = \tensor{\Lambda}{^{\mu}_{\nu}}x^{\nu}
    \end{equation*}

Como nas transformações de Galileu, que nos diz que valores espaciais não mudam de tamanho com mudanças de referencial, os quadrivetores não mudam de tamanho por transformações de Lorentz. Definimos o produto escalar entre dois quadrivetores $x^{\mu}$ e $y^{\mu}$ por 
    \begin{equation}\label{eq: produto escalar}
        A \coloneqq x^{0}y^{0} - \vb{x}\cdot\vb{y} = x^{0}y^{0} - x^{1}y^{1} - x^{2}y^{2} - x^{3}y^{3}
    \end{equation}

Desta forma, o ``tamanho'' de um quadrivetor é determinado simplesmente por
    \begin{equation*}
        A = x^{0}x^{0} - \vb{x}\cdot\vb{x}
    \end{equation*}

Com estas relações em mente, podemos mudar de referencial através das transformações de Lorentz e verificar se os quadrivetores são mesmo invariantes por transformações de Lorentz.
    \begin{align*}
        A' &\eq x^{\prime 0}x^{\prime 0} - \vb{x}^{\prime}\cdot\vb{x}^{\prime} \\
        &\eq \gamma(t - \beta z)\gamma(t - \beta z) - x^2 - y^2 - \gamma(z - \beta t)\gamma(z - \beta t) \\
        &\eq \gamma^2(t^2 - 2\beta t z + \beta^2z^2) - x^2 - y^2 - \gamma^2(z^2 - 2\beta t z + \beta^2 t^2) \\
        &\eq \dfrac{1}{1 - \beta^2}(t^2 - 2\beta tz + \beta^2 z^2 - z^2 + 2\beta t z - \beta^2 t^2) - x^2 - y^2 \\
        &\eq \dfrac{1}{1 - \beta^2}(t^2 - \beta^2 t^2 - z^2 + \beta^2 z^2) - x^2 - y^2 \\
        &\eq \dfrac{1}{1 - \beta^2}\Big[(1 - \beta^2)t^2 - (1 - \beta^2)z^2\Big] - x^2 - y^2 \\
        &\eq t^2 - x^2 - y^2 - z^2 \\
        &\eq x^{0}x^{0} - \vb{x}\cdot\vb{x} = A
    \end{align*}
concluindo o que queríamos mostrar. Agora analisando a expressão para o produto escalar, somos influenciados a olhar para o sinal negativo presente em \eqref{eq: produto escalar} e pensar que de alguma forma, podemos definir outro quadrivetor para mudar este sinal, o que será chamado de quadrivetor covariante, definido por
    \begin{equation*}
        x_{\mu} \coloneqq (x_{0}, -x_{1}, -x_{2}, -x_{3}) \equiv (t, -x, -y, -z) = (x_{0}, -\vb{r})
    \end{equation*}
de tal forma que o produto escalar entre 2 quadrivetores possa ser escrito como
    \begin{align*}
        x^{\mu}y_{\mu} &\eq \sum_{\mu=0}^{3} x^{\mu}y_{\mu} = x^{0}y_{0} + x^{1}y_{1} + x^{2}y_{2} + x^{3}y_{3} \\
        &\eq x^{0}y^{0} - x^{1}y^{1} - x^{2}y^{2} - x^{3}y^{3}
    \end{align*}

Tendo então as formas covariante e contravariante em mãos, precisamos de uma forma de escrever uma em relação a outra, e para isso utilizamos o tensor métrico de Minkowski\footnote{Hermann Minkowski (1864--1909).} dado por
    \begin{equation}\label{eq: Minkowski metric}
        (g_{\mu\nu}) \coloneqq 
        \begin{pmatrix}
            1 & 0 & 0 & 0 \\
            0 &-1 & 0 & 0 \\
            0 & 0 &-1 & 0 \\
            0 & 0 & 0 &-1
        \end{pmatrix} = (g^{\mu\nu})
    \end{equation}

\begin{note}{}
    Em muitos livros, infelizmente, utiliza-se um outra convenção para o tensor métrico, que ao invés de possuir $\diag{g_{\mu\nu}} = (+ - - -)$, inverte os sinais para ficar com $\diag{g_{\mu\nu}} = (- + + +)$, porém esta convenção tende a ser problemática em relação a algumas interpretações, o que não é muito conveniente.
\end{note}

Utilizando então este tensor métrico, podemos passar da notação covariante para a contravariante (e vice-versa) através das relações
    \begin{equation*}
        x_{\mu} = g_{\mu\nu}x^{\nu} \qquad \& \qquad 
        x^{\mu} = g^{\mu\nu}x_{\nu}
    \end{equation*}
e com isso, podemos escrever o produto escalar entre 2 quadrivetores por
    \begin{equation*}
        A = x^{\mu}g_{\mu\nu}y^{\nu} = x_{\mu}g^{\mu\nu}y_{\nu}
    \end{equation*}

Uma outra propriedade importante relativa ao tensor métrico é o fato de que como $g_{\mu\nu} = g^{\mu\nu}$, temos que
    \begin{equation*}
        g_{\mu\nu}g^{\nu\rho} = \boldone = \tensor{\delta}{_{\mu}^{\rho}}
    \end{equation*}

Através de tais conversões e propriedades, somos capazes agora de encontrar relações entre as velocidades. Consideremos uma partícula de massa $m$ se movendo com velocidade $v$ em um referencial, de tal forma que as coordenadas desta partícula neste referencial assumem a forma
    \begin{equation*}
        \dd{\vb{r}} = v\dd{t},
    \end{equation*}
tal que o quadrivetor $(\dd{t}, \dd{x}, \dd{y}, \dd{z})$ se transformam com as transformações de Lorentz e portanto seu comprimento é invariante sob elas, isto é
    \begin{equation*}
        \dd{x^{\mu}} = (\dd{t},\dd{\vb{r}})
    \end{equation*}
é invariante e a quantidade
    \begin{align*}
        \dd{\tau^2} &\eq \dd{x^{\mu}}\dd{x_{\mu}} = \dd{t}^2 - \dd{\vb{r}}^2 \\
        &\eq \dd{t}^2 - v^2 \dd{t}^2 \\
        &\eq (1 - v^2)\dd{t^2} \\
        &\eq \dfrac{1}{\gamma^2}\dd{t}^2
    \end{align*}
é também invariante. O tempo $\tau$, definido na forma $\dd{\tau}^2 = \dd{t}^2(1-v^2)$ é denominado de ``tempo prórpio''. Então se $\dd{x}^{\mu}$ e $\dd{\tau}$ são invariantes, o quadrivetor construído por
    \begin{equation*}
        u^{\mu} \coloneqq \dv{x^{\mu}}{\tau} = \qty(\gamma \dv{t}{t}, \gamma \dv{x}{t}, \gamma \dv{y}{t}, \gamma\dv{z}{t})
    \end{equation*}
é também invariante e é denominado ``quadrivetor velocidade'', que comumente é escrito no forma
    \begin{equation*}
        u^{\mu} = (\gamma, \gamma\vb{v})
    \end{equation*}

Com esta forma, temos
    \begin{equation*}
        u^{\mu}u_{\mu} = \gamma^2 - (\gamma \vb{v})\cdot (\gamma \vb{v}) = \gamma^2 - \gamma^2 v^2 = \gamma^2(1 - v^2) = \dfrac{\gamma^2}{\gamma^2} = 1
    \end{equation*}
o que fora do sistema de unidades naturais seria $u^{\mu}u_{\mu} = c^2$. Dado então que $u^{\mu}$ é o quadrivetor velocidade e é invariante, o quadrivetor $mu^{\mu}$ também será invariante, pois a massa se mantém invariante, de modo que defini-se o quadrivetor momento por
    \begin{equation*}
        p^{\mu} \coloneqq (\gamma m, \gamma m\vb{v}) \equiv (E, \vb{p})
    \end{equation*}
quer é de fato invariante por construção, além de que
    \begin{equation*}
        p^{\mu}p_{\mu} = (\gamma m)(\gamma m) - (\gamma m \vb{v})\cdot(\gamma m \vb{v}) = \gamma^2 m^2 - \gamma^2 m^2 v^2 = m^2\gamma^2(1-v^2) = m^2
    \end{equation*}

Este último desenvolvimento nos dá uma importante relação a ser salientada e enfatizada, que é chamada de relação de dispersão entre energia e momento
    \begin{answer}\label{eq: dispersion relation}
        p^{\mu}p_{\mu} = E^2 - p^2 = m^2 
    \end{answer}

Além dos quadrivetores posição, velocidade, momento, etc, podemos extrapolar a notação quadrivetorial para derivadas parciais, isto é, ao escrevermos
    \begin{equation*}
        \partial^{\mu} \coloneqq \pdv{}{x_{\mu}} \equiv \qty(\pdv{}{t}, -\vb{\nabla}) \qquad \& \qquad 
        \partial_{\mu} \coloneqq \pdv{}{x^{\mu}} \equiv \qty(\pdv{}{t}, \vb{\nabla})
    \end{equation*}
de modo que $\partial^{\mu}$ se comporta como um quadrivetor contravariante, mesmo que utilizemos em $\pdv{}{x_{\mu}}$ uma notação covariante no ``denominador''. De forma similar o contrário vai ocorrer com $\partial_{\mu}$, que se comporta como um quadrivetor covariante, mas se escreve por $\pdv{}{x^{\mu}}$. Com essas notações, temos
    \begin{align*}
        \partial_{\mu}\partial^{\mu} &\eq \pdv{}{x^{0}}\pdv{}{x_{0}} - \pdv{}{x^{1}}\pdv{}{x_{1}} - \pdv{}{x^{2}}\pdv{}{x_{2}} - \pdv{}{x^{3}}\pdv{}{x_{3}} \\
        &\eq \pdv[2]{}{t} - \vb{\nabla}^2
    \end{align*}

O operador $\partial_{\mu}\partial^{\mu}$ é chamado ``operador D'alambertiano''.
    \begin{note}{}
        Em muitos livros, podem-se encontrar diferentes notações para este operador, uma das mais comuns é representá-lo por $\square$ ou $\square^2$, que possui um sentido por trás, que é o fato de que por estarmos numa representação quadridimensional, o quadrado seria um análogo ao $\nabla^2$, que trata das derivadas em 3 dimensões, ou seja, como o quadrado tem 4 lados, seria uma boa ideia representar uma derivada em 4 dimensões com ele, já que o laplaciano é em 3 dimensões e é representado por um triângulo. Da mesma forma, representar $\square$ ou $\square^2$ varia de gosto pra gosto, pois a primeira forma é simples e contém as informações necessárias, já a segunda forma, faz questão de enfatizar que as derivadas dentro do D'alembertiano são de segunda ordem, e por isso são elevadas ao quadrado. Uma última maneira de representar este operador, utilizada bastante em TQC, é a forma $\partial^2$, que leva em conta simplesmente o fato de que $\partial_{\mu}\partial^{\mu}$ é como um produto de dois objetos ``iguais''.
    \end{note}

        \section{A equação de Klein--Fock--Gordon}
            Ao tratarmos de mecânica quântica não--relativística, conseguimos obter uma equação de onda, que é a equação de Schrödinger, no entanto esta equação não é válida ao considerarmos situações onde as partículas possuem caráter relativísticos. Sendo assim, como podemos obter uma equação de onda relativística no ponto de vista da mecânica quântica? Para responder essa pergunta, podemos utilizar os operadores de momento e energia, e considerarmos uma onda plana, dada por
    \begin{equation*}
        \psi(\vb{r},t) = e^{i(\vb{p}\cdot\vb{r} - E t)},\ \text{onde} \begin{cases}\begin{aligned}
            \vb{k} &= \dfrac{\vb{p}}{\hbar} = \vb{p}\ (\text{no SN}) \\
            \omega &= \dfrac{E}{\hbar} = E\ (\text{no SN})
        \end{aligned}
        \end{cases}
    \end{equation*}

Vemos então que, em uma onda plana, os operadores de momento e energia são facilmente observados. Se utilizarmos que $\hat{p} = -i\vb{\nabla}$ e aplicarmos a $\psi(\vb{r},t)$, temos que
    \begin{align*}
        -i\vb{\nabla}\psi(\vb{r},t) &\eq -i(i\vb{p}) e^{i(\vb{p}\cdot\vb{r} - E t)} \\
        &\eq \vb{p} e^{i(\vb{p}\cdot\vb{r} - E t)}
    \end{align*}
implicando que podemos de fato utilizar a forma usual do operador $\hat{p}$ da mecânica quântica neste caso. No caso do operador de energia, se utilizarmos que $\hat{\mathcal{H}} = i\pdv{}{t}$ e aplicarmos na onda plana, obtemos
    \begin{align*}
        i\pdv{\psi(\vb{r},t)}{t} &\eq i(-i E)e^{i(\vb{p}\cdot\vb{r} - E t)} \\
        &\eq E e^{i(\vb{p}\cdot\vb{r} - E t)}
    \end{align*}
de modo que a forma usual do operador de energia da mecânica quântica também se aplica aqui. Recorrendo então à relação de dispersão entre energia e momento, eq. \eqref{eq: dispersion relation}, podemos fazer a mudança
    \begin{align*}
        E &\mapsto i\pdv{}{t} \Rightarrow E^2 = -\pdv[2]{}{t} \\
        \vb{p} &\mapsto -i\vb{\nabla} \Rightarrow p^2 = \nabla^2
    \end{align*}
e através da \eqref{eq: dispersion relation} obtemos
    \begin{equation*}
        E^2 - p^2 = m^2 \Rightarrow -\pdv[2]{}{t} + \nabla^2 = m^2
    \end{equation*}
onde aqui o que obtemos o operador D'alembertiano, cujo autovalor é $-m^2$ (o sinal negativo vem por conta de que a equação acima possui os sinais trocados ao do operador). Sendo assim, ao aplicarmos este operador à onda plana, obtemos a famigerada equação de \textcite{Klein}--\textcite{Fock}--\textcite{Gordon}
    \begin{answer}\label{eq: KFG equation 1}
        \pdv[2]{\psi(\vb{r},t)}{t} - \nabla^2\psi(\vb{r},t) = -m^2 \psi(\vb{r},t)
    \end{answer}
que utilizando uma notação mais interessante se reescreve na forma
    \begin{answer}\label{eq: KFG equation}
        (\partial_{\mu}\partial^{\mu} + m^2)\psi(\vb{r},t) = 0
    \end{answer}

Em mecânica quântica não--relativística, temos a densidade de probabilidade $\rho$ sendo dada por $\rho = \psi^{\ast}\psi = |\psi|^2$. Se buscarmos por uma equação de continuidade, precisamos de uma quantidade $\vb{J}$ sendo uma densidade de corrente associada em que pode ser obtida através da equação de Schrödinger.

Partindo então da equação de Schrödinger, temos uma equação para $\psi(\vb{r},t)\equiv\psi$ e uma análoga para $\psi^{\ast}(\vb{r},t)\equiv\psi^{\ast}$. Temos então:
    \begin{align*}
        \qty[
            -\dfrac{\hbar^2}{2m} \nabla^2 + V(\vb{r})
        ]\psi = i\hbar\pdv{\psi}{t} \qquad \& \qquad
        \qty[
            -\dfrac{\hbar^2}{2m} \nabla^2 + V(\vb{r})
        ]\psi^{\ast} = -i\hbar\pdv{\psi^{\ast}}{t}
    \end{align*}

Explicitando $\nabla^2 = \vb{\nabla}\cdot\vb{\nabla}$, fica mais fácil ver quando a equação da continuidade aparecer. As equações ficam então:
    \begin{align*}
        -\dfrac{\hbar^2}{2m} \vb{\nabla}\cdot\vb{\nabla}\psi + V(\vb{r})\psi
        &= i\hbar\pdv{\psi}{t} \\
        -\dfrac{\hbar^2}{2m} \vb{\nabla}\cdot\vb{\nabla}\psi^{\ast} + V(\vb{r})\psi^{\ast}
        &= -i\hbar\pdv{\psi^{\ast}}{t}
    \end{align*}

Multiplicando a equação de cima por $\psi^{\ast}$ pela esquerda, e multiplicando por $\psi$ na equação de baixo pela direita:
    \begin{align*}
        -\dfrac{\hbar^2}{2m} \vb{\nabla}\cdot[\psi^{\ast}(\vb{\nabla}\psi)] + V(\vb{r})\psi^{\ast}\psi
        &= i\hbar\psi^{\ast}\pdv{\psi}{t} \\
        -\dfrac{\hbar^2}{2m} \vb{\nabla}\cdot(\vb{\nabla}\psi^{\ast})\psi + V(\vb{r})\psi^{\ast}\psi
        &= -i\hbar\pdv{\psi^{\ast}}{t}\psi
    \end{align*}

Dividindo ambas por $-i\hbar$:
    \begin{align*}
        \dfrac{\hbar}{2mi} \vb{\nabla}\cdot[\psi^{\ast}(\vb{\nabla}\psi)] - \dfrac{1}{i\hbar}V(\vb{r})\psi^{\ast}\psi
        &= -\psi^{\ast}\pdv{\psi}{t} \\
        \dfrac{\hbar}{2mi} \vb{\nabla}\cdot(\vb{\nabla}\psi^{\ast})\psi - \dfrac{1}{i\hbar}V(\vb{r})\psi^{\ast}\psi
        &= \pdv{\psi^{\ast}}{t}\psi
    \end{align*}

Subtraindo a da esquerda pela da direita, os termos com potencial se cancelam e obtemos:
    \begin{equation*}
        \vb{\nabla}\cdot \qty{
            \dfrac{\hbar}{2mi}\qty[
                (\vb{\nabla}\psi^{\ast}) \psi - \psi^{\ast}(\vb{\nabla}\psi)
            ]
        } = 
        \pdv{\psi^{\ast}}{t}\psi + \psi^{\ast}\pdv{\psi}{t}
    \end{equation*}

É fácil ver que:
    \begin{equation*}
        \pdv{\psi^{\ast}}{t}\psi + \psi^{\ast}\pdv{\psi}{t} = \pdv{(\psi^{\ast}\psi)}{t}
    \end{equation*}

Portanto, podemos escrever uma equação de continuidade para $\psi^{\ast}\psi$:
    \begin{equation*}
        \pdv{(\psi^{\ast}\psi)}{t} + \vb{\nabla}\cdot\qty{
            \dfrac{\hbar}{2mi}\qty[
                \psi^{\ast}(\vb{\nabla}\psi) - (\vb{\nabla}\psi^{\ast})\psi
            ]
        } = 0
    \end{equation*}

Em que a densidade de corrente associada $\vb{J}$ é definida por
    \begin{answer}\label{eq: current density}
        \vb{J} \coloneqq \dfrac{\hbar}{2mi}\qty[
                \psi^{\ast}(\vb{\nabla}\psi) - (\vb{\nabla}\psi^{\ast})\psi
            ]
    \end{answer}
ou seja, podemos escrever em mecânica quântica não--relativística que
    \begin{answer}\label{eq: continuity equation}
        \pdv{|\psi|^2}{t} + \vb{\nabla}\cdot\vb{J} = 0
    \end{answer}

A pergunta que surge após este desenvolvimento, dado que partimos da equação de Schrödinger, é se $\rho \overset{?}{=} |\psi|^2$ se mantém ao partirmos da equação de Klein--Fock--Gordon para encontrar uma equação de continuidade.

Pela equação de Klein--Fock--Gordon, temos duas formas:
    \begin{align*}
        \pdv[2]{\psi(\vb{r},t)}{t} - \nabla^2\psi(\vb{r},t) &= -m^2 \psi(\vb{r},t) \\
        \pdv[2]{\psi^{\ast}(\vb{r},t)}{t} - \nabla^2\psi^{\ast}(\vb{r},t) &= -m^2 \psi^{\ast}(\vb{r},t)
    \end{align*}

Multiplicando a primeira equação por $-i\psi^{\ast}$ e a de baixo por $i\psi$, ficamos com
    \begin{align*}
        i\pdv[2]{\psi}{t}\psi^{\ast} - i(\vb{\nabla}^2\psi)\psi^{\ast} &= -im^2|\psi|^2 \\
        -i\pdv[2]{\psi^{\ast}}{t}\psi + i(\vb{\nabla}^2\psi^{\ast})\psi &= im^2|\psi|^2
    \end{align*}

Somando as duas,
    \begin{equation}\label{eq: summing two equations}
        i\qty[\pdv[2]{\psi}{t} \psi^{\ast} - \pdv[2]{\psi^{\ast}}{t}\psi] - i\qty[(\nabla^2\psi)\psi^{\ast} - (\nabla^2\psi^{\ast})\psi] = 0
    \end{equation}

Note que o primeiro termo da equação pode ser representado de uma forma mais interessante, de modo que ao somarmos e subtrairmos a quantidade
    \begin{equation*}
        \pdv{\psi^{\ast}}{t} \pdv{\psi}{t} = \pdv{\psi}{t} \pdv{\psi^{\ast}}{t}
    \end{equation*}
teremos
    \begin{align}
        \pdv[2]{\psi}{t} \psi^{\ast} - \pdv[2]{\psi^{\ast}}{t}\psi &\eq 
        \pdv[2]{\psi}{t}\psi^{\ast} + \pdv{\psi}{t} \pdv{\psi^{\ast}}{t} - \pdv[2]{\psi^{\ast}}{t}\psi - \pdv{\psi}{t} \pdv{\psi^{\ast}}{t} \nonumber \\
        &\eq \pdv{}{t}\qty(\pdv{\psi}{t}\psi^{\ast} - \pdv{\psi^{\ast}}{t}\psi) \label{eq: another way to write 1}
    \end{align}
e de forma similar, podemos somar e subtrair no segundo termo a quantidade
    \begin{equation*}
        \vb{\nabla}\psi^{\ast} \cdot \vb{\nabla}\psi = \vb{\nabla}\psi \cdot \vb{\nabla}\psi^{\ast}
    \end{equation*}
ficamos com
    \begin{align}
        (\vb{\nabla}^2\psi)\psi^{\ast} - (\vb{\nabla}^2\psi^{\ast})\psi 
        &\eq (\nabla^2\psi)\psi^{\ast} + \vb{\nabla}\psi \cdot \vb{\nabla}\psi^{\ast} - (\nabla^2\psi^{\ast})\psi - \vb{\nabla}\psi^{\ast} \cdot \vb{\nabla}\psi \nonumber \\
        &\eq \vb{\nabla}\cdot \qty[(\vb{\nabla}\psi)\psi^{\ast} - (\vb{\nabla}\psi^{\ast})\psi] \label{eq: another way to write 2}
    \end{align}

Então ao substituirmos \eqref{eq: another way to write 1} e \eqref{eq: another way to write 2} na equação \eqref{eq: summing two equations} chegamos na equação
    \begin{equation*}
        \pdv{}{t}\qty[i\qty(\pdv{\psi}{t}\psi^{\ast} - \pdv{\psi^{\ast}}{t}\psi)] + \vb{\nabla}\cdot \qty{-i\qty[(\vb{\nabla}\psi)\psi^{\ast} - (\vb{\nabla}\psi^{\ast})\psi]} = 0
    \end{equation*}
que se uma equação de continuidade ao considerarmos que 
    \begin{equation*}
        \rho \coloneqq i\qty(\pdv{\psi}{t}\psi^{\ast} - \pdv{\psi^{\ast}}{t}\psi) \qquad \& \qquad 
        \vb{J} \coloneqq -i\qty[(\vb{\nabla}\psi)\psi^{\ast} - (\vb{\nabla}\psi^{\ast})\psi]
    \end{equation*}
ou seja, a densidade de probabilidade $\rho$ não é mais $|\psi|^2$ como no caso anterior.

Uma forma mais elegante de se escrever esta equação de continuidade é a partir do quadrivetor fluxo, construído na forma $j^{\mu} = (\rho,\vb{J})$ e definido por
    \begin{equation*}
        j^{\mu} = i[(\partial^{\mu}\psi)\psi^{\ast} - (\partial^{\mu}\psi^{\ast})\psi]
    \end{equation*}
de tal forma que a equação de continuidade fica
    \begin{answer}\label{eq: continuity equation by KFG}
        \partial_{\mu}j^{\mu} = 0
    \end{answer}

Mas o que significa $\rho \neq |\psi|^2$? Qual o impacto dese resultado? O fato de $\rho = i\qty(\pdv{\psi}{t}\psi^{\ast} - \pdv{\psi^{\ast}}{t}\psi)$ faz com que esse valor passa, inclusive, assumir valores negativos. Dessa forma, a grandeza não pode ser interpretada com uma densidade de probabilidade e a quantidade
    \begin{equation*}
        \mathds{P} = \int \rho\dd{\vb{r}} = \text{constante}
    \end{equation*}
também não pode ser interpretada como uma probabilidade. Mas por quê não podemos fazer este tipo de interpretação? Como, então, interpretaremos tais grandezas? Para responder isso, tomemos um caso simples de uma partícula livre descrita por uma onda plana da forma $\psi = Ne^{i(\vb{p}\cdot\vb{r} - Et)}$. A partir das definições de $\rho$ e $\vb{J}$, temos para este caso que
    \begin{align*}
        \rho &= i\qty[(-iE)|N|^2 - iE |N|^2] = 2E|N|^2 \\
        \vb{J} &= -i\qty[(i\vb{p})|N|^2 + i\vb{p}|N|^2] = 2\vb{p}|N|^2
    \end{align*}

Mas pela relação de dispersão \eqref{eq: dispersion relation} temos que
    \begin{equation*}
        E^2 = m^2 + p^2 \Rightarrow E = \pm \sqrt{m^2 + p^2}
    \end{equation*}
implicando na possibilidade de $E < 0$ e portanto possibilitando que $\rho$ possa ser negativo. Com isso em mente, se energias positivas e negativas são soluções da equação de Klein--Fock--Gordon, logo, a combinação linear delas também é solução. Logo, uma forma mais geral de escrever $\psi$ é através da forma
    \begin{equation*}
        \psi = N\qty[a e^{i(\vb{p}\cdot\vb{r} - |E|t)} + b e^{i(\vb{p}\cdot\vb{r} + |E|t)}]
    \end{equation*}
com a primeira parte sendo relativa à energias $E > 0$, a segunda para $E < 0$ e os fatores $a$ e $b$ são constantes de normalização. Com esta forma e através da definição de $\rho$, obtemos agora
    \begin{answer}
        \rho = 2|E| |N|^2 (|a|^2 - |b|^2)
    \end{answer}

Isso indica que a equação de Klein--Fock--Gordon não trada da solução para uma única partícula, mas sim para uma par de partículas, uma com $E>0$ e outra com $E<0$, cada uma com uma amplitude de probabilidade que depende de $|a|^2$ e $|b|^2$. 
Apesar deste resultado ser muito interessante, alguns problemas surgem em relação a $E<0$ e $\rho < 0$. O primeiro é o fato de que para $E<0$, ocorrem transições espontâneas e infinitos estados, em segundo lugar qual é o significado de $\rho < 0$? Pergunta esta que só foi respondida por \textcite{Dirac}. 

% Outro problema ocorre para partículas de spin 0, pois neste caso onde $p = 0$ temos que a energia mínima assume os valores
%     \begin{equation*}
%         E_{\text{min}} = \pm\sqrt{m^2 + 0^2} = \pm m
%     \end{equation*}
% ou seja, para partículas sem spin, a energia possui um limite inferior e superior, no entanto, não há limite superior para o momento, o que indica que não há limite superior negativo ou positivo da energia

        \section{A equação de Dirac}
            O fato da equação de Klein--Fock--Gordon ser de segunda ordem no tempo faz com que $\rho$ possa assumir valores negativos, como vimos anteriormente. Ao estudar sobre a teoria relativística dos elétrons, Dirac não ficou satisfeito com esta possibilidade, mesmo que houvessem ``remendos'' interpretativos em relação ao significado de $\rho < 0$. Além disso, a equação de Klein--Fock--Gordon não explica a existência do spin do elétron.

Afim de tentar solucionar estes problemas, Dirac buscou obter uma equação relativística de primeira ordem no tempo que incluísse soluções para diferentes spins do elétron. O que ele buscava era uma equação que satisfizesse as transformações de Lorentz, ou seja, se a equação é de primeira ordem no tempo, então ela também deve ser de primeira ordem no espaço, pois apenas assim as transformações serão satisfeitas.

Para encontrar tal equação, Dirac postulou uma hamiltoniana que fosse linear nas derivadas espaciais, mas que mantivesse a hermiticidade que a própria equação de Schrödinger possui. Esta equação é dada por
    \begin{equation}\label{eq: Dirac proposals}
        \hat{\mathcal{H}}\psi(\vb{r},t) = \hat{\boldsymbol{\alpha}}\cdot\vb{p} \psi(\vb{r},t) + \hat{\beta} m \psi(\vb{r},t)
    \end{equation}
onde $\psi(\vb{r},t)$ é um estado quântico, $\vb{\alpha}$ e $\beta$ operadores a serem determinados que devem ser ao mesmo tempo, independentes de $(\vb{p},t)$ por linearidade e independente de $(\vb{r},t)$ pela homogeneidade do espaço-tempo e $\vb{p}$ denota o vetor momento em 3 dimensões espaciais. Além disso, os operadores a serem determinados são adimensionais e devemos respeitar a equação \eqref{eq: dispersion relation}, tal que 
    \begin{equation}\label{eq: needs to satisfy this Dirac}
        \hat{\mathcal{H}}^{2} \psi(\vb{r},t) = (p^{2} + m^{2})\psi(\vb{r},t)
    \end{equation}

Podemos reescrever a equação \eqref{eq: Dirac proposals} na forma
    \begin{equation*}
        \hat{\mathcal{H}}\psi(\vb{r},t) = \sum_{i=1}^{3}\hat{\alpha}_{i}p_{i}\psi(\vb{r},t) + \hat{\beta} m \psi(\vb{r},t)
    \end{equation*}
de modo que ao fazer esta equação ao quadrado, temos a expansão
    \begin{align*}
        \hat{\mathcal{H}}^{2}\psi(\vb{r},t) &\eq \Bigg[
            \sum_{i=1}^{3}\hat{\alpha}_{i}^{2}p_{i}^{2} + 
            \sum_{i=1}^{3}(\hat{\alpha}_{i}\hat{\beta} + \hat{\beta}\hat{\alpha}_{i})p_{i}m + \\
        &\noeq
            \sum_{i=1}^{3}\sum_{j\neq i}(\hat{\alpha}_{i}\hat{\alpha}_{j} + \hat{\alpha}_{j}\hat{\alpha}_{i})p_{i}p_{j} + 
            \hat{\beta}^{2} m^{2}
        \Bigg]\psi(\vb{r},t) \\
        (\color{myLColor}\text{deve ser}) &\eq (p^{2} + m^{2})\psi(\vb{r},t)
    \end{align*}

Comparando então termo-a-termo, temos que
    \begin{align*}
        \sum_{i=1}^{3}\hat{\alpha}_{i}^{2}p_{i}^{2} \overset{!}{=} p^{2} &{\color{myLColor}\ \Rightarrow\ } \hat{\alpha}_{i}^{2} = 1 \\
        \sum_{i=1}^{3}(\hat{\alpha}_{i}\hat{\beta} + \hat{\beta}\hat{\alpha}_{i})p_{i}m \overset{!}{=} 0 &{\color{myLColor}\ \Rightarrow\ } \hat{\alpha}_{i}\hat{\beta} + \hat{\beta}\hat{\alpha}_{i} = \{\hat{\alpha}_{i},\hat{\beta}\} = 0 \\
        \sum_{i=1}^{3}\sum_{j\neq i}(\hat{\alpha}_{i}\hat{\alpha}_{j} + \hat{\alpha}_{j}\hat{\alpha}_{i})p_{i}p_{j} \overset{!}{=} 0 &{\color{myLColor}\ \Rightarrow\ } \hat{\alpha}_{i}\hat{\alpha}_{j} + \hat{\alpha}_{j}\hat{\alpha}_{i} = \{\hat{\alpha}_{i},\hat{\alpha}_{j}\} = 0 \\
        \hat{\beta}^{2}m^{2} \overset{!}{=} m^{2} &{\color{myLColor}\ \Rightarrow\ } \hat{\beta}^{2} = 1
    \end{align*}
e a partir destas equações/condições, é evidente que os operadores $\hat{\alpha}_{i}$ e $\hat{\beta}$ não podem ser números, mas sim \textbf{matrizes}. Matrizes estas que satisfazem relações de anti-comutação entre $\hat{\alpha}_{1},\ \hat{\alpha}_{2},\ \hat{\alpha}_{3},\ \hat{\alpha}_{3}$ e $\hat{\beta}$, além de que
    \begin{equation*}
        \hat{\alpha}_{1}^{2} = \hat{\alpha}_{2}^{2} = \hat{\alpha}_{3}^{2} = \hat{\beta} = \boldone
    \end{equation*}

    Para que tais condições sejam satisfeitas, os operadores $\hat{\alpha}_{i}$ e $\hat{\beta}$ devem ser matrizes hermitianas, de traço nulo e auto-valores $\pm 1$, cuja menor dimensão possível destas matrizes é 4, onde a demonstração deste último pode ser vista em detalhes em \textcite{Schiff}. Para demonstrar as outras 3 propriedades, precisamos de uma representação para esses operadores e nestas notas, a representação que será utilizada é a representação de Dirac-Pauli, dada por
        \begin{equation*}
            (\hat{\boldsymbol{\alpha}}) = \begin{pmatrix}
                0 & \hat{\boldsymbol{\sigma}} \\
                \hat{\boldsymbol{\sigma}} & 0
            \end{pmatrix} \qquad \& \qquad 
            (\hat{\beta}) = \begin{pmatrix}
                \boldone_{2\times 2} & 0 \\
                0 & -\boldone_{2\times 2}
            \end{pmatrix}
        \end{equation*}
    onde $\hat{\boldsymbol{\sigma}} = (\sigma_{1},\sigma_{2},\sigma_{3})$ são as matrizes de Pauli e $\boldone_{2\times2}$ a matriz identidade de dimensão 2.







    

    Agora, sabendo que podemos representar os operadores $\hat{\mathcal{H}} \mapsto i\pdv{}{t}$ e $\vb{p} = -i\vb{\nabla}$, temos que a equação \eqref{eq: Dirac proposals} se reescreve na forma
        \begin{equation*}
            i \pdv{\psi(\vb{r},t)}{t} = -i\hat{\boldsymbol{\alpha}}\cdot\vb{\nabla}\psi(\vb{r},t) + \beta m \psi(\vb{r},t)
        \end{equation*}

    Multiplicando ambos os lados por $\hat{\beta}$ à esquerda e usando o fato de que $\hat{\beta}^{2} = \boldone$, temos
        \begin{equation*}
            i\hat{\beta}\pdv{\psi(\vb{r},t)}{t} = -i\hat{\beta}\boldsymbol{\alpha}\cdot\vb{\nabla}\psi(\vb{r},t) + \boldone m\psi(\vb{r},t)
        \end{equation*}
    e portanto, sendo $\boldone m \mapsto m$ agora uma matriz com $\diag{m} = (m,m,m,m)$,
        \begin{equation*}
            \qty(i\hat{\beta}\pdv{}{t} + i\hat{\beta}\boldsymbol{\alpha}\cdot\vb{\nabla} - m)\psi(\vb{r},t) = 0
        \end{equation*}

    Definindo então o quadrivetor
        \begin{equation*}
            \gamma^{\mu} = (\hat{\beta}, \hat{\beta}\hat{\boldsymbol{\alpha}})
        \end{equation*}
    onde $\gamma^{\mu}$ será um quadrivetor com seus elementos sendo matrizes de dimensão 4, denominadas ``matrizes gamma'', podemos obter de fato a equação de Dirac em sua forma covariante:
        \begin{answer}\label{eq: Dirac equation 1}
            (i\gamma^{\mu}\partial_{\mu} - m)\psi(\vb{r},t) = 0
        \end{answer}

    Ou, através da notação ``slash'' de Feynman, onde ele define para um quadrivetor $A_{\mu}$ a seguinte notação: $\slashed{A} \coloneqq \gamma^{\mu}A_{\mu}$. Implicando em
        \begin{answer}\label{eq: Dirac equation 2}
            (i\slashed{\partial} - m)\psi(\vb{r},t) = 0
        \end{answer}

    Note então que a única forma da equação de Dirac ser satisfeita é que $\psi(\vb{r},t)$ seja um objeto de 4 componentes na forma
        \begin{equation*}
            \psi(\vb{r},t) \equiv \begin{pmatrix}
                \psi_{1}(\vb{r},t) \\
                \psi_{2}(\vb{r},t) \\
                \psi_{3}(\vb{r},t) \\
                \psi_{4}(\vb{r},t)
            \end{pmatrix}
        \end{equation*}

    Com tais informações em mente, o que podemos aprender com essa equação? Uma primeira forma de responder essa pergunta é analisando a solução de um dos casos mais simples possível: partículas livres. 

    Para uma partícula livre, podemos escrever simplesmente que
        \begin{equation}\label{eq: free particle for Dirac}
            \psi(\vb{r},t) = u e^{i(\vb{p}\cdot\vb{r} - Et)}
        \end{equation}
    porém, agora, $\psi(\vb{r},t)$ é um objeto com 4 componentes, o que nos dá
        \begin{equation*}
            \psi(\vb{r},t) = \begin{pmatrix}
                u_{1} \\
                u_{2} \\
                u_{3} \\
                u_{4}
            \end{pmatrix} e^{i(\vb{p}\cdot\vb{r} - Et)}
        \end{equation*}
    onde $u$ é um objeto de 4 componentes chamado \textbf{espinor}.

    Tomando como base a equação \eqref{eq: Dirac proposals} para facilitar a análise, temos
        \begin{align*}
            i\pdv{\psi(\vb{r},t)}{t} &\eq -i\hat{\boldsymbol{\alpha}}\cdot\vb{\nabla}\psi(\vb{r},t) + \hat{\beta}m\psi(\vb{r},t) \\
            i(-iE)\psi(\vb{r},t) &\eq -i\hat{\boldsymbol{\alpha}}\cdot(i\vb{p})\psi(\vb{r},t) + \hat{\beta} m \psi(\vb{r},t) 
        \end{align*}
    e portanto
        \begin{equation}\label{eq: eigenenergy free particle Dirac equation}
            Eu = (\hat{\boldsymbol{\alpha}}\cdot\vb{p} + \hat{\beta} m)u
        \end{equation}

    Olhando então para soluções de partículas no limite não-relativístico, ou seja $p\ll m$, temos
        \begin{equation*}
            Eu = \hat{\beta} m u = 
            \begin{pmatrix}
                \boldone_{2\times 2} m & 0 \\
                0 & -\boldone_{2\times 2} m
            \end{pmatrix} u = 
            \begin{pmatrix}
                m & 0 & 0 & 0 \\
                0 & m & 0 & 0 \\ 
                0 & 0 & -m & 0 \\
                0 & 0 & 0 & -m
            \end{pmatrix}u
        \end{equation*}

    É evidente então o fato de que tempos 4 auto-valores de energia $E = +m,\ +m,\ -m,\ -m$ e consequentemente 4 auto-estados
        \begin{equation*}
            u_{1} = \begin{pmatrix}
                1 \\ 0 \\ 0 \\ 0
            \end{pmatrix} \qquad \& \qquad 
            u_{2} = \begin{pmatrix}
                0 \\ 1 \\ 0 \\ 0
            \end{pmatrix} \qquad \& \qquad 
            u_{3} = \begin{pmatrix}
                0 \\ 0 \\ 1 \\ 0
            \end{pmatrix} \qquad \& \qquad 
            u_{4} = \begin{pmatrix}
                0 \\ 0 \\ 0 \\ 1
            \end{pmatrix}
        \end{equation*}
    onde associamos os 2 primeiros espinores $u_{1}$ e $u_{2}$ à energia $E = +m$, em que $u_{1}$ representa o espinor de uma partícula de energia $E = m$ com spin positivo e $u_{2}$ representa o espinor de uma partícula de energia $E = m$ com spin negativo. Analogamente, $u_{3}$ representa o espinor de uma partícula de energia $E = -m$ com spin positivo e $u_{4}$ o espinor de uma partícula de energia $E = -m$ com spin negativo.

    Agora passando para o caso relativístico, onde $p \sim m$, temos através da equação \eqref{eq: eigenenergy free particle Dirac equation} que
        \begin{equation*}
            Eu = \begin{pmatrix}
                \boldone_{2\times 2}m & \hat{\boldsymbol{\sigma}}\cdot\vb{p} \\
                \hat{\boldsymbol{\sigma}}\cdot\vb{p} & -\boldone_{2\times 2}m
            \end{pmatrix}u
        \end{equation*}

    Sendo então
        \begin{equation*}
            u = \begin{pmatrix}
                u_{p} \\ u_{d}
            \end{pmatrix}
        \end{equation*}
    podemos determinar os valores dos estados $u_{p}$ e $u_{d}$ por
        \begin{equation*}
            \begin{pmatrix}
                \boldone_{2\times 2}E & 0 \\
                0 & \boldone_{2\times 2}
            \end{pmatrix}
            \begin{pmatrix}
                u_{p} \\ u_{d}
            \end{pmatrix} = 
            \begin{pmatrix}
                \boldone_{2\times 2}m & \hat{\boldsymbol{\sigma}}\cdot\vb{p} \\
                \hat{\boldsymbol{\sigma}}\cdot\vb{p} & -\boldone_{2\times 2}m
            \end{pmatrix}
            \begin{pmatrix}
                u_{p} \\ u_{d}
            \end{pmatrix}
        \end{equation*}
    implicando em 
        \begin{equation*}
            \begin{cases}
                \boldone_{2\times 2}(E - m)u_{p} - \hat{\boldsymbol{\sigma}}\cdot \vb{p} u_{d} = 0 \\
                -\hat{\boldsymbol{\sigma}}\cdot\vb{p}u_{p} + \boldone_{2\times 2}(E + m)u_{d} = 0
            \end{cases}
        \end{equation*}

    A segunda equação nos dá que
        \begin{equation}\label{eq: equation for ud}
            u_{d} = \dfrac{\hat{\boldsymbol{\sigma}}\cdot\vb{p}}{E+m}u_{p}
        \end{equation}

    Substituindo isto na primeira equação e usando o fato de que $(\hat{\boldsymbol{\sigma}} - \vb{p})^{2} = p^{2}$, obtemos para $u_{p}$ que
        \begin{equation*}
            \boldone_{2\times 2}\qty[E^{2} - \qty(m^{2} + p^{2})]u_{p} = 0
        \end{equation*}
    onde as soluções possíveis são: o caso trivial de $u_{p} = 0$ e 
        \begin{equation*}
            E^{2} - (m^{2} + p^{2}) = 0 \Rightarrow E = \pm \sqrt{p^{2} + m^{2}}
        \end{equation*}

    O que é condizente com o que obtivemos utilizando a equação de Klein-Fock-Gordon! Olhando particularmente para o caso em que $E>0$ (onde o caso $E<0$ é análogo), podemos escolher por conveniência
        \begin{equation*}
            u_{p} = \begin{pmatrix}
                1 \\ 0
            \end{pmatrix} \qquad \text{ou} \qquad 
            u_{p} = \begin{pmatrix}
                0 \\ 1
            \end{pmatrix}
        \end{equation*}
    e utilizar a equação \eqref{eq: equation for ud} para obter duas formas para $u_{d}$:
        \begin{equation*}
            u_{d} = \dfrac{\hat{\boldsymbol{\sigma}}\cdot\vb{p}}{E+m}\begin{pmatrix}
                1 \\ 0
            \end{pmatrix} \qquad \text{ou} \qquad 
            u_{d} = \dfrac{\hat{\boldsymbol{\sigma}}\cdot\vb{p}}{E+m}\begin{pmatrix}
                0 \\ 1
            \end{pmatrix}
        \end{equation*}
    em que, como $\hat{\boldsymbol{\sigma}}\cdot\vb{p} = \sigma_{1}{p_{1}} + \sigma_{2}{p_{2}} + \sigma_{3}{p}_{3}$, temos explicitamente
        \begin{equation*}
            \hat{\boldsymbol{\sigma}}\cdot\vb{p} = \begin{pmatrix}
                {p}_{3} & {p}_{1} - ip_{2} \\
                {p}_{1} + i{p}_{2} & -{p}_{3}
            \end{pmatrix}
        \end{equation*}

    Concluindo então que
        \begin{equation*}
            u_{d} = \dfrac{1}{E+m}\begin{pmatrix}
                {p}_{3} \\
                {p}_{1} + i{p}_{2}
            \end{pmatrix} \qquad \text{ou} \qquad 
            u_{d} = \dfrac{1}{E+m}\begin{pmatrix}
                {p}_{1} - i{p}_{2} \\
                -{p}_{3}
            \end{pmatrix}
        \end{equation*}
    ou seja, as soluções de $u$ para $E > 0$ são
        \begin{equation*}
            u = \dfrac{1}{E+m}\begin{pmatrix}
                E+m \\ 0 \\ p_{3} \\ p_{1} + ip_{2}
            \end{pmatrix} \qquad \text{ou} \qquad 
            u = \dfrac{1}{E+m}\begin{pmatrix}
                0 \\ E+m \\ p_{1} - ip_{2} \\ -p_{3}
            \end{pmatrix}
        \end{equation*}

    Lembrando então da equação \eqref{eq: free particle for Dirac}, temos para $E > 0$ que os estados possíveis para partícula livre são
        \begin{equation*}
            \psi(\vb{r},t) = \dfrac{e^{i(\vb{p}\cdot\vb{r} - Et)}}{E+m}\begin{pmatrix}
                E+m \\ 0 \\ p_{3} \\ p_{1} + ip_{2}
            \end{pmatrix} \qquad \text{ou} \qquad 
            \psi(\vb{r},t) = \dfrac{e^{i(\vb{p}\cdot\vb{r} - Et)}}{E+m}\begin{pmatrix}
                0 \\ E+m \\ p_{1} - ip_{2} \\ -p_{3}
            \end{pmatrix}
        \end{equation*}

    No caso de $E < 0$, temos que
        \begin{equation*}
            u = \dfrac{1}{E-m}\begin{pmatrix}
                p_{3} \\
                p_{1} + ip_{2} \\
                E-m \\
                0
            \end{pmatrix} \qquad \text{ou} \qquad 
            u = \dfrac{1}{E-m}\begin{pmatrix}
                p_{1} - ip_{2} \\
                -p_{3} \\
                0 \\
                E - m
            \end{pmatrix}
        \end{equation*}
    e portanto os possíveis estados possíveis para partícula livre são
        \begin{equation*}
            \psi(\vb{r},t) = \dfrac{e^{i(\vb{p}\cdot\vb{r} - Et)}}{E-m}\begin{pmatrix}
                p_{3} \\
                p_{1} + ip_{2} \\
                E-m \\
                0
            \end{pmatrix} \qquad \text{ou} \qquad 
            \psi(\vb{r},t) = \dfrac{e^{i(\vb{p}\cdot\vb{r} - Et)}}{E-m}\begin{pmatrix}
                p_{1} - ip_{2} \\
                -p_{3} \\
                0 \\
                E - m
            \end{pmatrix}
        \end{equation*}

    Para determinar a equação da continuidade através da equação de Klein-Fock-Gordon, tomamos o conjugado complexo das equações e desenvolvemos as contas. No caso da equação de Dirac, o fato de termos matrizes e vetores, nos leva a fazer algo análogo a tomar o conjugado complexo, ou seja, tomar os hermitianos conjugados!

    A partir da definição das matrizes gamma e da construção dos operadores $\hat{\alpha}_{i}$ e $\hat{\beta}$, temos que
        \begin{equation}\label{eq: conjugated hermitian for gamma0}
            \gamma^{0} = \beta = \begin{pmatrix}
                \boldone_{2\times 2} & 0 \\
                0 & -\boldone_{2\times 2}
            \end{pmatrix} \Rightarrow (\gamma^{0})^{\dagger} = \beta = \gamma^{0}
        \end{equation}
        \begin{equation}
            \gamma^{i} = \hat{\beta}\hat{\alpha}_{i} = \begin{pmatrix}
                0 & \sigma_{i} \\
                -\sigma_{i} & 0
            \end{pmatrix} \Rightarrow (\gamma^{i})^{\dagger} = -\gamma^{i},\ i = 1,2,3
        \end{equation}

    Tendo então estas equações, temos primeiro que a equação de Dirac expandida é
        \begin{equation*}
            i\gamma^{0} \pdv{\psi(\vb{r},t)}{t} + i\sum_{k=1}^{3} \gamma^{k}\pdv{\psi(\vb{r},t)}{x^{k}} - m\psi(\vb{r},t) = 0
        \end{equation*}
    e em segundo que a equação de Dirac conjugada é
        \begin{equation*}
            -i\pdv{\psi^{\dagger}(\vb{r},t)}{t}\gamma^{0} - i\sum_{k=1}^{3} {\pdv{\psi^{\dagger}(\vb{r},t)}{t}} (-\gamma^{k}) - m\psi^{\dagger}(\vb{r},t) = 0
        \end{equation*}

    O sinal negativo em $(-\gamma^{k})$ é inconveniente se compararmos as duas equações, de modo que para resolver isso, multiplicamos por $\gamma^{0}$ pela direita e usemos o fato de que $\gamma^{k}\gamma^{0} = -\gamma^{0}\gamma^{k}$, tal que ficamos com
        \begin{equation*}
            -i\pdv{\psi^{\dagger}(\vb{r},t)}{t}\; \gamma^{0}\gamma^{0} - i\sum_{k=1}^{3} {\pdv{\psi^{\dagger}(\vb{r},t)}{t}}\gamma^{0}\gamma^{k} - m\psi^{\dagger}(\vb{r},t)\gamma^{0} = 0
        \end{equation*}

    Se definirmos $\bar{\psi}(\vb{r},t) \coloneqq \psi^{\dagger}(\vb{r},t)\gamma^{0}$, obtemos
        \begin{equation*}
            -i\pdv{\bar{\psi}(\vb{r},t)}{t}\gamma^{0} - \sum_{k=1}^{3}\pdv{\bar{\psi}(\vb{r},t)}{x^{k}}\gamma^{k} - m\bar{\psi}(\vb{r},t) = 0
        \end{equation*}
    que pode ser escrita de forma mais compacta como
        \begin{equation*}
            i\partial_{\mu}\bar{\psi}(\vb{r},t)\gamma^{\mu} + m \bar{\psi}(\vb{r},t) = 0
        \end{equation*}

    Multiplicando a equação de Dirac por $\bar{\psi}(\vb{r},t)$ e esta última equação por $\psi(\vb{r},t)$, temos, respectivamente
        \begin{equation*}
            i\bar{\psi}(\vb{r},t)\gamma^{\mu}\partial_{\mu}\psi(\vb{r},t) - m\bar{\psi}(\vb{r},t)\psi(\vb{r},t) = 0
        \end{equation*}
        \begin{equation*}
            i\partial_{\mu}\bar{\psi}(\vb{r},t)\gamma^{\mu}\psi(\vb{r},t) + m\bar{\psi}(\vb{r},t)\psi(\vb{r},t) = 0
        \end{equation*}
    e somando ambas,
        \begin{equation*}
            \bar{\psi}(\vb{r},t)\gamma^{\mu}[\partial_{\mu}\psi(\vb{r},t)] + [\partial_{\mu}\bar{\psi}(\vb{r},t)]\gamma^{\mu}\psi(\vb{r},t) = 0
        \end{equation*}

    Se definirmos $j^{\mu} \coloneqq \bar{\psi}(\vb{r},t)\gamma^{\mu}\psi(\vb{r},t)$, temos pela regra da cadeia que
        \begin{equation*}
            \partial_{\mu}j^{\mu} = \bar{\psi}(\vb{r},t)\gamma^{\mu}[\partial_{\mu}\psi(\vb{r},t)] + [\partial_{\mu}\bar{\psi}(\vb{r},t)]\gamma^{\mu}\psi(\vb{r},t) 
        \end{equation*}

    Ou seja, a equação acima é uma equação da continuidade
        \begin{answer}
            \partial_{\mu}j^{\mu} = 0
        \end{answer}
    para $j^{\mu} \coloneqq \bar{\psi}(\vb{r},t)\gamma^{\mu}\psi(\vb{r},t)$. Mas qual o impacto deste resultado? Tendo que $\rho = j^{0}$, temos também que
        \begin{align*}
            \rho &\eq \bar{\psi}(\vb{r},t)\gamma^{0}\psi(\vb{r},t) \\
            &\eq \psi^{\dagger}(\vb{r},t)\underbrace{\gamma^{0}\gamma^{0}}_{\boldone}\psi(\vb{r},t) \\
            &\eq \psi^{\dagger}(\vb{r},t)\psi(\vb{r},t) \\
            &\eq |\psi(\vb{r},t)|^{2}
        \end{align*}
    implicando então em
        \begin{answer}\label{eq: rho is a probability density again}
            \rho = j^{0} = |\psi(\vb{r},t)|^{2}
        \end{answer}
    ou seja, resgatamos a ideia de que $\rho$ é sempre positivo, o que vai equivaler a uma densidade de probabilidade, resolvendo o problema que a equação de Klein-Fock-Gordon possuía!

    \begin{note}{}
        Normalmente definimos $j^{\mu} \coloneqq \pm e\bar{\psi}(\vb{r},t)\gamma^{\mu}\bar{\psi}(\vb{r},t)$, onde $+e$ seria para partículas e $-e$ para antipartículas.
    \end{note}

    \begin{example}[Partícula de Dirac em um campo eletromagnético]
        Antes de tratar de fato as equações, salientamos aqui que faremos a abordagem utilizando o potencial vetor $\vb{A}(\vb{r})$ (associado ao campo magnético por $\vb{B}(\vb{r}) = \vb{\nabla}\times\vb{A}(\vb{r})$) e o potencial escalar $\phi(\vb{r})$ (associado ao campo elétrico por $\vb{E}(\vb{r}) = -\vb{\nabla}V(\vb{r})$). Além disso, fazemos um deslocamento no vetor momento $\hat{\vb{p}} \mapsto \hat{\vb{p}} - q\vb{A}$, pois esta mudança é invariante de gauge e nos permite determinar o momento cinético da partícula de forma mais exata. 

        Com estas ideias em mente, podemos passar de fato ao problema em questão. Utilizando as mudanças enunciadas, a equação \eqref{eq: Dirac proposals} se altera para forma
            \begin{equation*}
                \hat{\mathcal{H}}\psi(\vb{r},t) = \qty[\hat{\boldsymbol{\alpha}}\cdot(\hat{\vb{p}} - q\vb{A}) + \hat{\beta}m + q\phi]\psi(\vb{r},t)
            \end{equation*}
        que se estende para equação \eqref{eq: eigenenergy free particle Dirac equation} para assumir a forma
            \begin{equation}\label{eq: eigenenergy for a dirac particle}
                Eu = \qty[\hat{\boldsymbol{\alpha}}\cdot(\hat{\vb{p}} - q\vb{A}) + \hat{\beta}m + q\phi]u
            \end{equation}

        Sendo então
            \begin{equation*}
                \psi(\vb{r},t) \equiv \begin{pmatrix}
                    \psi_{u}(\vb{r},t) \\ \psi_{d}(\vb{r},t)
                \end{pmatrix}
            \end{equation*}
        podemos reescrever a equação \eqref{eq: eigenenergy for a dirac particle} por
            \begin{equation*}
                E \begin{pmatrix}
                    \psi_{u} \\ \psi_{d}
                \end{pmatrix} = \begin{pmatrix}
                    0 & \hat{\boldsymbol{\sigma}} \\
                    \hat{\boldsymbol{\sigma}} & 0
                \end{pmatrix}[\hat{\vb{p}} - q\vb{A}]
                \begin{pmatrix}
                    \psi_{u} \\ \psi_{d}
                \end{pmatrix} + 
                \begin{pmatrix}
                    m+q\phi & 0 \\
                    0 & -m+q\phi
                \end{pmatrix}
                \begin{pmatrix}
                    \psi_{u} \\
                    \psi_{d}
                \end{pmatrix}
            \end{equation*}

        Formamos então um sistema de equações tal que
            \begin{equation*}
                \begin{cases}
                    E\psi_{d} = \hat{\boldsymbol{\sigma}}\cdot(\hat{\vb{p}} - q\vb{A})\psi_{d} + (m + q\phi)\psi_{u} \\
                    E\psi_{u} = \hat{\boldsymbol{\sigma}}\cdot (\hat{\vb{p}} - q\vb{A})\psi_{u} - (m - q\phi)\psi_{d}
                \end{cases}
            \end{equation*}

        A partir da segunda equação, temos 
            \begin{equation*}
                \psi_{d}(\vb{r},t) = \dfrac{\hat{\boldsymbol{\sigma}}\cdot(\hat{\vb{p}} - q\vb{A})}{E - q\phi + m}\psi_{u}(\vb{r},t)
            \end{equation*}
        de modo que colocando esta expressão na primeira equação, acabamos com
            \begin{equation*}
                E\psi_{u}(\vb{r},t) = \qty{
                    \dfrac{[\hat{\boldsymbol{\sigma}}\cdot(\hat{\vb{p}} - q\vb{A})]^{2}}{E - q\phi + m} + m+q\phi
                }\psi_{u}(\vb{r},t)
            \end{equation*}

        No limite não-relativístico, ou seja, para $E\sim m$ e $E \gg q\phi$, temos que o denominador $E-q\phi+m \approx 2m$ e portanto
            \begin{equation*}
                E\psi_{u}(\vb{r},t) = \qty{
                    \dfrac{[\hat{\boldsymbol{\sigma}}\cdot(\hat{\vb{p}} - q\vb{A})]^{2}}{2m} + m + q\phi
                }\psi_{u}(\vb{r},t)
            \end{equation*}

        Utilizando a relação
            \begin{equation*}
                (\hat{\boldsymbol{\sigma}}\cdot\vb{u})(\hat{\boldsymbol{\sigma}}\cdot\vb{v}) = \vb{u}\cdot\vb{v} + 
                i\hat{\boldsymbol{\sigma}}\cdot(\vb{u}\times\vb{v})
            \end{equation*}
        obtemos
            \begin{align*}
                [\hat{\boldsymbol{\sigma}}\cdot(\hat{\vb{p}} - q\vb{A})]^{2} &\eq 
                [\hat{\boldsymbol{\sigma}}\cdot(\hat{\vb{p}} - q\vb{A})][\hat{\boldsymbol{\sigma}}\cdot(\hat{\vb{p}} - q\vb{A})] \\
                &\eq (\hat{\vb{p}} - q\vb{A})^2 + i\hat{\boldsymbol{\sigma}}\cdot(q\hat{\vb{p}}\times\vb{A} + q\vb{A}\times\hat{\vb{p}}) \\
                &\eq (\hat{\vb{p}} - q\vb{A})^2 + iq\hat{\boldsymbol{\sigma}}(\hat{\vb{p}}\times\vb{A} + \vb{A}\times\hat{\vb{p}})
            \end{align*}
        implicando em
            \begin{equation*}
                E\psi_{u}(\vb{r},t) = \qty[
                    \dfrac{(\hat{\vb{p}} - q\vb{A})^2}{2m} + \dfrac{iq\hat{\boldsymbol{\sigma}}}{2m}\cdot (\hat{\vb{p}}\times\vb{A} + \vb{A}\times\hat{\vb{p}}) + m + q\phi
                ]\psi_{u}(\vb{r},t)
            \end{equation*}

        Podemos simplificar esta expressão usando $\hat{\vb{p}} = -i\vb{\nabla}$ no termo intermediário, tal que
            \begin{align*}
                \hat{\vb{p}}\times\vb{A} + \vb{A}\times\hat{\vb{p}} &\eq -i\vb{\nabla}\times\vb{A} -i\vb{A}\times\vb{\nabla} \\
                &\eq -i\vb{B} - i\vb{A}\times\vb{\nabla}
            \end{align*}

        Note, porém, que tudo isto é aplicado a $\psi_{u}(\vb{r},t)$, de modo que ao tomarmos $\vb{A}\times\vb{\nabla}\psi_{u}(\vb{r},t)$ temos um termo identicamente nulo, pois $\vb{\nabla}\psi_{u}(\vb{r},t)$ vai ser paralelo ao potencial vetor, o que indica que a expressão se resume a
            \begin{equation*}
                E\psi_{u}(\vb{r},t) = \qty[
                    \dfrac{(\hat{\vb{p}} - q\vb{A})^2}{2m} + \dfrac{q\hat{\boldsymbol{\sigma}}}{2m}\cdot \vb{B} + m + q\phi
                ]\psi_{u}(\vb{r},t)
            \end{equation*}

        Neste limite não-relativístico, $E = m + E'$, logo 
            \begin{equation*}
                E'\psi_{u}(\vb{r},t) = \qty[\dfrac{1}{2m}(\hat{\vb{p}} - q\vb{A})^2 + q\phi]\psi_{u}(\vb{r},t) + \dfrac{q}{2m}(\hat{\boldsymbol{\sigma}}\cdot\vb{B})\psi_{u}(\vb{r},t)
            \end{equation*}

        Analisando esta expressão, temos no primeiro termo a estrutura da hamiltoniana não-relativística para partículas em um campo eletromagnético e no segundo termo uma quantidade que relaciona as matrizes de Pauli com o campo magnético. Então, definindo o spin como sendo $\hat{\vb{S}} = \dfrac{1}{2}\hat{\boldsymbol{\sigma}}$, caímos em
            \begin{equation*}
                E'\psi_{u}(\vb{r},t) = \qty[\dfrac{1}{2m}(\hat{\vb{p}} - q\vb{A})^2 + q\phi + \dfrac{q}{m}(\hat{\vb{S}}\cdot\vb{B})]\psi_{u}(\vb{r},t)
            \end{equation*}
        e assim vemos que a quantidade que conhecemos como spin aparece de forma automática, basicamente como uma propriedade puramente quântica intrínseca à característica relativística das partículas.
    \end{example}




\part{Desenhos (Apagar isso)}

\begin{center}
% Stern - Gerlach
    \begin{tikzpicture}[scale = 0.3846]
        \draw[fill = myLLColor!20] (19,8) -- (25,6) -- (25,11) -- (19,13) -- (19,8);
        \draw[->] (22,6) -- (22,13) node[right]{$z$};
        \draw[] (22-1/3,8+1/9) -- (22+1/3,8-1/9) node[right, rotate = -18.43]{$\mu_{-}$};
        \draw[] (22-1/3,11+1/9) -- (22+1/3,11-1/9) node[right, rotate = -18.43]{$\mu_{+}$};

    
        \draw[fill = myLColor!60] (6,3.66) -- (15,6.66) -- (14,7) -- (5,4) -- (6,3.66) -- cycle;
        \draw[fill = myLColor!80] (6,3.66) -- (15,6.66) -- (15,5.33) -- (6,2.33) -- (6,3.66) -- cycle;
        \draw[fill = myLColor!60] (6,2.33) -- (15,5.33) -- (16,5) -- (7,2) -- (6,2.33) -- cycle;
        
        \draw[fill = myLColor!80] (8,0) -- (17,3) -- (17,6) -- (8,3) -- (8,1) -- cycle;
        
        \draw[] (15,6.1) -- (22,8);
        \draw[] (15,6.1) -- (22,11);

        \draw[fill = myLLColor!90, xshift = -0.5 cm, yshift = -0.7cm] (6,7) -- (9,6) -- (18,9) -- (15,10) -- (6,7) -- cycle;
        \draw[fill = myLLColor!70, xshift = -0.5 cm, yshift = -0.7cm] (18,8.33) -- (9,5.33) -- (9,6) -- (18,9) -- (18,8.33) -- cycle;
        \draw[fill = myLLColor, xshift = -0.5 cm, yshift = -0.7cm] (7,4) -- (16,7) -- (18,8.33) -- (9,5.33) -- (7,4) -- cycle;
        \draw[fill = myLLColor!90, xshift = -0.5 cm, yshift = -0.7cm] (7,4) -- (6,6.33) -- (6,7) -- (9,6) -- (9,5.33) -- (7,4) -- cycle;
        
        \draw[fill = myLColor!60] (7,3.33) -- (8,3) -- (17,6) -- (16,6.33) -- (7,3.33) -- cycle;
        \draw[fill = myLColor!80] (5,4) -- (5,1) -- (8,0) -- (8,3) -- (7,3.33) -- (7,2) -- (6,2.33) -- (6,3.66) -- (5,4) -- cycle;

        \node[rotate = -18.43] at (6.75,5){S};
        \node[rotate = -18.43, white] at (6.4,1.5){N};

        \draw[->] (0,1) -- (1,1.33);
        \draw[->] (1,2) -- (2,2.33);
        \draw[->] (1,1) -- (2,1.33);
        \draw[->] (2,2) -- (3,2.33);
        
        \draw[fill = myLColor] (0,1) circle (2pt);
        \draw[fill = myLColor] (1,2) circle (2pt);
        \draw[fill = myLColor] (1,1) circle (2pt);
        \draw[fill = myLColor] (2,2) circle (2pt);
    \end{tikzpicture}
\end{center}

\divider

\begin{center}
    \begin{tikzpicture}
        \draw[thick] (0,0) ellipse (2cm and 0.6cm);

        \draw[->, myLLColor, thick] (0,-0.6) -- (2,-0.6) node[right]{$I$};
        \draw[->, myLColor!50, thick] (0,0) -- (0,1.5) node[right]{$\vb{\mu}$};
        \draw[<->] (0,0) -- (2,0) node[midway, above]{$R$};
    \end{tikzpicture}
\end{center}

\divider

\begin{center}
    \begin{tikzpicture}
        \draw[fill = myLLColor] plot[domain=0:1, samples = 300] (\x,{1.5*exp(-6.5*((\x - 1)^2)}) -- (6,1.5) -- plot[domain=6:7, samples = 300](\x,{1.5*exp(-6.5*((\x - 6)^2)});
        \draw[fill = myLColor!50] plot[domain=0:7, samples = 300](\x,{4*exp(-70*0.4*((\x - 1)^2))});
        \draw[fill = myLColor!50] plot[domain=0:7, samples = 300](\x,{4*exp(-70*0.4*((\x - 6)^2))});
        \draw[step = 0.5cm, draw opacity = 0.15] (0,0) grid (7,4);
        \draw[<->] (0,4) node[above]{n° de partículas} |- (7,0) node[right]{$z$};
        \node[below] at (1,0){$\mu_{-}$};
        \node[below] at (6,0){$\mu_{+}$};

        \node[below] at (3.5,1.5){Curva esperada};
        \node[below] at (3.5,3.5){Curva obtida};
        \draw[->] (4.75,3.25) -- (5.5,3.25);
        \draw[<-] (1.5,3.25) -- (2.25,3.25);
    \end{tikzpicture}
\end{center}

\divider

\begin{center}
    \begin{tikzpicture}
        \draw[fill = myLColor!50] (0,0) rectangle (1.2,1.2);
        \node[] at (0.6,0.6){$x$};
        \node[above] at (0.6,1.2){Medidor de $\hat{\mu}_{x}$};

        \draw[fill = myLColor!50, xshift = 4cm] (0,0) rectangle (1.2,1.2);
        \node[, xshift = 4cm] at (0.6,0.6){$y$};
        \node[above, xshift = 4cm] at (0.6,1.2){Medidor de $\hat{\mu}_{y}$};

        \draw[fill = myLColor!50, xshift = 8cm] (0,0) rectangle (1.2,1.2);
        \node[, xshift = 8cm] at (0.6,0.6){$z$};
        \node[above, xshift = 8cm] at (0.6,1.2){Medidor de $\hat{\mu}_{z}$};
    \end{tikzpicture}
\end{center}

\divider

\begin{center}
    \begin{tikzpicture}
        \draw[fill = myLColor!50] (0,0) rectangle (1.2,1.2);
        \node[] at (0.6,0.6){$z$};

        \draw[->] (-2,0.6) -- (0,0.6) node[midway, above]{$\ket{\psi} =~ ?$};
        \draw[->] (1.2,0.3) -- (3.2,0.3) node[midway, below]{$50\%~ \ket{-}_{z}$};
        \draw[->] (1.2,0.9) -- (3.2,0.9) node[midway, above]{$50\%~ \ket{+}_{z}$};
    \end{tikzpicture}
\end{center}

\divider

\begin{center}
    \begin{tikzpicture}
        \draw[fill = myLColor!50] (0,0) rectangle (1.2,1.2);
        \node[] at (0.6,0.6){$z$};

        \draw[->] (-2,0.6) -- (0,0.6) node[midway, above]{$\ket{\psi} =~ ?$};
        \draw[->] (1.2,0.3) -- (3.2,0.3) node[midway, below]{$50\%~ \ket{-}_{z}$};
        \draw[->] (1.2,0.9) -- (3.4,0.9) node[midway, above]{$50\%~ \ket{+}_{z}$};

        \draw[fill = myLColor] (3.2,0) rectangle (3.3,0.6);

        \draw[fill = myLColor!50] (3.4,0) rectangle (4.6,1.2);
        \node[] at (4,0.6){$z$};
    \end{tikzpicture}
\end{center}

\divider 

\begin{center}
    \begin{tikzpicture}
        \draw[fill = myLColor!50] (0,0) rectangle (1.2,1.2);
        \node[] at (0.6,0.6){$z$};

        \draw[->] (-2,0.6) -- (0,0.6) node[midway, above]{$\ket{\psi} =~ ?$};
        \draw[->] (1.2,0.3) -- (3.2,0.3) node[midway, below]{$50\%~ \ket{-}_{z}$};
        \draw[->] (1.2,0.9) -- (3.4,0.9) node[midway, above]{$50\%~ \ket{+}_{z}$};

        \draw[fill = myLColor] (3.2,0) rectangle (3.3,0.6);

        \draw[fill = myLColor!50] (3.4,0) rectangle (4.6,1.2);
        \node[] at (4,0.6){$z$};

        \draw[->] (4.6,0.6) -- (6.6,0.6) node[midway, above]{$100\%~ \ket{+}_{z}$};
    \end{tikzpicture}
\end{center}

\divider

\begin{center}
    \begin{tikzpicture}
        \draw[fill = myLColor!50] (0,0) rectangle (1.2,1.2);
        \node[] at (0.6,0.6){$z$};

        \draw[->] (-2,0.6) -- (0,0.6) node[midway, above]{$\ket{\psi} =~ ?$};
        \draw[->] (1.2,0.3) -- (3.2,0.3) node[midway, below]{$50\%~ \ket{-}_{z}$};
        \draw[->] (1.2,0.9) -- (3.4,0.9) node[midway, above]{$50\%~ \ket{+}_{z}$};

        \draw[fill = myLColor] (3.2,0) rectangle (3.3,0.6);

        \draw[fill = myLColor!50] (3.4,0) rectangle (4.6,1.2);
        \node[] at (4,0.6){$x$};

        \draw[->] (4.6,0.3) -- (6.6,0.3) node[midway, below]{$50\%~ \ket{-}_{x}$};
        \draw[->] (4.6,0.9) -- (6.6,0.9) node[midway, above]{$50\%~ \ket{+}_{x}$};
    \end{tikzpicture}
\end{center}

\divider

\begin{center}
    \begin{tikzpicture}
        \foreach \y in {1,0.6,...,-1}{
        \draw[->, myDColor](0,\y) node[left]{$\overline{\nu}$} -- (1,\y);
    }
    
    \draw[fill = myDColor!20] (1.5,-2) rectangle (6.8,2);
    
    \foreach \y in {1.8,1.6,...,-1.8}{
        \draw[ball color = myDColor!50] (1.5,\y) circle (2pt);
        \draw[ball color = myDColor!50] (6.8,\y) circle (2pt);
    }
    
    \draw[->] (2.2,0) -- (3,0);
    
    \node[myDColor!50] at (4.0,0){(1)};
    
    \draw[ball color = blue!10!black] (3.2,0) circle (2pt) node[below]{$p$};
    
    \draw[->,] (3.4,0.1) -- (4,1);
    \draw[->,] (3.4,-0.1) -- (4,-1);
    
    \draw[ball color = blue!50!black] (4.1,1.1) circle (1pt) node[above]{$e^{+}$};
    \draw[ball color = blue!40!black] (4.1,-1.2) circle (2pt) node[below]{$n$};
    
    \draw[myDColor!80, dotted] (4.33,2) -- (4.33,-2);
    
    \draw[->] (4.4,1.1) -- (5.1,1.1) node[midway, below, myDColor!50]{(2)};
    \draw[ball color = blue!50!black] (5.3,1.1) circle (1pt) node[above]{$e^{-}$};
    
    \draw[->] (4.4,-1.2) -- (5.1,-1.2) node[midway, above, myDColor!50]{(3)};
    \draw[ball color = blue!40!black] (5.3,-1.2) circle (5pt) node[below=2.5]{\footnotesize{$^{108}\textrm{Cd}$}};
    
    \draw[decorate, decoration=snake, myLLColor!70!myLColor, ->,]  (5.5,1.0) - - (6.6,0.4) node[midway, below=0.1] {$\gamma$};
    \draw[decorate, decoration=snake, myLLColor!70!myLColor, ->,]  (5.5,1.2) - - (6.6,1.8) node[midway, above=0.1] {$\gamma$};
    \draw[decorate, decoration=snake, myLLColor!70!myLColor, ->,]  (5.5,-1.2) - - (6.7,-1.2) node[midway, below=0.1] {$\gamma$};
    
    \draw[->] (5.3,-1.0) -- (5.3, -0.4);
    
    \draw[ball color = blue!40!black] (5.3, -0.2) circle (5pt) node[right=2]{$^{109}\textrm{Cd}$};
    
    \foreach \y in {1.8,1.6,...,-1.8}{
        \draw[] (1.43,\y) -- (1.3,\y) -- (1.3,\y-0.2);
        \draw[] (6.87,\y) -- (7.02,\y) -- (7.02,\y-0.2);
    }
    
    \draw[arrow inside, myLLColor!70!myLColor, thick] (6.87,1.8) -- (7.02,1.8) -- (7.02,0.4);
    \draw[arrow inside, myLLColor!70!myLColor, thick] (6.87,0.4) -- (7.02,0.4) -- (7.02,-1.2);
    
    \draw[ball color = blue!40!black] (2,0) circle (0.5pt) node[below]{$\overline{\nu}$};
    
    \draw[] (1.3,-2) -- (1.3,-2.2) -- (7.02,-2.2) -- (7.02,-2);
    
    \draw[->, myDColor!50] (1.58,-1.8) -- (1.72,-1.8) node[right]{\footnotesize{fotomultiplicador}};
    
    \draw[arrow inside, myLLColor!70!myLColor, thick] (6.87,-1.2) -- (7.02,-1.2) -- (7.02,-2.2);
    \draw[arrow inside, myLLColor!70!myLColor, thick] (7.033,-2.2) -- (4.6,-2.2) node[midway,below]{$i$};
    
    \foreach \y in {1.8,0.4,-1.2}{
        \draw[ball color = green!40] (6.8,\y) circle (2pt);
    }
    
    \draw[fill = myDColor!50] (4,-2.3) rectangle (4.6,-2.1) node[midway,below=0.1]{Analisador};
    
    \node[right] (r1) at (7.2,1.8){(1) $\overline{\nu} + p \rightarrow n + e^{+}$};
    \node[right] (r2) at (7.2,0){(2) $e^{+} + e^{-} \rightarrow \gamma + \gamma$};
    \node[right] (r3) at (7.2,-1.8){(3) $n + ^{108}\textrm{Cd} \rightarrow ^{109}\textrm{Cd} + \gamma$};
    \end{tikzpicture}
\end{center}

\divider

\begin{center}
    \begin{tikzpicture}
        \draw[step = 0.5 cm, draw opacity = 0.15] (0,0) grid (5.5,5.5);
        \draw[<->] (0,5.5) node[above]{$\mathds{P}$} |- (5.5,0) node[right]{$t$};

        \draw[myLColor!50] plot[domain=0:5.5, samples=300](\x,{2.5*(sin(\x r))^2}) node[right]{$\mathds{P}(\nu_{\mu})$};
        \draw[myLLColor] plot[domain=0:5.5, samples=300](\x,{2.5*(cos(\x r))^2 + 2.5}) node[right]{$\mathds{P}(\nu_{e})$};

        \draw[dotted, thick] (0,2.5) node[left]{$(sc)^2$} -- (5.5,2.5);
        \node[left] at (0,5){$1$};
    \end{tikzpicture}
\end{center}

\divider

\begin{center}
    \begin{tikzpicture}
        \draw[step = 0.5cm, draw opacity = 0.15] (0,1.5) grid (6,-4.5);
        \draw[->] (0,-4.5) -- (0,1.5) node[left]{$V(x)$};
        \draw[->] (0,0) -- (6,0) node[right]{$x$};

        \draw[myLColor!50] plot[domain=1.38:6, samples=300](\x,{4*4*( (1.4/(\x))^(12) - (1.4/(\x))^(6) )});
        \draw[dotted] plot[domain=1.268:1.873, samples=300](\x,{60*(\x - 1.57)^2 - 4});
        \node[above] at (1.57,0){$\;x_{0}$};
        \draw[] (1.57,0.12) -- (1.57,-0.12);
    \end{tikzpicture}
\end{center}


\nocite{*}

\end{document}