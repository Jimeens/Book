Classicamente, quando consideramos um sistema em um certo instante de tempo $t$ e sabemos como descrever a posição $\vb{r}$ e a velocidade $\vb{v}$, conseguimos caracterizar completamente o estado da partícula. 
    \begin{example}\label{exemple 1.1}
        Suponha que uma partícula clássica esteja em $t_{0}$ na posição $\vb{r}_{0}$ com velocidade $v_{0}=0$ e aceleração constante $\vb{a}$. Sabendo que essa partícula esteja com $\vb{v}=v_{x}\vb{e}_{x}+v_{y}\vb{e}_{y}+v_{z}\vb{e}_{z}$ em $t\neq t_{0}$, qual será o vetor $\vb{r}$ no instante de tempo $t$?
        
        \divider
        
        Utilizando a equação horária do movimento, podemos determinar a posição da partícula facilmente, tal que
            \begin{equation*}
                \vb{r}(t) = \vb{r}_{0} + \vb{v}_{0}t + \dfrac{1}{2}\vb{a}t^2
            \end{equation*}
    \end{example}
    
Em sistemas quânticos a situação é outra. A variável temporal $t$ torna-se um objeto matemático vinculado a uma distribuição de probabilidade, e esta distribuição está diretamente relacionada ao \textbf{vetor de estado} $\ket{\psi}$.

Resumidamente um vetor de estado $\ket{\psi}$ é um vetor em um espaço de Hilbert que armazena todas as informações do estado da partícula associada a ele. Para entender um pouco melhor, vamos entender o que é um \Acomment{espaço de Hilbert}.

