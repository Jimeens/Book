Ao passarmos por diversas partes da mecânica quântica, podemos finalmente tratar de uma das partes mais interessantes a ser estudada e que gera uma diversidade de descrições mais avançadas, como por exemplo a Teoria Quântica de Campos. Nesta capítulo, trataremos inicialmente de uma básica revisão dos conceitos de relatividade \textit{restrita}, seguida de uma dedução minuciosa da famosa equação de equação de Klein\footnote{Oskar Benjamin Klein (1894--1977).}--Fock\footnote{Vladimir Aleksandrovich Fock (1898--1974).}--Gordon\footnote{Walter Gordon (1893--1939).} para bósons relativísticos e em seguida, um desenvolvimento apropriado para descrição de férmions relativísticos, através da também famosa equação de Dirac\footnote{Paul Adrien Maurice Dirac (1902--1984).}

A relatividade restrita, desenvolvida em seus primórdios por \textcite{Einstein1}, é parte de um dos tópicos mais importantes da física, juntamente com a relatividade geral. Em essência, a relatividade restrita descreve a estrutura básica do espaço--tempo (onde o tempo e o espaço são equivalentes), tal que a teoria não depende da escolha de nenhum referencial \textit{inercial}, em que este fato se resume na chamadas \textit{transformações de Lorentz}\footnote{Hendrik Antoon Lorentz (1853--1928).}, no entanto, a mecânica quântica \textit{não}-relativística possui limitações que fazem com que tais requisitos não sejam satisfeitos. Um exemplo simples disso é o fato de que o tempo e o espaço são assimétricos, o que podemos ver na própria equação de Schrödinger, em que a derivada no tempo é de primeira ordem e a derivada no espaço é de segunda ordem.

A conciliação total entre a mecânica quântica e a relatividade restrita é a chamada ``teoria quântica de de campos'', ou simplesmente TQC, no entanto, este tópico exige uma maturidade física e matemática muito superior ao que de fato abordamos até agora, o que nos leva a querer abordar o assunto de uma forma menos radical, ou seja, a partir de mudanças sutis encontrar novas equações onde o espaço e o tempo sejam tratados de forma mais equilibrada. A construção destas novas equações acabam tendo consequências cruciais na teoria, como por exemplo o caso de antipartículas e a relação do spin com o momento magnético.

No entanto, antes de realmente tratarmos da parte relativística, é conveniente tratar os sistemas estudados em sistemas de unidades diferentes do SI. Isto não é feito por uma necessidade, mas sim pela utilidade, pois na prática, o desenvolvimento teórico se simplifica de forma imensurável ao tratá-lo com um sistema de unidades diferente.